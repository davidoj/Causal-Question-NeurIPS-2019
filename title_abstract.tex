\section{Titles}

How to ask a Causal Question
Statistical Causal Decision Problems
A Data-Driven Decision Theoretic Approach to Causality
Posing Causal Problems without Causation


\section{Abstract}

There are multiple competing approaches to the handling of causality in statistical inference including Causal Bayesian Networks and Potential Outcomes which differ in part in their underlying conceptions of causality. In an approach similar to \cite{dawid_decision-theoretic_2012}, we develop the notion of a statistical causal decision problem patterned after the statistical decision theory of \cite{wald_statistical_1950}. Our approach is motivated by a simple consideration: suppose we have a dataset, some set of available decisions and we know what state we would like the world to occupy, but we are uncertain about how our decisions affect the state of the world. We introduce the notion of consequence maps that relate decisions to states of the world and causal theories that relate observations to consequences. These definitions are not motivated by any causal considerations, but by the need to connect observations, decisions and consequences. We connect statistical causal decision problems to statistical decision problems via a reduction that allows results from the latter to sometimes be imported to the former. We show that Causal Bayesian Networks and Potential Outcomes both have a natural mapping to causal theories, and demonstrate a straightforward example of a causal theory that cannot be unambiguously represented by either. We argue by example that, given this more general perspective, the standard understanding of a Causal Bayesian Network is only justified under additional nontrivial assumptions. Finally, we conclude with a long list of open questions raised by this new approach.