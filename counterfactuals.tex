
\section{Potential Outcomes}

Potential Outcomes is a major rival to the approach typified by Causal Bayesian Networks for formulating causal questions and hypotheses. Causal queries in the Potential Outcomes framework concern the distribution of random variables $\RV{Y}_0, \RV{Y}_1$ representing potential outcomes, or ``the value $\RV{Y}$ would take if action 0 or 1 were taken respectively'' (\cite{angrist_mastering_2014}). 

Queries based on the Potential Outcomes framework concern properties of the joint distribution over observations and potential outcome variables - that is, they resemble ordinary statistical decision problems. If, as we have supposed, our starting point is preferences over outcomes then the identification of a causal effect in this framework does not automatically allow for us to determine preferences over decisions, as the Potential Outcomes framework does not formally specify a canonical consequence map given a particular joint distribution over potential outcome variables. 

We do have an informal notion of how this should be constructed - namely that the potential outcome $\RV{Y}_1$ is the value $\RV{Y}$ would take if action 1 were taken. We can formalise this with the notion of consequence consistency. Let $\Delta(\mathcal{Y}_\circ)$ be the space of joint distributions over real and potential outcomes of $\RV{Y}$. A consequence map $\kappa:D\to \Delta(\mathcal{Y}_\circ)$ is consequence consistent if
\begin{align}
    \kappa(i;(\RV{Y}_i-\RV{Y})^{-1}(A))=\delta_0(A) \label{eq:oc_consist}
\end{align} 
Consequence consistency is similar to the consistency condition (e.g., see \cite{richardson2013single}).

A causal theory $\mathscr{T}^{cc}$ that is consequence consistent need not require any particular relationship between an ``observed'' distribution $\mu\in \Delta(\mathcal{Y}_\circ)$ and an associated consequence $\kappa$. A straightforward option for making this connection is to assert equality of distributions; a causal theory $\mathscr{T}^{cc,eq}$ is consequence consistent and counterfactually equal if for all $(\mu,\kappa)\in\mathscr{T}^{cc,eq}$
\begin{align}
    \kappa(i;\RV{X}_i^{-1}(A))=\mu(\RV{X}_i^{-1}(A))
\end{align}
There are many choices that could be made here, as can be seen in Example \ref{ex:nonst_distn}.

\begin{example}[Performance bias]\label{ex:nonst_distn}
Suppose we have a randomised experiment $(E,\mathcal{E},\mu)$ with random variables $\RV{X}$, $\RV{X}_0$ and $\RV{X}_1$ and have determined $\mathbb{E}_\mu[\RV{X}_0]$ and $\mathbb{E}_\mu[\RV{X}_1]$ via ignorability. Consequence consistency is assumed, but performance bias is suspected - that is, we believe the value of $\delta_i \kappa \RV{X}_i$ may differ from $\RV{X}_i$ (performance bias is defined by \cite{collaboration_cochrane_nodate}). There are many ways to handle this suspicion:
\begin{itemize}
    \item Assume performance bias is absent $\delta_i \kappa \RV{X}_i = \mathbb{E}_\mu[\RV{X}_i]$
    \item Assume performance bias has a uniform additive effect: $\delta_i \kappa \RV{X}_i = \mathbb{E}_\mu[\RV{X}_i]+k$; in this case the average treatment effect is still known $\delta_1 \kappa \RV{X}_1 - \delta_0\kappa\RV{X}_0 = \mathbb{E}[\RV{X}_1]-\mathbb{E}[\RV{X}_0]$
    \item Avoid assumptions about the effect of performance bias; then there is no particular relationship between $\mu F_{\RV{X}_i}$ and $\delta_i \kappa F_{\RV{X}_i}$.
\end{itemize}

\end{example}

