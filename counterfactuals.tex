
\section{Potential Outcomes}

Potential Outcomes is an alternative to the approach typified by Causal Bayesian Networks for formulating causal questions and hypotheses. Causal queries in the Potential Outcomes framework concern the distribution of random variables $\RV{Y}_0, \RV{Y}_1$ representing potential outcomes, or ``the value $\RV{Y}$ would have taken if action 0 or 1 were taken respectively'' (\cite{angrist_mastering_2014}). This is similar, but not the same, as the question answered by a consequence map which is ``what is the distribution of $\RV{Y}$ if I take actions 0 or 1?''

A natural connection between these informal notions of potential outcomes and consequence maps is given by the notion of consequence consistency. Let $\Delta(\mathcal{Y}_\circ)$ be the space of joint distributions over real and potential outcomes of $\RV{Y}$. A consequence map $\kappa:D\to \Delta(\mathcal{Y}_\circ)$ is consequence consistent if
\begin{align}
    \kappa F_{\RV{Y}_i-\RV{Y}} (i;A)=\delta_0(A) \label{eq:oc_consist}
\end{align} 
Consequence consistency is similar to the consistency condition (e.g., see \cite{richardson2013single}), but the latter does not involve consequences.

A causal theory $\mathscr{T}^{cc}$ that is consequence consistent need not require any particular relationship between an ``observed'' distribution $\mu\in \Delta(\mathcal{Y}_\circ)$ and an associated consequence $\kappa$. Example \ref{ex:nonst_distn} shows that there are multiple choices that may be appropriate for making this connection.

\begin{example}[Performance bias]\label{ex:nonst_distn}
Suppose we have a CSDP $\langle (\mathscr{T}, E), D, \RV{X}, (U,E) \rangle$ where the observed data $\RV{X}$ is from a randomised controlled trial (RCT), $\RV{Y}_0:E\to Y$ and $\RV{Y}_1:E\to Y$ are random variables representing a particular outcome of interest under no treatment and treatment respectively and $\RV{Y}:E\to Y$ represents the ``realised'' outcome of interest and for $\xi\in \Delta(\mathcal{E})$, $U(\xi) = \mathbb{E}_{\xi}[\RV{Y}]$. 

Under usual assumptions about RCTs, if we suppose the observed data are distributed according to $\mu \in \Delta(\mathcal{E})$ it is possible (given infinite data $\RV{X}$) to determine $\mathbb{E}_\mu[\RV{Y}_0]$ and $\mathbb{E}_\mu[\RV{Y}_1]$ \citep{rubin_causal_2005}.

Consequence consistency is assumed, but performance bias is suspected, which can lead to $\delta_i \kappa \RV{Y}_i$ differing from $\mathbb{E}_\mu[\RV{Y}_i]$ \citep{mansournia_biases_2017}.
\begin{enumerate}
    \item Assume performance bias is absent, so the theory must satisfy $\delta_i \kappa \RV{Y}_i = \mathbb{E}_\mu[\RV{Y}_i]$
    \item Assume performance bias has a uniform additive effect: the theory satisfies $\delta_i \kappa \RV{Y}_i = \mathbb{E}_\mu[\RV{Y}_i]+k$. In this case the average treatment effect can still be estimated from the data: $\delta_1 \kappa \RV{Y}_1 - \delta_0\kappa\RV{Y}_0 = \mathbb{E}[\RV{Y}_1]-\mathbb{E}[\RV{Y}_0]$ which may be sufficient to find a decision function minimising the risk
    \item Avoid assumptions about the effect of performance bias; the theory satisfies no particular relationship between $\mathbb{E}_\mu[\RV{Y}_i]$ and $\delta_i \kappa \RV{Y}_i$ and we may therefore expect preferred decision function to ignore the data
\end{enumerate}
The question of specifying this relationship arises naturally when we consider connecting Potential Outcomes to CSDPs. Nonetheless, the possibility of deviations from option 1 above are often treated as ``external to the causal problem''. For example,  \cite{mansournia_biases_2017} states:
\begin{quote}
    In this case, it might be more appropriate to say that the intention-to-treat effect from the trial is not generalizable or transportable to other settings rather than saying that it is “biased”
\end{quote}
\end{example}

