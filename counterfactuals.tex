
\section{Potential Outcomes}

Potential Outcomes is a major rival to the approach typified by Causal Bayesian Networks for formulating causal questions and hypotheses. Causal queries in the Potential Outcomes framework concern the distribution of random variables $\RV{Y}_0, \RV{Y}_1$ representing potential outcomes, or ``the value $\RV{Y}$ would take if action 0 or 1 were taken respectively'' (\cite{angrist_mastering_2014}). 

Queries based on the Potential Outcomes framework concern properties of the joint distribution over observations and potential outcome variables - that is, they resemble ordinary statistical decision problems. If, as we have supposed, our starting point is preferences over outcomes then the identification of a causal effect in this framework does not automatically allow for us to determine preferences over decisions, as the Potential Outcomes framework does not formally specify a canonical consequence map given a particular joint distribution over potential outcome variables. 

We do have an informal notion of how this should be constructed - namely that the potential outcome $\RV{Y}_1$ is the value $\RV{Y}$ would take if action 1 were taken. We can formalise this with the notion of consequence consistency. Let $\Delta(\mathcal{Y}_\circ)$ be the space of joint distributions over real and potential outcomes of $\RV{Y}$. A consequence map $\kappa:D\to \Delta(\mathcal{Y}_\circ)$ is consequence consistent if
\begin{align}
    \kappa(i;(\RV{Y}_i-\RV{Y})^{-1}(A))=\delta_0(A) \label{eq:oc_consist}
\end{align} 
Consequence consistency is similar to the consistency condition (e.g., see \cite{richardson2013single}).

A causal theory $\mathscr{T}^{cc}$ that is consequence consistent need not require any particular relationship between an ``observed'' distribution $\mu\in \Delta(\mathcal{Y}_\circ)$ and an associated consequence $\kappa$. A straightforward option for making this connection is to assert equality of distributions; a causal theory $\mathscr{T}^{cc,eq}$ is consequence consistent and counterfactually equal if for all $(\mu,\kappa)\in\mathscr{T}^{cc,eq}$
\begin{align}
    \kappa(i;\RV{X}_i^{-1}(A))=\mu(\RV{X}_i^{-1}(A))
\end{align}
There are many choices that could be made here, as can be seen in Example \ref{ex:nonst_distn}.

\begin{example}[Nonstationary distribution]\label{ex:nonst_distn}

\end{example}

A causal theory for Potential Outcomes is associated with a much larger hypothesis class than any causal theory that works only with distributions over observable variables. Theorem \ref{th:scdp_obu_red} shows that given any SCDP based on Potential Outcomes, provided that the potential outcome variables are unobserved and the utility does not depend on them, a reduced SCDP can be constructed by marginalising over potential outcomes. While these conditions may seem reasonable, there are some examples of problems where one does apparently care about the values of potential outcome variables. The \emph{effect of treatment on the treated} (ETT) that depends on counterfactual quantities and has some relevance to decision preferences \cite{rubin_estimating_1974}, though it is controversial whether this dependence is necessary \cite{geneletti2007defining}. More straightforwardly, the legal standard of ``no harm but for the defendant's negligence'' does seem to invoke fundamentally counterfactual considerations \cite{pearl_causality:_2009}.

Recall that we have proposed making the connection between data, decisions and outcomes in two steps: firstly a causal theory relates data to possible consequences (the ``inference'' step), and secondly a consequence then relates decisions to outcomes (the ``control'' step). We could consider a generalised consequence $D\times E\to \mathscr{P}(\Delta(\mathcal{X}))$ that jointly performs the inference and control steps. We speculate that this generalisation provides an alternative connection between SCDPs and counterfactual reasoning; in particular, Nonparametric Structural Equation Models (NPSEMs) which are often considered appropriate tools for modelling counterfactual distributions (\cite{pearl_causality:_2009,richardson2013single}) can be seen as a special case of generalised consequences.

\begin{definition}[NPSEM]\label{def:NPSEM}
A non-parametric structural equation model (NPSEM) is a tuple $\langle \{\RV{X}^i, \RV{U}^i, f^i\}_{i\in[N]}, (D,\mathcal{D}), (E,\mathcal{E})\rangle$ where, for all $i\in N$, $\RV{X}^i:E\times D \to X^i$, $\RV{U}^i:E\to U^i$, $\mathscr{H}\subset\Delta(\mathcal{E}$ and $D=\times_{i\in[N]} X^i\cup\{*\}$ and $f^i:\times_{j<i} X^i\times U^i\to X^i$ are functions measurable with respect to the implied product sigma algebras. The $\RV{X}^i$ are given by

\begin{align}
    \RV{X}^i(e,d) = \begin{cases} f^i(\RV{X}^{<i}(e,d),\RV{U}^i(e)) &\RV{D}^i(d)=*\\ 
    \RV{D}^i(d)  &\RV{D}^i(d)\neq * \end{cases}
\end{align}

Where $\RV{X}^{<i}(e,d)=[\RV{X}^0(e,d),...\RV{X}^{i-1}(e,d)]$.
\end{definition}

Given an NPSEM $\mathscr{M}:\langle \{\RV{X}_i, \RV{U}_i, f_i\}_{i\in[N]}, (D,\mathcal{D}), (E,\mathcal{E})\rangle$ we can let $\RV{X}$ be the joint space of all the $\RV{X}_i$ and an NPSEM induces a measurable function $M:D\times E\to X$. In general there are many NPSEMs that induce the same measurable function in this manner. For example $f_0:e\mapsto e$ and $f_1:(x_0,e)\mapsto x_0$ induces the same function $E\to X_0\times X_1$ as $f_0:e\mapsto e$ and $f_1:(x_0,e)\mapsto e$. If consider all NPSEMs inducing the same function $M$ to be equivalent, then we can, abusing terminology somewhat, consider an NPSEM to be a measurable function $D\times E\to X$. A straightforward generalisation of this is a stochastic NPSEM $D\times E\to \Delta(\mathscr{X})$, and a set of stochastic NSPEMs induces a set valued stochastic map $D\times E\to \mathscr{P}(\Delta(\mathcal{X}))$.
