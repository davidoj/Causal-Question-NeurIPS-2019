%!TEX root = main.tex

\section{Counterfactual models}

A counterfactual model poses a number of ordinary random variables $\RV{X}, \RV{Y}$ and a number of counterfactual random variables $\RV{Y}_x$. It is not clear what the set of counterfactual indices should be in general; we will posit that it can be taken to be the set of decisions $D$. It is also not clear whether every random variable should have counterfactual versions; we will consider that this is so (so we have $\RV{X}_x$, though this may be trivial).

Additionally, a counterfactual model poses certain restrictions on the values and joint distribution of random variables. Consistency is sometimes stronger than a restriction on the joint distribution:
\begin{align}
\forall e\in E, \RV{X}(e) = x\implies \RV{Y}(e) = \RV{Y}_x (e)
\end{align}

May additionally pose ignorability (or something else):
\begin{align}
\forall \mu\in \mathscr{H}, \{\RV{Y}_x\}_{D} \CI_\mu \RV{X}\}
\end{align}

and SUTVA:
\begin{align}

\end{align}

\subsection{Can we consider causal theories to be counterfactual models?}

There is an obvious way to do this, but it fails to capture the assumption of consistency.

More generally we can employ the standard method of turning a SDP into a CSDP.

\subsection{Can we consider counterfactual models to be causal theories?}

Given a CSDP we can construct an SDP. Not a counterfactual SDP...but we can always turn an SDP into a censored version of an SDP with additional structure. 

The backwards construction of interest: ill-posed counterfactual SDP to CSDP. ``Can a counterfactual SE do the job of a CT''?

Yes, there is an obvious correspondence between a causal theory and a class of counterfactual models (injection CT -> P(CF)?). But: this correspondence is at odds with the way that people actually use counterfactual (see ATE (why take a difference?) and ETT (Heckman: incentives might mean ETT is appropriate rather than ATE)). Have people been using counterfactuals wrongly, or should we consider a richer set of CF->CT rules?

More importantly, \textbf{any CF->CT rule is itself a causal theory!}. We can't avoid invoking a causal theory to go from ill-posed CF SDP to CSDP.

\textbf{Proposition:} Isomorphism between rules: ill-posed SDP to CSDP and causal theories.

Conclusion: contra three-level hierarchy, if we care about consequences we need a causal theory.