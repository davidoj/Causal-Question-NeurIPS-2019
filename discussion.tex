%!TEX root = main.tex

\section{Discussion}

We have introduced an original approach to formulating questions of causal inference and analysing approaches to causal modelling. We take cues from statistical decision theory and make heavy use of the theory of Markov kernels for reasoning about causal theories, the central object of our approach. Our approach makes crystal clear the distinction between ``statistical'' and ``causal'' knowledge -- the former is represented by a statistical experiment and the latter by a causal theory. Causal Bayesian networks and Potenial Outcomes models can both be used to generate causal theories.

While we do not address the unique questions that can be posed using counterfactual models \citep{pearl_causality:_2009}, our approach suggests an alternative view for the relationship between counterfactual and interventional causal models. Rather than occupying different levels of a hierarchy, each yields a different kind of rich causal theory. We might speculate that CBN theories are particularly suited to some domains and PO theories to others. Indeed, we see extensive discussion of counterfactual treatment effects in the econometrics literature, where decisions usually involve changing incentives which can plausibly be understood as altering the assignment function $\mathbf{W}$ in unpredictable ways \citep{angrist_mastering_2014,carneiro_evaluating_2010,imbens_identification_1994}. Causal Bayesian Networks, on the other hand, have found applications in the study of biological systems which typically feature large numbers of variables which permit a wide variety of targetted interventions \citep{sachs_causal_2005,maathuis_estimating_2009}.

Though we develop CSDT in the context of ``small world decision problems'' \citep{joyce_foundations_1999}, we can apply causal theories to the study of questions beyond this context, such as the questions of coarsening and reusable inference discussed here. Theorem \ref{th:mod_extn} is a means by which rich theories can inform decisions involving more realistic theories, though it is by no means the only one. It raises a number of questions: what other methods allow for inferences to be reused? Given that we know a coarsening exists, how much do we need to know about the realistic theory in order to determine what this coarsening is? More broadly, how can we formally pose the question ``what makes a rich causal theory a `good' one''?

A number of the results here are predicated on discrete spaces which allows us to disregard questions of measurability. A second important direction is extending this theory to continuous spaces and understanding what limitations this introduces. Relatedly, the notions of conditional probability, conditioning, independence and Bayesian inversion are well understood in the context of probability measures, including in their string diagrammatic treatment \citep{cho_disintegration_2019}, but we are not aware of analogues of these notions for general Markov kernels. These basic notions would be invaluable tools in the analysis of causal theories.

The string diagram notation we use has a strong connection with the DAGs \citep{fong_causal_2013} used in causal graphical models as well as to influence diagrams\citep{dawid_influence_2002}, as do Markov kernels themselves. It would not be surprising if there were a deep connection between the two. 

Causal statistical decision theory is new, and many details are still being worked out. We have shown how CSDT can bring new understanding to existing approaches to causality by formalising fundamental but previously informal notions such as that of ``stable causal models'', and we believe that it is an approach that has a great deal of potential in furthering our understanding of causal inference. 