%!TEX root = main.tex

\section{Statistical Decision Problems to Causal Statistical Decision Problems}

Statistical decision theory (SDT) is a theory of data-driven decision making that was set out in detail by \citet{wald_statistical_1950}. Many core concepts of machine learning have precursors in SDT including statistical experiments, losses and risk functionals. SDT also sets out foundational theorems such as the complete class theorem, which establishes the completeness of Bayes decision rules, and the Blackwell-Sherman-Stein theorem characterising the comparison of statistical experiments. SDT, however, does not feature any notion of \emph{consequences} and as a result is not suited to causal problems. CSDT is an extension of SDT that features consequences.

A statistical decision problem (SDP) poses the following scenario: suppose we have a set of ``states of nature'' $\Theta$, a set of decisions $D$ and a loss function $l:\Theta\times D\to \mathbb{R}$. For each state of nature $\theta\in \Theta$ there is an associated probability measure $\mu_\theta\in \Delta(\mathcal{E})$ where $(E,\mathcal{E})$ is some measurable space. Call the stochastic map $\mathbf{H}:\theta\mapsto \mu_\theta$ a \emph{statistical experiment}. Given a \emph{decision strategy} $\mathbf{J}:E\to \Delta(\mathcal{D})$, define the \emph{risk} of $\mathbf{J}$ given state $\theta$ to be the expected loss of $\mathbf{J}$ in state $\theta$. Specifically, the risk $R:\Delta(\mathcal{D})^E\times \Theta\to \mathbb{R}$ is given by $R:(\mathbf{J},\theta)\mapsto \mathbf{H}_\theta \mathrm{J} l_\theta$, where we make use of kernel product notation for brevity.

We would ideally find a strategy $\mathbf{J}$ that minimises the risk in the ``true state'' $\theta^*$. Unfortunately, we don't know the true state. If there were a decision strategy that minimised the loss in every state, such a strategy would clearly minimise the loss in the true state, but most statistical decision problems don't admit such a strategy.  Two alternative decision rules are available:

Given a measure $\xi\in \Delta(\Theta)$ called a prior, $\xi$-\emph{Bayes decision rule} is a decision rule $\mathbf{J}^*_{\mathrm{Ba}}$ such that the \emph{Bayes risk} $R_\xi:\mathbf{J}\mapsto \int_\Theta H_\theta \mathbf{J} l_\theta d\xi$ is minimised. A \emph{minimax} decision rule $\mathbf{J}^*_{\mathrm{MM}}$ minimises the worst-case risk: $\mathbf{J}^*_{\mathrm{Mm}}\in \argmin_{\mathbf{J}} \max_{\theta\in \Theta} R(\mathbf{J},\theta)$ Unlike a Bayes rule, a minimax rule does not invoke a prior. In general, a decision rule is some rule that selects a decision on the basis of the risk functional $R(\mathbf{J},\cdot)$.

Our representation of statistical experiment is slightly different to, for example, \citet{le_cam_comparison_1996}, who introduces statistical experiments as an ordered collection of probability measures. Both representations do the same job, and the representation as a map makes for a clearer connection with causal statistical decision problems. 

Formally, we define an SDP as the tuple $\langle \Theta, E, D, \mathbf{H}, l\rangle$ where $\Theta, E$ and $D$ are measurable sets, $\mathbf{H}$ is a stochastic map $\Theta\to \Delta(\mathcal{E})$ and $l$ a measurable function $E\to \mathbb{R}$. We leave implicit the set of decision strategies $E\to \Delta(\mathcal{D})$. This is a very bare bones exposition of the theory of SDPs, and for more details we refer readers to \cite{toutenburg_ferguson_1967}.

Observe that a statistical decision problem supplies a loss $l$ that tells us immediately how desirable a pair $(\theta,d)\in\Theta\times D$ is. It is more typical to talk about how desirable the \emph{consequences} of a decision are than how desirable a (state, decision) pair is. If the set of possible consequences of a decision is denoted by a set $F$, let the desirability of an element $f\in F$ be given by a utility function $u:F\to \mathbb{R}$. Given such a $u$, the tuple $\langle \Theta, E, D, \mathbf{H}, u\rangle$ is an ill-posed problem: we want to evaluate the desirability of decision strategies $\mathbf{J}$, but we have no means of connecting decisions with consequences $F$. We introduce for each state of nature $\theta$ a \emph{consequence map} $\mathbf{C}_\theta:D\to \Delta(\mathcal{F})$; let $\mathbf{C}$ be the Markov kernel $\theta\mapsto \kappa_\theta$. We can then define the \emph{causal risk} $S:\Delta(\mathcal{D})^E\times \Theta\to \mathbb{R}$ by $S:(\mathbf{J},\theta)\mapsto -H_\theta \mathbf{F} C_\theta u$, and Bayes and minimax risks are defined analogously.

For each state $\theta\in \Theta$, the Markov kernel
\begin{align}
    \mathbf{T}_\theta := 
\begin{tikzpicture}
\path (0,0) node[dist] (theta) {$\delta_\theta$}
      +(0,-0.5) coordinate (D)
      ++(0.5,0) coordinate (copy0)
      ++(0.5,0) node[kernel] (H) {$H$}
      +(0,-0.5) node[kernel] (C) {$C$}
      ++(0.7,0) node (E) {$E$}
      +(0,-0.5) node (F) {$F$};
\draw (theta) -- (copy0);
\draw (D) -- (C) -- (F);
\draw (copy0) to [bend right] (C);
\draw (copy0) to [bend left] (H);
\draw (H) -- (E);
\end{tikzpicture}\label{eq:ttheta_def}
\end{align}

Is sufficient to compute the causal risk. Thus we can replace $\mathbf{H}$ and $\mathbf{C}$ with the \emph{causal theory} $\mathbf{T}:\Theta\times D\to \Delta(\mathcal{E}\otimes\mathcal{F})$ given by $(\theta,d)\mapsto \mathbf{T}_\theta(d;\cdot)$. A causal statistical decision problem (CSDP) is therefore a tuple $\langle \Theta, E, F, D, \mathbf{T}, u\rangle$.

Given a CSDP $\alpha = \langle \Theta, E, F, D, T, u\rangle$ where $\mathbf{T}$ is a theory arising from some $\mathbf{H}$ and $\mathbf{C}$ as in Equation \ref{eq:ttheta_def}, we can recover the original kernels by marginalisation: $\mathbf{H}= T(mathbf{Id}\otimes *)$ and $\mathbf{C}=\mathbf{T}(*\otimes \mathbf{Id})$. Given $\alpha$ and letting $l:= \mathbf{C}u$ we induce the canonical SDP $\beta=\langle \Theta, E, D, \mathbf{H}', l\rangle$ such that for any $\theta\in \Theta$, $\mathbf{J}$, $R^{(\beta)}(\mathbf{J},\theta) = S^{(\alpha)}(\mathbf{J},\theta)$, and thus $\alpha$ and $\beta$ will always produce identical recommendations.

It is also possible to induce a CSDP from an arbitrary SDP $\beta:=\langle \Theta, E, D, \mathbf{H}, l\rangle$. First, define $F:=\Theta\times D$ and then let $u:=-l$. Define $\mathbf{C}:\Theta\to (D\to \Delta(\mathcal{F}))$ by $\mathbf{C}:\theta\mapsto (d\mapsto (\theta,d))$, and then construct $\mathbf{T}$ from $\Theta, \mathbf{H}$ and $\mathbf{C}$ as in \ref{eq:ttheta_def}. Then the CSDP $\alpha:=\langle \Theta, E, F, D, \mathbf{T}, u\rangle$ has the property $S^{(\alpha)}(\mathbf{J}, \theta) = R^{(\beta)}(\mathbf{J},\theta)$.

Causal theories are the central object of study here. They provide a bridge between the experiment $\mathbf{H}$ and the consequences $\mathbf{C}$ and allow us to use the former to make inferences about the latter. 