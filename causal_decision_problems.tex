\section{Causal Statistical Decision Problems}
\begin{table}[ht]
    \centering
\begin{tabular}{ |c|c|c| } 
 \hline
  & SDPs & CSDPs \\ 
 \hline
 State of the world & $\mathscr{H}$ & $\mathscr{T}$ \\ 
 Observations & $\RV{X}$ & $\RV{X}$ \\ 
 Decisions & $D$ & $D$ \\
 Known preferences & $\ell$ & $U$ \\
 Derived preferences & $\ell$ & $L$\\
 \hline
\end{tabular}
    \caption{Comparison of SDPs and CSDPs}
    \label{tab:sdp_cdp_comparison}
\end{table}



Following \citep{toutenburg_ferguson_1967}, we consider a statistical decision problems and causal statistical decision problems to represent normal form two person games. The game represents at the highest level the options and possible payoffs available to the decision maker. If two problems represent ``essentially'' the same game from the perspective of the decision maker, we say that one game can be reduced to the other.

\cheng{Can we split this section into two? Statistical decision theory, perhaps move back to the previous section, and call that section ``Background: Statistical Decision Theory''. Clarify your introduced concepts and notation in terms of existing work. We want to have a section which contains all the new results. Right now, it is unclear what is known, what is known (but expressed in a new way), and what you are proposing.}

\begin{definition}[Two person game (normal form)]
A normal form game is a triple $\langle \mathscr{S}, A, L\rangle$ where $\mathscr{S}$ and $A$ are arbitrary sets and $L:\mathscr{S}\times A\to [0,\infty)$ is a loss function.
\end{definition}
The set $\mathscr{S}$ is a set of possible states that the environment may occupy and $A$ is a set of actions the decision maker may take. The decision maker seeks an action in $A$ that minimises the loss $L$. Generally there is no action that minimises the loss for all environment states. A minimax solution is an action that minimises the worst case loss: $a^*_{mm} = \argmin_{a\in A} [\sup_{s\in \mathscr{S}} L(s,a)]$.

If the set $\mathscr{S}$ is equipped with a $\sigma$-algebra $\mathcal{S}$ and a probability measure $\xi\in \Delta(\mathcal{S})$ which we will call a ``prior'', a Bayes solution minimizes the expected risk with respect to $\xi$: $a^*_{ba} = \argmin_{a\in A} \int_{\mathcal{S}} L(s,a) \xi(ds)$.
=======
\cheng{Justify why you need a game. Seems to be needed for reduction of states.}


\begin{definition}[Admissible Action]
Given a normal form two person game $\langle \mathscr{S}, A, L\rangle$, an action $a\in A$ is strictly better than $a'\in A$ iff $L(s,a)\leq L(s,a')$ for all $s\in\mathscr{S}$ and $L(s_0,a)<L(s_0,a')$ for some $s_0\in \mathscr{S}$. If only the first holds, then $a$ is as good as $a'$. An admissible action is an action $a\in A$ such that there is no action strictly better than $A$.
\end{definition}

\begin{definition}[Complete Class]
A class $C$ of decisions is a complete class if for every $a\not\in C$ there is some $a'\in C$ that is strictly better than $a$.

A class $C$ is an essentially complete class if for every $a\not\in C$ there is some $a'\in C$ that is as good as $a$.
\end{definition}

\begin{definition}[Reduction]\label{def:red_sdp_CSDP}
A normal form two person game $\alpha = \langle \mathscr{S}^\alpha, A, L^\alpha\rangle$ can be reduced to a different game sharing the same action set $\beta = \langle \mathscr{S}^\beta, A, L^\beta \rangle$ if there is a surjective function $g:\mathscr{S}^\alpha\to \mathscr{S}^\beta$ such that for every $a\in A$, $s\in \mathscr{S}^\alpha$, $L^\alpha(s,a) = L^\beta(g(s),a)$.
\end{definition}

Reduction preserves admissibility and completeness as shown in Appendix \ref{app:csdps}

A statistical decision problem represents a normal form two-person game where the available actions are \emph{decision functions} that output a decision given data, the states of the environment are associated with probability measures on some measurable space and we assume a loss expressing preferences over decisions and states is known.

\begin{definition}[Statistical Decision Problem]
A statistical decision problem (SDP) is defined by a tuple $\langle (\mathscr{H},(E,\mathcal{E}), \RV{X}), (D,\mathcal{D}), \ell\rangle$. $\mathscr{H}\subset\Delta(\mathcal{E})$ is a hypothesis class representing possible states of the environment, $D$ is the set of available decisions, $\RV{X}:(E,\mathcal{E})\to (X,\mathcal{X})$ is a random variable representing the information available for the statistician to make a decision and $\ell:\mathcal{H}\times D\to [0,\infty)$ is a loss function.

Denote by $\mathscr{J}$ the set of decision kernels $X\to \Delta(\mathcal{D})$. Recall that $\mu F_{\RV{X}}(A)=\mu(\RV{X}^{-1}(A))$. For $J\in \mathscr{J}$ and $\mu\in \mathcal{H}$, the risk $R:\mathscr{J}\times\mathscr{H}\to [0,\infty)$ is defined as $R(J,\mu) = \int_D \ell(\mu,y) \mu F_{\RV{X}} J(dy)$.

Denoting by $\mathscr{J}$  the set of kernels $X\to \Delta(\mathcal{D})$, the triple $\langle \mathscr{H}, \mathscr{J}, R\rangle$ forms a two player normal form game.
\end{definition}


The loss function $\ell$ expresses preferences over (state, decision) pairs. However, it may be the case that our preferences don't naturally apply directly to such pairs. For a doctor deciding whether to prescribe a treatment to a patient, it is clear that this patient being healthy in the future is preferable to them being sick. This motivates the definition of a causal statistical decision problem, utilising a preferences defined over outcomes rather than (state, decision) pairs. In order to compute the loss associated with a decision a map from decisions to outcomes is required, which we term a \emph{consequence}.

\cheng{You should justify why you have $E$ and $F$, and not just one thing. This is a major departure from statistical decision problems.}


\begin{definition}[Consequences]
Given a measurable outcome space $(F,\mathcal{F})$ and a measurable decision space $(D,\mathcal{D})$, a Markov kernel $\kappa:D \to \Delta(\mathcal{F})$ is a \emph{consequence mapping}, or just a consequence for short.
\end{definition}

\begin{definition}[Causal state]
Given a consequence $\kappa:D\to \Delta(\mathcal{F})$, a measurable observation space $(E,\mathcal{E})$ and some distribution $\mu\in \Delta(\mathcal{E})$, the pair $\tau:=(\kappa,\mu)$ is a \emph{causal state} on $D, E$ and $F$. We refer to $\kappa$ as the consequence and $\mu$ as the observed state.
\end{definition}

\begin{definition}[Causal Theory]\label{def:causal_theory}
A causal theory $\mathscr{T}$ is a set of causal states sharing the same decision, observation and outcome spaces.
\end{definition}

\cheng{You need to define the different uses of ``utility''. E.g. utility, pseudo-utility, ordinary utility, generalized utility, normal utility. Or perhaps be more careful with your usage.}

\cheng{The change from loss $\ell$ to $(U,F)$ is a surprising, so you need to explain why you need to do so.}

\cheng{The following definition is too long. Defines too many things.}

\begin{definition}[Causal Statistical Decision Problem]\label{def:CSDP}
A causal statistical decision problem (CSDP) is a tuple $\langle (\mathscr{T}, (E,\mathcal{E}) \RV{X}), (D,\mathcal{D}), (U,(F,\mathcal{F})) \rangle$. $\mathscr{T}$ is a causal theory on $D, E$ and $F$, $D$ is the decision set, $\RV{X}:(E,\mathcal{E})\to (X,\mathcal{X})$ is a random variable representing the given information and $U:\Delta(\mathcal{F}\otimes \mathcal{D})\to \mathbb{R}$ is a pseudo-utility expressing preference over joint distributions of decisions and outcomes which we assume is bounded above.

From the pseudo-utility $U$ we can define a loss $L:\mathscr{T}\times\Delta(\mathcal{D})\to [0,\infty]$ by
\begin{align}
    L((\kappa,\mu),\gamma) := \sup_{\gamma'\in\Delta(\mathcal{D})} U(\gamma'\splitter{0.15}(I_D\otimes \kappa)) - U(\gamma\splitter{0.15}(I_D\otimes \kappa))\label{eq:canonical_loss}
\end{align}
For $(\kappa,\mu)\in \mathscr{T}$ and $\gamma\in \Delta(\mathcal{D})$. This is well defined wherever $U$ is bounded above. Note that $L$ does not depend on the data generating distribution $\mu$; henceforth we will suppress this argument and write $L(\kappa,\gamma):= L((\kappa,\mu),\gamma)$.

Given a decision function $J\in\mathscr{J}$ and $(\kappa,\mu)\in \mathscr{T}$, we define the risk $R:\mathscr{J}\times \mathscr{T}\to [0,\infty)$ by $R(J,\kappa,\mu) := L(\kappa,\mu F_{\RV{X}} J)$.

The triple $\langle \mathscr{T}, \mathscr{J}, R\rangle$ is a normal form two person game.

If there exists some measurable $u:F\times D\to \mathbb{R}$ such that for all $\xi\in \Delta(\mathcal{F}\otimes\mathcal{D})$, $U(\xi)=\mathbb{E}_{\xi}[u]$ then we call $u$ an ordinary utility and by extension $U$ an ordinary pseudo-utility.

An ordinary utility induces a loss $L(\kappa,\gamma) = \mathbb{E}_{\gamma}[l^\kappa]$
where $l^\kappa:D\to [0,\infty)$ is defined by
\begin{align}
    l^\kappa(d) := \sup_{\gamma'\in \Delta(\mathcal{D})} \mathbb{E}_{\gamma'\splitter{0.08}(I_D\otimes \kappa)}[u] - \mathbb{E}_{\kappa(d;\cdot)}[u(\cdot,d)]\label{eq:induced_l}
\end{align}
\end{definition}

\cheng{Create a table that summarises the similarities and differences between the two decision problems.}

Reduction from a CSDP to an ordinary SDP is sufficient to import results from statistical decision theory such as Theorem \ref{th:complete_class}.

\cheng{Explain why complete class is important.}

\begin{theorem}[Complete class theorem (CSDP)]\label{th:complete_class}
Given an CSDP $\alpha:=\langle (\mathscr{T},E),D,\RV{X},U\rangle$ with risk $R$, if there is a reduction to an SDP $\beta:=\langle (\mathscr{H},F),D,\RV{Y},\ell \rangle$ with risk $R'$ such that $|\mathscr{H}|<\infty$ and $\inf_{J\in\mathscr{J},\mu\in\mathscr{H}} R'(J,\mu)<-\infty$, then the set of all Bayes decision functions is a complete class for $\beta$ and the set of all admissible Bayes decision functions is a minimal complete class for $\beta$.
\end{theorem}

\begin{proof}
This follows from Lemmas \ref{cor:red_comp} and \ref{lem:IB_rule}. See appendix \ref{app:csdps}.
\end{proof}

Any statistical decision problem can be reduced to a CSDP featuring a causal theory where decisions have no effect.

\cheng{Justify the choice of word ``reduce''.}


\begin{theorem}\label{th:sdp_to_CSDP}
Every SDP $\langle (\mathscr{H},E,\RV{X}),D,\ell\rangle$ can be reduced to a CSDP.
\end{theorem}
\begin{proof}
Choose a theory that matches every probability measure $\mu\in\Delta(\mathcal{E})$ with a consequence map that itself always yields $\mu$. Then construct a utility $U$ that induces an identical risk. See Appendix \ref{app:csdps}.
\end{proof}

The pseudo-utility $U$ enables Theorem \ref{th:sdp_to_CSDP}. A reduction is sometimes possible to an CSDP featuring an ordinary utility with a nontrivial theory, but whether this is always possible remains an open question.

A CSDP cannot, in general, be reduced to a statistical decision problem - for example, if we choose the utililty to be the variance of some random variable we may be able to achieve higher utility through a randomised decision than any nonrandomised decision under conditions where a regular SDP cannot have this property (see Example \ref{ex:ired_csdp} in Appendix \ref{app:csdps}). It is an open question whether this reduction is generally possible if the problem features a normal utility.

Theorem \ref{th:red_CSDP} reduces a CSDP to a SDP by associating each pair $(\mu,\kappa)$ in the causal theory with a distribution over $E\times F\times D$. This may be possible by finding a distribution on the product space for which $\mu$ is a marginal probability and $\kappa$ is a conditional distribution, but this is not necessary.

\begin{theorem}\label{th:red_CSDP}
Given a CSDP $\beta=\langle (\mathscr{T},E,\RV{X}),D,(U,F)\rangle$ where $U$ is an ordinary pseudo-utility, let $\mathscr{K}=\{\kappa|(\kappa,\mu)\in \mathscr{T}\}$ be the set of consequences. $\beta$ is reducible to a statistical decision problem on the measurable space $(E\times F\times D,\mathcal{E}\otimes \mathcal{F}\otimes \mathcal{D})$ if there is some surjective map $m:\Delta(\mathcal{F}\otimes\mathcal{D})\to \mathscr{K}$.
\end{theorem}

\begin{proof}
We construct the map $h$ based on the map $m$ and show that, given an ordinary utility $U$, it is possible to construct a loss $\ell$ such that the resulting SDP features the same risk assignments as the original CSDP. See Appendix \ref{app:csdps}.
\end{proof}

\begin{corollary}\label{cor:card_reduc}
If the cardinality of $\Delta(\mathcal{F}\otimes\mathcal{D})$ is at least as large as the cardinality of the set of Markov kernels $D\to \Delta(\mathcal{F})$ then an CSDP with an ordinary utility can always be reduced to a SDP.
\end{corollary}

\begin{proof}
This follows from Theorem \ref{th:red_CSDP} and the definition of cardinality.
\end{proof}

A major open question, then, is if $(E,\mathcal{E})$ and $(D,\mathcal{D})$ are standard measurable spaces, the conditions for Corollary \ref{cor:card_reduc} hold in general. Corollary \ref{cor:CSDP_to_sdp} shows that the reduction can be made in general if $D$ is a denumerable set.

\begin{corollary}\label{cor:CSDP_to_sdp}
A CSDP $\langle (\mathscr{T},(E,\mathcal{E}),\RV{X}),(D,\mathcal{D}),(U,(F,\mathcal{F}))\rangle$ where $D$ is a denumerable set and $U$ is an ordinary pseudo-utility can be reduced to a statistical decision problem.
\end{corollary}

\begin{proof}
Take some probability measure $\pi\in \Delta(\mathcal{D})$ such that $\pi(\{y\})>0$ for all $y\in D$. Such a $\pi$ exists by the denumerability of $\mathcal{D}$. The map $m:\Delta(\mathcal{F}\otimes\mathcal{D})\to \mathscr{K}$ given by $m(\xi)(y;A) := \frac{\xi(A\times\{y\})}{\pi(\{y\})}$ is surjective. The result follows from Theorem \ref{th:red_CSDP}.
\end{proof}

This reduction is not particularly practically useful, but it is sufficient to lift results such as Theorem \ref{th:complete_class} from the theory of statistical decision functions.
