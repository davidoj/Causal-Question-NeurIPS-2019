%!TEX root = main.tex

Note on terminology: I am trying the name ``See-Do model'' to describe the folowing:

\begin{tikzpicture}
  \path (0,0) node(theta) {$\Theta$}
     +(0,-1) node (P) {$D$}
     ++(0.5,0) coordinate (copy0)
     ++(0.5,0) node[kernel] (H) {$\mathbf{See}$}
     +(0,-1) node[kernel] (I) {$\mathbf{Do}$}
     ++(0.7,0.) node (N1) {$\RV{X}$}
     +(0,-1) node (N2) {$\RV{Y}$};
  \draw (theta) -- (copy0);
  \draw (P) -- (I);
  \draw (I) -- (N2);
  \draw (copy0) to [bend right] ($(I.west)+(0,0.1)$);
  \draw (copy0) -- (H);
  \draw (H) -- (N1);
  \end{tikzpicture}

I was calling it a ``causal theory'' before. Reasons for the change: I think ``See-Do'' helps to understand what the model does, and the name doesn't make premature claims to explain causality. Also, it's only two syllables which I like.