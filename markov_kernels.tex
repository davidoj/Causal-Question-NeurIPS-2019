\section{Definitions}
\begin{definition}[Markov kernel]
Given two measureable sets $(E,\mathcal{E})$ and $(F,\mathcal{F})$, a Markov kernel $K$ is a map $E\times \mathcal{F} \to [0,1]$ where
\begin{enumerate}
    \item The map $x\mapsto K(x;B)$ is $\mathcal{E}$-measurable for every $B\in\mathcal{F}$
    \item The map $B\mapsto K(x;B)$ is a probability measure on $(F,\mathcal{F})$ for every $x\in E$ 
\end{enumerate}

We will give Markov kernels the alternate type signature $K:E\to \Delta(\mathcal{F})$ to emphasize part 2 of this definition, bearing in mind that they must also satisfy the measurability of part 1.
\end{definition}

\begin{definition}[Measure-kernel-function product]\label{def:kernel_products}
If $K$ is a Markov kernel from $E\to \Delta(F)$ and $\mu$ is a probability measure on $(E,\mathcal{E})$, then
\begin{align}
    \mu K(B)=\int_E \mu(dx) K(x;B),\qquad B\in\mathcal{F}
\end{align}
defines a probability measure $\mu K$ on $(F,\mathcal{F})$.

If $f$ is a nonnegative measurable function $F\to \mathbb{R}$ then
\begin{align}
    Kf(x) = \int_F K(x;dy)f(y), \qquad x\in E
\end{align}
is a nonnegative measureable function $E\to \mathbb{R}$.

If $L$ is a Markov kernel from $F\to \Delta(G)$, then
\begin{align}
    KL(x;B) = \int_F K(x;dy) L(y;B),\qquad x\in E, B\in \mathcal{G}
\end{align}
is a Markov kernel $KL:E\to \Delta(G)$. \cite{cinlar_probability_2011}
\end{definition}

In addition
\begin{align}
    \underline{K\otimes L}(x;A\times B) = \int_A K(x;dy) L(y;B),\qquad x\in E, A\in \mathcal{F}, B\in \mathcal{G} \label{eq:outer_product}
\end{align}
is a Markov kernel $\underline{K\otimes L}:E\to \Delta(\mathcal{F}\otimes\mathcal{G})$.

$I_E:E\to \Delta(\mathcal{E})$ is the identity kernel $I_E:x\mapsto \delta_x$.

We extend the underline notation to write, given random variables $\RV{X}:E\to X$ and $\RV{Y}:E\to Y$, $\underline{\RV{X}\otimes\RV{Y}}=\underline{(I_E\otimes I_E)}(\RV{X}\otimes \RV{Y})$. Given some $\mu\in \Delta(\mathcal{E})$, $\underline{\RV{X}\otimes\RV{Y}}_*\mu$ is the usual joint distribution of $\RV{X}$ and $\RV{Y}$.

For proofreader's benefit, the notation $\underline{K\otimes L}$ actually represents the ordinary product $\splitter{0.15}(K\otimes L)$ where $\splitter{0.15}$ is a Markov kernel that makes two copies of its input, and this product yields the integral in Eq. \ref{eq:outer_product}.