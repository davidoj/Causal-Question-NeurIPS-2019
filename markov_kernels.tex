\section{Definitions}

\cheng{One sentence describing the difference between $E$ and $\mathcal{E}$. Explicitly say that you are using caligraphic letters to mean measures, e.g. $\mathcal{G}, \mathcal{H}$ below.}

\cheng{Generally, you need to define notation (instead of expecting the reader to infer from type signatures).}

\cheng{You need to clarify the distinction between measures and probability measures.}

Given two measureable sets $(E,\mathcal{E})$ and $(F,\mathcal{F})$, a \emph{Markov kernel} $K$ is a map $E\times \mathcal{F} \to [0,1]$ where
\begin{enumerate}
    \item The map $x\mapsto K(x;B)$ is $\mathcal{E}$-measurable for every $B\in\mathcal{F}$
    \item The map $B\mapsto K(x;B)$ is a probability measure on $(F,\mathcal{F})$ for every $x\in E$
\end{enumerate}

Abusing notation somewhat, we will give Markov kernels the alternate type signature $K:E\to \Delta(\mathcal{F})$ to emphasize part 2 of this definition, noting that part 1 adds additional restrictions. We will sometimes write the set of Markov kernels of type $E\to \Delta(\mathcal{F})$ as $\Delta(\mathcal{F})^D$, noting again that given the measurability restriction of part 1, the set of Markov kernels of this type may be smaller than $\Delta(\mathcal{F})^D$.

\cheng{$D$ not defined.}

\cheng{$\Delta$ not defined.}

\subsection{Operations with Markov kernels}

For the following, assume $K$ is a Markov kernel from $E\to \Delta(\mathcal{F})$, $K'$ a kernel $E\to \Delta(\mathcal{H})$, L is a Markov kernel $F\to \Delta(\mathcal{G})$, $\mu$ is a probability measure on $(E,\mathcal{E})$, $\nu$ is a probability measure on $(F,\mathcal{F})$ and $f$ is a nonnegative measurable function $F\to \mathbb{F}$.

\cheng{$\mathbb{F}$ needs clarification.}

The notation here borrows heavily from \cite{cinlar_probability_2011} and \cite{fong_causal_2013}.

\cheng{You need to write a small tutorial about Markov kernels, otherwise reader cannot follow.}

\subsubsection{Kernel products}

\cheng{Explicitly say that $KL$ is a product of kernels $K$ and $L$. Symmetric? Distributive? Similarly for all the other products.}

$KL$ is a Markov kernel $E\to \Delta(\mathcal{G})$ such that
\begin{align}
    KL(x;B):= \int_F K(x;dy) L(y;B),\qquad x\in E, B\in \mathcal{G}
\end{align}

$\mu K$ is a probability measure on $(F,\mathcal{F})$ such that
\begin{align}
    \mu K(B)=\int_E \mu(dx) K(x;B),\qquad B\in\mathcal{F}
\end{align}

$Kf$ is a nonnegative measurable function $E\to \mathbb{R}$ such that
\begin{align}
    Kf(x) := \int_F K(x;dy)f(y), \qquad x\in E
\end{align}

\subsubsection{Special kernels and operations}

\cheng{Motivate why we need these special definitions. I think the kernels are special, but operations not.}

\cheng{Also teach the reader how to say the symbols $\splitter{0.15}, \stopper{0.15}$.}

$I_E$ is a kernel $E\to \Delta(\mathcal{E})$ defined by $x\mapsto \delta_x$. It has the properties $\mu I_E=\mu$, $KI_F = K$, $I_E K = K$, $I_F f=f$.

$\splitter{0.15}_E$ is a kernel $E\to \Delta(\mathcal{E}\otimes\mathcal{E})$ defined by $x\mapsto \delta_{(x,x)}$. We will subsequently leave the space implicit.

Given $M:H\to \Delta(\mathcal{I})$, $K\otimes M$ is a Markov kernel $E\times H\to \Delta(\mathcal{F}\otimes\mathcal{I})$ where
\begin{align}
    K\otimes M(x,y;A\times B) := K(x;A) M(y;B)
\end{align}

Given $N:I\to \Delta(\mathcal{J})$, it can be verified that $(K\otimes M)(L\otimes N)=KL\otimes MN$.

$\splitter{0.15}(K\otimes K')$ is a Markov kernel $E\to \Delta(\mathcal{F}\otimes\mathcal{H})$ and

\begin{align}
    \splitter{0.15}(K\otimes K')(x;A\times B) &= \int_E K(x';A)K'(x'';B) \delta_{(x,x)} (dx'\times dx'')\\
                                              &= K(x;A)K'(x;B) \label{eq:outer_product}
\end{align}

\cheng{$*$ and $\mathds{1}$ undefined.}

Let $(*,\{\emptyset,*\})$ be an indiscrete measurable set. $\stopper{0.15}_E$ is a kernel $E\to \Delta(\{\emptyset,*\})$ defined by $x\mapsto \mathds{1}_*$. We have $\splitter{0.15}(I\otimes \stopper{0.15}) = I$.

Given some measurable function $g:E\to F$, the kernel $F_g:E\to \Delta(\mathcal{F})$ is defined by $x\mapsto \delta_{g(x)}$. It is easy to check that $F_g F_g = F_g$. For $\mu\in \Delta(\mathcal{E})$, the product $\mu F_g$ is the push forward measure $g_*\mu$.

\cheng{$g_*$ undefined.}

\begin{align}
    \mu F_g (A) &= \int_E \delta_{g(x)}(A) d\mu\\
                &= \mu(g^{-1}(A))\\
                &= g_*\mu(A)
\end{align}

Given two random variables $\RV{X}:(E,\mathcal{E})\to (X,\mathcal{X})$ and $\RV{Y}:(E,\mathcal{E})\to (Y,\mathcal{Y})$, the product $\mu\splitter{0.15}(F_{\RV{X}}\otimes F_{\RV{Y}})$ is the joint distribution of $\RV{X}$ and $\RV{Y}$.

\begin{align}
    \mu \splitter{0.15}(F_{\RV{X}}\otimes F_{\RV{Y}}) (A\times B) &= \int_E \delta_{\RV{X}(x)}(A) \delta_{\RV{Y}(x)}(B) d\mu \\
                        &= \mu(\RV{X}^{-1}(A)\cap \RV{Y}^{-1}(B))
\end{align}

\cheng{Reviewers are not going to be patient enough to sit through 2 pages of definitions. You need to motivate why you need so many spaces, measures, products, etc.}
