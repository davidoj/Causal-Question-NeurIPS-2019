\section{Definitions}


\begin{itemize}
    \item A subscript on a random variable $\RV{X}_i$ refers to its position in a sequence of (usually IID) random variables taking values in the same space
    \item A superscript on a random variable $\RV{X}^i$ marks it out as an element of a set of random variables that do not necessarily take values in the same space
    \item The notation $[N]$ refers to the set of natural numbers $\{0,...,N\}$
    \item A measurable space is a pair $(E,\mathcal{E})$ where $E$ is a set and $\mathcal{E}$ is a $\sigma$-algebra on that set. We assume all measurable spaces discussed are isomorphic to either a subset of $\mathbb{N}$ with the discrete $\sigma$-algebra, or $\mathbb{R}$ with the Borel $\sigma$-algebra (such spaces are known as standard measurable spaces)
    \item Given a measurable space $(E,\mathcal{E})$, $\Delta(\mathcal{E})$ is the set of probability distributions on this space
    \item The script $\mathcal{G}$ refers to a directed acyclic graph; other script letters refer to $\sigma$-algebras
\end{itemize}


\cheng{You need to clarify the distinction between measures and probability measures.}

Given two measureable sets $(E,\mathcal{E})$ and $(F,\mathcal{F})$, a \emph{Markov kernel} $K:E\to \Delta(\mathcal{F})$ is a map where
\begin{enumerate}
    \item The map $x\mapsto K(x;B)$ is $\mathcal{E}$-measurable for every $B\in\mathcal{F}$
    \item The map $B\mapsto K(x;B)$ is a probability measure on $(F,\mathcal{F})$ for every $x\in E$
\end{enumerate}

Abusing notation somewhat, we will write the set of Markov kernels of type $E\to \Delta(\mathcal{F})$ as $\Delta(\mathcal{F})^D$.

Two Markov kernels $K:E\to \Delta(\mathcal{F})$ and $K':E\to \Delta(\mathcal{F})$ are $\mu$-almost surely equivalent given $\mu\in \Delta(\mathcal{E})$ if
\begin{align}
    \int_A K(x;B) d\mu = \int_A K'(x;B) d\mu\qquad\forall A\in \mathcal{E}, B\in\mathcal{F}
\end{align}

Given $\mu\in \Delta(\mathcal{E})$ and a sub-$\sigma$-algebra $\mathcal{D}\subset\mathcal{E}$, there is a Markov kernel $\mu_{\mathcal{D}}:E\to\Delta(\mathcal{E})$ such that for $A\in\mathcal{E}$ and $B\in \mathcal{D}$, $\int_B \mu_{\RV{X}}(y;A) d\mu(y) = \mu(A\cap B)$. $\mu_{\mathcal{D}}$ is a \emph{conditional probability distribution} with respect to $\mathcal{D}$. Given a set of random variables $\mathbf{X}=\{\RV{X}^i\}_{i\in [N]}$ with domain $(E,\mathcal{E})$, $\mu_{\mathbf{X}}$ is a conditional probability distribution with respect to the $\sigma$-algebra generated by $\mathbf{X}$: $\sigma(\cup_{i\in[N]}\sigma(\mathcal{X}^i))$.

\subsection{Kernel products}

For the following, assume $K$ is a Markov kernel from $E\to \Delta(\mathcal{F})$, L is a Markov kernel $F\to \Delta(\mathcal{G})$, $\mu$ is a probability measure on $(E,\mathcal{E})$ and $f$ is a nonnegative measurable function $F\to \mathbb{R}$. More details can be found in Appendix \ref{app:markov_kernels}

The notation here borrows heavily from \cite{cinlar_probability_2011} and \cite{fong_causal_2013}.

\cheng{You need to write a small tutorial about Markov kernels, otherwise reader cannot follow.}


The product of $K$ and $L$, $KL$, is a Markov kernel $E\to \Delta(\mathcal{G})$ such that
\begin{align}
    KL(x;B):= \int_F K(x;dy) L(y;B),\qquad x\in E, B\in \mathcal{G}
\end{align}

The left product of $\mu$ and $K$, $\mu K$ is a probability measure on $(F,\mathcal{F})$ such that
\begin{align}
    \mu K(B)=\int_E \mu(dx) K(x;B),\qquad B\in\mathcal{F}
\end{align}

\subsection{Special kernels}

$I_E$ is the identify kernel $E\to \Delta(\mathcal{E})$ defined by $x\mapsto \delta_x$. It has the properties $\mu I_E=\mu$, $KI_F = K$, $I_E K = K$, $I_F f=f$.

Given some measurable function $g:E\to F$, the kernel $F_g:E\to \Delta(\mathcal{F})$ is defined by $x\mapsto \delta_{g(x)}$. It is easy to check that $F_g F_g = F_g$. For $\mu\in \Delta(\mathcal{E})$, the product $\mu F_g$ is the push forward measure $g_*\mu$. 

This convention allows us to talk about a marginal distribution $\mu F_\RV{X}$ or a marginal kernel $\kappa F_{\RV{X}}$ with consistent notation.

Given $\mu\in \Delta(\mathcal{E}$, $\mu\splitter{0.15}(I_E\otimes K)$ is a distribution in $\Delta(\mathcal{E}\otimes\mathcal{F})$ given by
\begin{align}
    \mu\splitter{0.15}(I_E\otimes K)(A\times B) = \int_A K(x;B) d\mu(x)\qquad \forall A\in \mathcal{E},B\in \mathcal{F}
\end{align}
The symbol $\splitter{0.15}$ is read ``splitter''.




