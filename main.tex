\documentclass{article}

% If your paper is accepted, change the options for the package
% aistats2020 as follows:
%
% \usepackage[accepted]{aistats2020}
%
% This option will print headings for the title of your paper and
% headings for the authors names, plus a copyright note at the end of
% the first column of the first page.

% If you set papersize explicitly, activate the following three lines:
%\special{papersize = 8.5in, 11in}
%\setlength{\pdfpageheight}{11in}
%\setlength{\pdfpagewidth}{8.5in}

% If you use natbib package, activate the following three lines:
\usepackage[round]{natbib}
\renewcommand{\bibname}{References}
\renewcommand{\bibsection}{\subsubsection*{\bibname}}

% If you use BibTeX in apalike style, activate the following line:
%\bibliographystyle{apalike}

\usepackage[utf8]{inputenc} % allow utf-8 input
\usepackage[T1]{fontenc}    % use 8-bit T1 fonts
\usepackage{hyperref}       % hyperlinks
\usepackage{url}            % simple URL typesetting
\usepackage{booktabs}       % professional-quality tables
\usepackage{amsfonts}       % blackboard math symbols
\usepackage{nicefrac}       % compact symbols for 1/2, etc.
\usepackage{microtype}      % microtypography

% My packages

\usepackage[mathscr]{euscript}
\usepackage{graphicx}
\usepackage {tikz}
\usetikzlibrary {positioning}
\usetikzlibrary{shapes.misc}
\usetikzlibrary{shapes.geometric}
\usetikzlibrary{calc}
\usetikzlibrary{arrows.meta}
\usepackage{amsthm}
\usepackage{amsmath}
\usepackage{amssymb}
\usepackage{dsfont}
\usepackage{stmaryrd }
\usepackage{csquotes}
\usepackage{wasysym}
\usepackage[]{todonotes}



\theoremstyle{plain}
\newtheorem{theorem}{Theorem}[section]
\newtheorem{corollary}[theorem]{Corollary}
\newtheorem{lemma}[theorem]{Lemma}
\newtheorem{proposition}[theorem]{Proposition}


\newtheorem{innercustomthm}{Theorem}
\newenvironment{customthm}[1]
  {\renewcommand\theinnercustomthm{#1}\innercustomthm}
  {\endinnercustomthm}

\theoremstyle{definition}
\newtheorem{definition}[theorem]{Definition}
\newtheorem{example}[theorem]{Example}

\newcommand{\CI}{\mathrel{\text{\scalebox{1.07}{$\perp\mkern-10mu\perp$}}}}
\newcommand{\CII}{\mathrel{\text{\scalebox{1.07}{$\perp\mkern-10mu\perp\mkern-10mu\perp$}}}}
\newcommand{\RV}[1]{\ensuremath{\mathsf{#1}}}
\newcommand{\PA}[2]{\ensuremath{\text{Pa}_{#1}(#2)}}
\newcommand{\ND}[2]{\ensuremath{\text{ND}_{#1}(#2)}}
\newcommand{\CH}[2]{\ensuremath{\text{Ch}_{#1}(#2)}}
\newcommand{\DE}[2]{\ensuremath{\text{De}_{#1}(#2)}}
\newcommand{\ID}[1]{\ensuremath{\text{Id}_{#1}}}
\newcommand{\utimes}{\ensuremath{\underline{\otimes}}}

\makeatletter
\newcommand*\bigcdot{\mathpalette\bigcdot@{.5}}
\newcommand*\bigcdot@[2]{\mathbin{\vcenter{\hbox{\scalebox{#2}{$\m@th#1\bullet$}}}}}
\makeatother

\tikzset{
	triangle/.style = {regular polygon, regular polygon sides=3 },
    node rotated/.style = {rotate=90},
    border rotated/.style = {shape border rotate=90},
    dist/.style = {triangle,draw,border rotated, inner sep=0pt},
    kernel/.style={rectangle,draw,inner sep = 2pt},
    expectation/.style = {triangle,draw,inner sep=0pt,shape border rotate=270}}

\newcommand\DCI{
	\begin{tikzpicture}[scale=0.35]
	\draw[->] (1,0) -- (0,0);
	\draw (0.6,0) -- (0.6,0.75);
	\draw (0.4,0) -- (0.4,0.75);
	\end{tikzpicture}
}

\newcommand\splitter[1]{%
\begin{tikzpicture}[scale=#1]
\draw (0,-1) -- (0,0);
\draw (0,0) to [bend right] (1,1);
\draw (0,0) to [bend left] (-1,1);
\end{tikzpicture}
}

\newcommand\stopper[1]{%
\begin{tikzpicture}[scale=#1]
\draw[-{Rays [n=8]}] (0,-1) -- (0,0);
\end{tikzpicture}
}

\DeclareMathOperator*{\argmax}{arg\,max}
\DeclareMathOperator*{\argmin}{arg\,min}
\DeclareMathOperator*{\arginf}{arg\,inf}
\DeclareMathOperator*{\argsup}{arg\,sup}

\newcommand{\cheng}[1]{ {\color{purple}[{\bf Cheng:~{#1}}]} }

\title{Thesis Proposal Review: How Hard is a Causal Inference Problem}

\author{ David Johnston }

\begin{document}

\maketitle

% \begin{abstract}
% We develop \emph{causal statistical decison theory} (CSDT) a novel theory of causal inference which we derive by introducing the idea that ``decisions have consequences'' to statistical decision theory. CSDT features \emph{causal theories} as the central object of study. We show that causal Bayesian networks have a natural representation as a causal theory and that potential outcomes models may arguably be represented as causal theories as well. In both cases the resulting theories feature unreasonably rich sets of decisions, which we suggest is because both approaches aim to produce reusable causal models. Using causal theories, we investigate reusability -- when can knowledge gained using one causal theory be applied to another -- and show that this is possible when the theories are related by a \emph{coarsening}.
% \end{abstract}


\section{Introduction}

The decision theoretic approach to statistics plays a role of fundamental importance in modern machine learning; loss functions underpin the development of algorithms, and the analysis of losses is critical to the theoretical treatment of learning algorithms.

Problems of causal inference make different demands to statistical problems, though statistical inference often plays a role in addressing them (\cite{pearl_causality:_2009}). There are two broad approaches to causality, one based on graphical models and one based on the joint distribution of observations and ``potential outcomes''. While both approaches can be applied to decision making problems, such applications require decision theoretic interpretation of notions more basic to the causal frameworks such as ``causal effect'' and ``treatment effect''.

The identification of causal effects under either framework typically requires strong assumptions that are untestable and do not hold generically. This is the case, for example, for \emph{conditional ignorability} under potential outcomes (\cite{gordon_comparison_2018, heckman_randomization_1991}), or for assessing whether an appropriate causal graph admits the identifiability of some effect (\cite{tian2002general}).The evaluation of causal effects, therefore, usually requires expert input in order to assess these assumptions. 

In contrast, substantial progress in machine learning has been the result of developing generic principles and learning techniques that are relevant to many datasets from many domains and are less reliant on the judgement of domain experts. There is substantial interest in discovering and studying analogous generic principles of causal inference. Generic approaches cannot be expected to yield results as strong as approaches that involve expert input. For example, the non-generic assumption of ignorability is often licenced by knowledge of experimental design that is not itself to be found in the data in question, so any learner equipped with only generic assumptions and the data in question must be starting at a disadvantage. Nonetheless, this does not exclude the possibility that generic approaches may yield nontrivial results in the right circumstances.

There are two main assumptions used so far in pursuing generic causal inference: \emph{faithfulness} and the \emph{independence of cause and mechanism}. The assumption of faithfulness (together with the Causal Markov Condition) facilitates the exclusion of some graphical models on the basis of conditional independences found in the data \citep{spirtes_causation_1993} while the principle of independence of cause and mechanism enables the use of a number of special purpose techniques to assess the \emph{algorithmic independence} of marginal and conditional distributions which leads to preferment of some graphical models over others \citep{lemeire_replacing_2013, peters_identifiability_2012}.



Both faithfulness and the independence of cause and mechanism are employed in learning causal Bayesian networks. While these are undoubtedly useful tools for causal reasoning, causal Bayesian networks are not ideal objects for the analysis of causal learning at a very general level:
\begin{itemize}
    \item There are causal models which cannot be captured by a DAG \citep{dawid_beware_2010,bongers_theoretical_2016}
    \item There is controversy over how causal Bayesian networks should be adapted to answer counterfactual questions \citep{richardson2013single}
    \item Under appropriate conditions, different graphs may induce the same set of interventional distributions \citep{peters_structural_2015}
    \item A standard causal Bayesian network posits an intervention operation for every variable under consideration, while the actions to be evaluated may be much more limited. Learning such a graph appears to run afoul of Vapnik's precept: \emph{When solving a given problem, try to avoid solving a more general problem as an intermediate step} \citep{vapnik_nature_2013}
\end{itemize}

We propose a general account of causal inference problems motivated by a decision theoretic approach. While we do not claim that such an approach subsumes every causal question that someone may be interested in, we do claim that a very large number of causal questions have a natural decision theoretic formulation.

Many problems of causal inference are concerned with the consequences of decisions; for example, whether a treatment will promote a patient's recovery, whether a policy will, on average, benefit those affected by it or whether inactivating a particular gene will have a desired effect on the resulting organism. Others may be concerned with the comparison of actual consequences and counterfactual states, such as the test in tort law that asks ``but for the defendant's negligence, would the plaintiff have been injured''? 

We take the view that in both cases, given preferences over decisions, consequences (and, if necessary, counterfactuals), a good decision is a sufficient answer to a causal inference problem. We propose an approach to causal inference based on extending the classic framework of statistical decision theory developed by \cite{wald_statistical_1950}. A classical statistical decision problem arises when we have access to data $\RV{X}$ generated by some state $\mu\in \mathscr{H}$, have some set of available decisions $\RV{D}$ and can articulate preferences over (state,decision) pairs. In contrast, a causal decision problem arises when we cannot easily articulate preferences over (state, decision) pairs but we can articulate preferences over the possible states of the world that follow in consequence from our decision.

In order to connect decisions with preferences, we therefore require a stochastic mapping from decisions $\RV{D}$ to possible future states. We term this mapping a \emph{consequence}, which can be analogized with the data-generating distribution in statistical decision theory. A \emph{causal theory} relates the states that we may observe with possible consequences.

We develop connections between causal decision problems and statistical decision problems, and show that a causal theory is a generalization of a causal Bayesian network. We show that a CBN, in general, leads to an underspecified causal theory with different choices of theory leading to very different behaviour. The counterfactual case requires an extension of the notion of a consequence, which we connect with a number of existing representations of counterfactual distributions including single world intervention graphs and structural equation models.



%!TEX root = main.tex

\section{Definitions and key notation}\label{sec:dfin}

We use three notations for working with probability theory. The ``elementary'' notation makes use of regular symbolic conventions (functions, products, sums, integrals, unions etc.) along with the expectation operator $\mathbb{E}$. This is the most flexible notation which comes at the cost of being verbose and difficult to read. Secondly, we use a semi-formal string diagram notation extending the formal diagram notation for symmetric monoidal categories \cite{selinger_survey_2010}. Objects in this diagram refer to stochastic maps, and by interpreting diagrams as symbols we can, in theory, be just as flexible as the purely symbolic approach. However, we avoid complex mixtures of symbols and diagrams elements, and fall back to symbolic representations if it is called for. Finally, we use a matrix-vector product convention that isn't particularly expressive but can compactly express some common operations.


\subsection{Standard Symbols}

\begin{center}
\begin{tabular}{ |c|c|c| } 
 \hline
 Symbol & Meaning \\ 
 $[n]$& The natural numbers $\{1,...,n\}$ \\ 
 $f:a\mapsto b$ & Function definition, equivalent to $f(a):=b$\\
 Dots appearing in function arguments: $f(\cdot,\cdot,z)$ & The ``curried'' function $(x, y )\mapsto f(x,y,z)$\\
 Capital letters: $A,B, X$ & sets \\ 
 Script letters: $\mathcal{A},\mathcal{B},\mathcal{X}$ & $\sigma$-algebras on the sets $A, B, X$ respectively\\
 Script $\mathcal{G}$ & A directed acyclic graph made up of nodes $V$ and edges $E$\\
 Blackboard $\prob{P}$, $\mu$, $\nu$: & Probability measures\\
 $\delta_x$ & The Dirac delta measure: $\delta_x(A) = 1$ if $x\in A$ and $0$ otherwise\\
 Capital delta: $\Delta(\mathcal{E})$ & The set of all probability measures on $\mathcal{E}$\\
 Blackboard capitals: $\kernel{A}$ & Markov kernel $\kernel{A}:X\times\mathcal{Y}\to [0,1]$ (stochastic maps)\\
 Subscripted Markov kernels: $\kernel{A}_x$ & The probability measure given by the curried Markov kernel $\kernel{A}(x,\cdot)$\\
 $A\to\Delta(\mathcal{B})$ & Markov kernel signature, treated as equivalent to $A\times \mathcal{B}\to [0,1]$\\
 $\kernel{A}:x\mapsto \nu$ & Markov kernel definition, equivalent to $\kernel{A}(x,B) = \nu(B)$ for all $B$\\
 Sans serif capitals: $\RV{A},\RV{X}$ & Measurable functions; we will also call them random variables\\
 $\kernel{F}^{\RV{X}}$ & The Markov kernel associated with the function $\RV{X}$: $\kernel{F}^{\RV{X}} \equiv a\mapsto \delta_{\RV{X}(a)}$\\
 $\kernel{N}_{\RV{A}|\RV{B}}$ & The conditional probability (disintegration) of $\RV{A}$ given $\RV{B}$ under $\nu$\\
 $\nu \kernel{F}^{\RV{X}}$ & The marginal distribution of $\RV{X}$ under $\nu$\\
 \hline
\end{tabular}
\end{center}

\subsection{Probability Theory}

Given a set $A$, a $\sigma$-algebra $\mathcal{A}$ is a collection of subsets of $A$ where
\begin{itemize}
	\item $A\in \mathcal{A}$ and $\emptyset\in \mathcal{A}$
	\item $B\in \mathcal{A}\implies B^C\in\mathcal{A}$
	\item $\mathcal{A}$ is closed under countable unions: For any countable collection $\{B_i|i\in Z\subset \mathbb{N}\}$ of elements of $\mathcal{A}$, $\cup_{i\in Z}B_i\in \mathcal{A}$ 
\end{itemize}

A measurable space $(A,\mathcal{A})$ is a set $A$ along with a $\sigma$-algebra $\mathcal{A}$. Sometimes the sigma algebra will be left implicit, in which case $A$ will just be introduced as a measurable space.

\paragraph{Common $\sigma$ algebras}

For any $A$, $\{\emptyset,A\}$ is a $\sigma$-algebra. In particular, it is the only sigma algebra for any one element set $\{*\}$.

For countable $A$, the power set $\mathscr{P}(A)$ is known as the discrete $\sigma$-algebra.

Given $A$ and a collection of subsets of $B\subset\mathscr{P}(A)$, $\sigma(B)$ is the smallest $\sigma$-algebra containing all the elements of $B$. 

Let $T$ be all the open subsets of $\mathbb{R}$. Then $\mathcal{B}(\mathbb{R}):=\sigma(T)$ is the \emph{Borel $\sigma$-algebra} on the reals. This definition extends to an arbitrary topological space $A$ with topology $T$.

A \emph{standard measurable set} is a measurable set $A$ that is isomorphic either to a discrete measurable space $A$ or $(\mathbb{R}, \mathcal{B}(\mathbb{R}))$. For any $A$ that is a complete separable metric space, $(A,\mathcal{B}(A))$ is standard measurable. 

Given a measurable space $(E,\mathcal{E})$, a map $\mu:\mathcal{E}\to [0,1]$ is a \emph{probability measure} if
\begin{itemize}
	\item $\mu(E)=1$, $\mu(\emptyset)=0$
	\item Given countable collection $\{A_i\}\subset\mathscr{E}$, $\mu(\cup_{i} A_i) = \sum_i \mu(A_i)$
\end{itemize}

Write by $\Delta(\mathcal{E})$ the set of all probability measures on $\mathcal{E}$.

Given a second measurable space $(F,\mathcal{F})$, a \emph{stochastic map} or \emph{Markov kernel} is a map $\kernel{M}:E\times\mathcal{F}\to [0,1]$ such that
\begin{itemize}
	\item The map $\kernel{M}(\cdot;A):x\mapsto \kernel{M}(x;A)$ is $\mathcal{E}$-measurable for all $A\in \mathcal{F}$
	\item The map $\kernel{M}_x:A\mapsto \kernel{M}(x;A)$ is a probability measure on $F$ for all $x\in E$
\end{itemize}

Extending the subscript notation above, for $\kernel{C}:X\times Y\to \Delta(\mathcal{Z})$  and $x\in X$ we will write $\kernel{C}_x$ for the ``curried'' map $y\mapsto \kernel{C}_{x,y}$.

The map $x\mapsto \kernel{M}_x$ is of type $E\to \Delta(\mathcal{F})$. We will abuse notation somewhat to write $\kernel{M}:E\to \Delta(\mathcal{F})$, which captures the intuition that a Markov kernel maps from elements of $E$ to probability measures on $\mathcal{F}$. Note that we ``reverse'' this idea and consider Markov kernels to map from elements of $\mathcal{F}$ to measurable functions $E\to[0,1]$, an interpretation found in \citet{clerc_pointless_2017}, but (at this stage) we don't make use of this interpretation here.

Given an indiscrete measurable space $(\{*\},\{\{*\},\emptyset\})$, we identify Markov kernels $\kernel{N}:\{*\}\to \Delta(\mathcal{E})$ with the probability measure $\kernel{N}_*$. In addition, there is a unique Markov kernel $\stopper{0.2}:E\to \Delta(\{\{*\},\emptyset\})$ given by $x\mapsto \delta_*$ for all $x\in E$ which we will call the ``discard'' map.


\subsection{Product Notation}\label{ssec:product_notation}

We can use a notation similar to the standard notation for matrix-vector products to represent operations with Markov kernels. Probability measures $\mu\in \Delta(\mathcal{X})$ can be read as row vectors, Markov kernels as matrices and measurable functions $\RV{T}:Y\to T$ as column vectors. Defining $\kernel{M}:X\to \Delta(\mathcal{Y})$ and $\kernel{N}:Y\to \Delta(\mathcal{Z})$, the measure-kernel product $\mu \kernel{A} (G) := \int \kernel{A}_x (G) d\mu(x)$ yields a probability measure $\mu\kernel{A}$ on $\mathcal{Z}$, the kernel-kernel product $\kernel{M}\kernel{N}(x;H)=\int_Y \kernel{B}(y;H)d\kernel{A}_x$ yields a kernel $\kernel{M}\kernel{N}:X\to \Delta(\mathcal{Z})$ and the kernel-function product $\kernel{A}\RV{T}(x):=\int_Y \RV{T}(y) d\kernel{A}_x$ yields a measurable function $X\to T$. Kernel products are associative \citep{cinlar_probability_2011}.

The tensor product $(\kernel{M}\otimes \kernel{N})(x,y;G,H) := \kernel{M}(x;G)\kernel{N}(y;H)$ yields a kernel $(\kernel{M}\otimes \kernel{N}):X\times Y\to \Delta(\mathcal{Y}\otimes\mathcal{Z})$.

\subsection{String Diagrams}\label{ssec:mken_diagrams}

Some constructions are unwieldly in product notation; for example, given $\mu\in \Delta(\mathcal{E})$ and $\kernel{M}:E\to (\mathcal{F})$, it is not straightforward to construct a measure $\nu\in\Delta(\mathcal{E}\otimes\mathcal{F})$ that captures the ``joint distribution'' given by $A\times B\mapsto \int_A \kernel{M}(x;B)d\mu$. 

Such constructions can, however, be straightforwardly captured with string diagrams, a notation developed for category theoretic probability. \citet{cho_disintegration_2019} also provides an extensive introduction to the notation discussed here.

Some key ideas of string diagrams:
\begin{itemize}
	\item Basic string diagrams can always be interpreted as a mixture of kernel-kernel products and tensor products of Markov kernels
	\begin{itemize}
	\item Extended string diagrams can be interepreted as a mixture of kernel-kernel products, kernel-function products, tensor products of kernels and functions and scalar products 
	\end{itemize}
	\item String diagrams are the subject of a coherence theorem: taking a string diagram and applying a planar deformation yields a string diagram that represents the same kernel \citep{selinger_survey_2010}. This also holds for a number of additional transformations detailed below
\end{itemize}

A kernel $\kernel{M}:X\to \Delta(\mathcal{Y})$ is written as a box with input and output wires, probability measures $\mu\in \Delta(\mathcal{X})$ are written as triangles ``closed on the left'' and measurable functions (which are only elements of the ``extended'' notation) $\RV{T}:Y\to T$ as triangles ``closed on the right''. For this introduction we will label wires with the names of their corresponding spaces, but in practice we will usually name them with corresponding \emph{random variables}, though additional care is required when using random variables as labels (see paragraph \ref{par:random_variables}).

For $\kernel{M}:X\to \Delta(\mathcal{Y})$, $\mu\in \Delta(\mathcal{X})$ and $f:X\to W$:

\begin{align}
\begin{tikzpicture}
\path (0,0) node (A) {$X$}
++(0.75,0) node[kernel] (B) {$\kernel{M}$}
++(0.75,0) node (C) {$Y$};
\draw (A) -- (B) -- (C);
\end{tikzpicture}\qquad
\begin{tikzpicture}
\path (0,0) node[dist] (B) {$\mu$}
++(0.75,0) node (C) {$X$};
\draw (B) -- (C);
\end{tikzpicture}\qquad
\begin{tikzpicture}
\path (0,0) node (A) {$X$}
++(0.75,0) node[expectation] (B) {$f$};
\draw (A) -- (B);
\end{tikzpicture}
\end{align}


\paragraph{Elementary operations}

We can compose Markov kernels with appropriate spaces - the equivalent operation of the ``matrix products'' of product notation. Given $\kernel{M}:X\to\Delta(\mathcal{Y})$ and $\kernel{N}:Y\to \Delta(\mathcal{Z})$, we have 

\begin{align}
\kernel{M}\kernel{N} := \begin{tikzpicture}
 \path (0,0) node (E) {$X$}
 ++ (1,0) node[kernel] (M) {$\kernel{M}$}
 ++ (1,0) node[kernel] (N) {$\kernel{N}$}
 ++(1,0) node (G) {$Z$};
 \draw (E) -- (M) -- (N) -- (G);
\end{tikzpicture}\label{eq:sd_composition}
\end{align}

Probability measures are distinguished in that that they only admit ``right composition'' while functions only admit ``left composition''. For $\mu\in \Delta(\mathcal{E})$, $h:F\to X$:

\begin{align}
\mu\kernel{M} &:= \begin{tikzpicture}
 \path (0,0) node[dist] (M) {$\mu$}
 ++ (1,0) node[kernel] (N) {$\kernel{M}$}
 ++(1,0) node (G) {$Z$};
 \draw (M) -- (N) -- (G);
\end{tikzpicture}\\
\kernel{M}f&:= \begin{tikzpicture}
 \path (0,0) node (E) {$X$}
 ++ (1,0) node[kernel] (M) {$\kernel{M}$}
 ++ (1,0) node[expectation] (N) {$f$};
 \draw (E) -- (M) -- (N);
 \end{tikzpicture}
\end{align}

A diagram that is closed on the right and the left is an expectation:

\begin{align}
\mathbb{E}_{\mu \kernel{M}} (f) &= \mu \kernel{M} f\\
		&:= \begin{tikzpicture}
 \path (0,0) (0,0) node[dist] (mu) {$\mu$}
 ++ (1,0) node[kernel] (M) {$\kernel{M}$}
 ++ (1,0) node[expectation] (N) {$f$};
 \draw (mu) -- (M) -- (N);
 \end{tikzpicture}\label{def:diag_expectation}
\end{align}


We can also combine Markov kernels using tensor products, which we represent with vertical juxtaposition. For $\kernel{O}:Z\to \Delta(\mathcal{W})$:


\begin{align}
\kernel{M}\otimes\kernel{N}&:= \begin{tikzpicture}
\path (0,0) node (E) {$X$}
++(1,0) node[kernel] (M) {$\kernel{M}$}
++(1,0) node (F) {$Y$}
(0,-0.5) node (F1) {$Z$}
++(1,0) node[kernel] (N) {$\kernel{O}$}
+(1,0) node (G) {$W$};
\draw (E) -- (M) -- (F);
\draw (F1) -- (N) -- (G);
\end{tikzpicture}
\end{align}

Product spaces can be represented either by two parallel wires or a single wire:
\begin{align}
X\times Y \cong \mathrm{Id}_X\otimes \mathrm{Id}_Y &:= \begin{tikzpicture}
\path (0,0) node (E) {$X$}
++(1,0) node (F) {$X$}
(0,-0.5) node (F1) {$Y$}
+(1,0) node (G) {$Y$};
\draw (E) -- (F);
\draw (F1) -- (G);
\end{tikzpicture}\\
&= \begin{tikzpicture}
\path (0,0) node (X) {$X\times Y$}
++(2,0) node (Y) {$X\times Y$};
\draw (X) -- (Y);
\end{tikzpicture}
\end{align}

Because a product space can be represented by parallel wires, a kernel $\kernel{L}:X\to \Delta(\mathcal{Y}\otimes\mathcal{Z})$ can be written using either two parallel output wires or a single output wire:

\begin{align}
&\begin{tikzpicture}
\path (0,0) node (E) {$X$}
++ (1,0) node[kernel] (L) {$\kernel{L}$}
++ (1,0.15) node (F) {$Y$}
+(0,-0.3) node (G) {$Z$};
\draw (E) -- (L);
\draw ($(L.east) + (0,0.15)$) -- (F);
\draw ($(L.east)+ (0,-0.15)$) -- (G);
\end{tikzpicture}\\
&\equiv\\
&\begin{tikzpicture}
\path (0,0) node (E) {$X$}
++ (1,0) node[kernel] (L) {$\kernel{L}$}
++ (1.5,0) node (F) {$Y\times Z$};
\draw (E) -- (L) -- (F);
\end{tikzpicture}
\end{align}

\paragraph{Probability measures, Markov kernels and functions}

One has to exercise special care when including functions in diagrammatic notation. While any diagram that includes only probability measures (triangles pointing to the left) and Markov kernels (rectangles) is automatically a Markov kernel itself, while diagrams that include functions (triangles pointing to the right) only represent Markov kernels if they are correctly normalised, which is not a property that can be checked just by looking at the shape of the diagram.

\paragraph{Markov kernels with special notation}

A number of Markov kernels are given special notation distinct from the generic ``box'' representation above. These special representations facilitate intuitive graphical interpretations.

The identity kernel $\textbf{Id}:X\to \Delta(X)$ maps a point $x$ to the measure $\delta_x$ that places all mass on the same point:

\begin{align}
\textbf{Id} : x\mapsto \delta_x \equiv \begin{tikzpicture}\path (0,0) node (X) {$X$} + (1,0) node (X1) {$X$}; \draw (X)--(X1); \end{tikzpicture}\label{eq:identity}
\end{align}

The identity kernel acts as the identity under left or right products:

\begin{align}
	(\kernel{K}\textbf{Id})_w(A) &= \int_X \textbf{Id}_x(A) d\kernel{K}_w (x) \\
							 	 &= \int_X \delta_x(A) d\kernel{K}_w(x)\\
							 	 &= \int_A d\kernel{K}_w(x)\\
							 	 &= \kernel{K}_w(A)\\
	(\textbf{Id}\kernel{K})_w(A) &= \int_X \kernel{K}_x (A) d\textbf{Id}_w(x)\\
								 &= \int_X  \kernel{K}_x(A) d\delta_w(x)\\
								 &= \kernel{K}_w(A)								  
\end{align}

The copy map $\splitter{0.1}:X\to \Delta(\mathcal{X}\times \mathcal{X})$ maps a point $x$ to two identical copies of x:
\begin{align}
 \splitter{0.1}: x\mapsto \delta_{(x,x)} \equiv \begin{tikzpicture}
 \path (0,0) node (X) {$X$} ++ (0.5,0) coordinate (copy0) ++ (0.5,0.25) node (X1) {$X$} ++(0,-0.5) node (X2) {$X$};\draw (X)--(copy0) to [bend left] (X1) (copy0) to [bend right] (X2);
 \end{tikzpicture}\label{eq:copy}
 \end{align} 

The copy map ``copies'' its arguments under an integral:

\begin{align}
	\int_(X\times X) f(x,x',x'') d\splitter{0.1}_x (x',x'') &= \int_(X\times X) f(x,x',x'') d\delta_{(x,x)}(x',x'')\\
															&= f(x,x,x)\\
	\int_W \int_(X\times X) f(x',x'')d\splitter{0.1}_w (x',x'') d\mu(w)\\
															&= \int_W f(w,w) d\mu(w)
\end{align}

The swap map $\sigma:X\times Y\to \Delta(\mathcal{Y}\otimes\mathcal{X})$ swaps its inputs:

\begin{align}
\sigma := (x,y)\to \delta_{(y,x)} \equiv \begin{tikzpicture}
\path (0,0) node (X) {$X$}
+(1,0.3) node (X1) {$X$}
(0,0.3) node (Y) {$Y$}
+(1,-0.3) node (Y1) {$Y$};
\draw (X)--(X1) (Y) -- (Y1);
\end{tikzpicture}\label{eq:swap}
\end{align}

The swap map swaps its arguments under an integral:

\begin{align}
	\int_(X\times X) f(x,x') d\sigma_{(x_0,x_1)}(x,x') &= \int_(X\times X) f(x,x') d\delta_{(x_1,x_0)}(x,x')\\
													   &= f(x_1,x_0)
\end{align}

The discard map $\stopper{0.2}:X\to \Delta(\{*\})$ maps every input to $\delta_{*}$. Note that the only non-empty event in $\{\emptyset,\{*\}\}$ must have probability 1.
\begin{align}
\stopper{0.2}: x\mapsto \delta_{*} \equiv \begin{tikzpicture}
 \draw[-{Rays [n=8]}] (0,0) node (X) {$X$} (X) -- (1,0);
\end{tikzpicture}\label{eq:discard}
\end{align}

Any measurable function $F\to X$ has an associated Markov kernel $F\to \Delta(\mathcal{X})$. The Markov kernel associated with a function is different to the function itself - while the product of a probability measure $\mu$ with a function $f$ is an expectation $\mu f$ (see Definition \ref{def:diag_expectation}), the product of a probability measure with the associated Markov kernel is the pushforward measure $f_\# \mu$.

\begin{definition}[Function induced kernel]\label{def:functional_kernel}
Given a measurable function $g:F\to X$, define the function induced kernel $\kernel{F}^{g}:F\to \Delta(\mathcal{X})$ to be the the Markov kernel $a\mapsto \delta_{g(a)}$ for all $a\in X$.
\end{definition}

\begin{definition}[Pushforward kernel]
Given a kernel $\kernel{M}:E\to \Delta(\mathcal{F})$ and a measurable function $g:F\to X$, the \emph{pushforward kernel} $g_\# \kernel{M}:E\to \Delta(\mathcal{X})$ is the kernel such that $g_\# \kernel{M} (a;B) = \kernel{M}(a;g^{-1}(B))$.

If $E$ is the one element space $\{*\}$, then $\kernel{M}:\{*\}\to \Delta(\mathcal{F})$ can be identified with the probability measure $\kernel{M}_*$ and the pushforward kernel $g_{\#}\kernel{M}$ identified with the pushforward measure $g_{\#} \kernel{M}_*$, so pushforward kernels reduce to pushforward measures.
\end{definition}

\begin{lemma}[Pushforward kernels are functional kernel products]\label{lem:pushf_funk}
Given a kernel $\kernel{M}:E\to \Delta(\mathcal{F})$ and a measurable function $g:F\to X$,the pushforward $g_\# \kernel{M} = \kernel{M} \kernel{F}^{g}$.
\end{lemma}

\begin{proof}
\begin{align}
	\kernel{M}\kernel{F}^g(a;B) &= \int_F \delta_{g(y)}(B) d\kernel{M}_a(y)\\
								&= \int_F \delta_{y}(g^{-1}(B)) d\kernel{M}_a(y)\\
								&= \int_{g^{-1}(B)} d\kernel{M}_a(y)\\
								&= g_{\#} \kernel{M} (a;B)
\end{align}
\end{proof}


\subsubsection{Comparison of notations}

We are in a position to compare the three introduced notations using a few examples. Given $\mu\in\Delta(X),\kernel{A}:X\to \Delta(Y)$ and $A\in \mathcal{X}$, $B\in\mathcal{Y}$, the following correspondences hold, where we express the same object in elementary notation, product notation and string notation respectively:

\begin{align}
\nu:=A\times B\mapsto \int_A A(x;B)d\mu(x) \equiv \mu \splitter{0.1}(\textbf{Id}_X\otimes \kernel{A}) \equiv  \begin{tikzpicture}
\path (0,0) node[dist] (mu) {$\mu$}
++ (1,0) coordinate (copy0)
+ (1.2,0.5) node (X) {$X$}
++ (0.5,-0.5) node[kernel] (A) {$\kernel{A}$}
++(0.7,0) node (Y) {$Y$};
\draw (mu)--(copy0);
\draw (copy0) to [bend left] (X);
\draw (copy0) to [bend right] (A) (A) -- (Y);
\end{tikzpicture}\label{eq:joint_measure}
\end{align}

Where the resulting object is a probability measure $\nu\in \Delta(\mathcal{X}\otimes\mathcal{Y})$. Note that the elementary notation requires a function definition here, while the product and string notations can represent the measure without explicitly addressing its action on various inputs and outputs. \citet{cho_disintegration_2019} calls this construction ``integrating $\kernel{A}$ with respect to $\mu$''.

Define the marginal $\nu_Y\in \Delta(\mathcal{Y}):B\mapsto \nu(X\times B)$ for $B\in \mathcal{Y}$ and similarly for $\nu_X$. We can then express the result of marginalising \ref{eq:joint_measure} over $X$ in our three separate notations as follows:
\begin{align}
  \nu_Y (B) &= \nu(X\times B) = \int_X A(x;B) d\mu(x)\label{eq:marginalisation_elem}\\
  \nu_Y &= \mu \kernel{A} = \mu \splitter{0.1}(\textbf{Id}_X\otimes \kernel{A})(\stopper{0.2}\otimes \textbf{Id}_Y)\label{eq:marginalisation_prod}\\
  \nu_Y &= \begin{tikzpicture}
\path (0,0) node[dist] (mu) {$\mu$} ++ (1,0) node[kernel] (A) {$\kernel{A}$} ++ (0.7,0) node (Y) {$Y$}; \draw (mu) -- (A) -- (Y);
\end{tikzpicture} = \begin{tikzpicture}
\path (0,0) node[dist] (mu) {$\mu$}
++ (1,0) coordinate (copy0)
+ (1.2,0.5) node (X) {}
++ (0.5,-0.5) node[kernel] (A) {$\kernel{A}$}
++(0.7,0) node (Y) {$Y$};
\draw (mu)--(copy0);
\draw[-{Rays [n=8]}] (copy0) to [bend left] (X);
\draw (copy0) to [bend right] (A) (A) -- (Y);
\end{tikzpicture}\label{eq:marginalisation_graph}
\end{align}

The elementary notation \ref{eq:marginalisation_elem} makes the relationship between $\nu_Y$ and $\nu$ explicit and, again, requires the action on each event to be defined. The product notation \ref{eq:marginalisation_prod} is, in my view, the least transparent but also the most compact in the form $\mu \kernel{A}$, and does not demand the explicit definition of how $\nu_Y$ treats every event. The graphical notation is the least compact in terms of space taken up on the page, but unlike the product notation it shows a clear relationship to the graphical construction in\ref{eq:joint_measure}, and displays a clear graphical logic whereby marginalisation corresponds to ``cutting off branches''. Like product notation, it also allows for the definition of derived measures such as $\nu_Y$ without explicit definition of the handling of all events. It also features a much smaller collection of symbols than does elementary notation.

String diagrams often achieve a good balance between being ease of understanding at a glance and expressive power. On the downside, they can be time consuming to typeset, and formal reasoning with them takes some practice.

\subsubsection{Working With String Diagrams}\label{sssec:string_diagram_manipulation}

todo:
\begin{itemize}
\item Functional generalisation
\item Conditioning
\item Infinite copy map
\item De Finetti's representation theorem
\end{itemize}

There are a relatively small number of manipulation rules that are useful for string diagrams. In addition, we will define graphically analogues of the standard notions of \emph{conditional probability}, \emph{conditioning}, and infinite sequences of exchangeable random variables.

\paragraph{Axioms of Symmetric Monoidal Categories}

For the following, we either omit labels or label diagrams with their domain and codomain spaces, as we are discussing identities of kernels rather than identities of components of a condtional probability space. Recalling the unique Markov kernels defined above, the following equivalences, known as the \emph{commutative comonoid axioms}, hold among string diagrams:

\begin{align}
	\begin{tikzpicture}[scale=0.8]
	\path (0,0) node (X) {} 
	++ (0.5,0) coordinate (copy0)
	+ (1.5,0.5) node (X1) {}
	++ (0.5,-0.5) coordinate (copy1)
	+(1,0.5) node (X2) {}
	+(1,-0.5) node (X3) {};
	\draw (X) -- (copy0) to [bend left] (X1) (copy0) to [bend right] (copy1) to [bend left] (X2) (copy1) to [bend right] (X3);
	\end{tikzpicture}
	=
	\begin{tikzpicture}[scale=0.8]
	\path (0,0) node (X) {} 
	++ (0.5,0) coordinate (copy0)
	+ (1.5,-0.5) node (X1) {}
	++ (0.5,0.5) coordinate (copy1)
	+(1,0.5) node (X2) {}
	+(1,-0.5) node (X3) {};
	\draw (X) -- (copy0) to [bend right] (X1) (copy0) to [bend left] (copy1) to [bend left] (X2) (copy1) to [bend right] (X3);
	\end{tikzpicture}
	:=
	\begin{tikzpicture}[scale=0.8]
	\path (0,0) node (X) {} 
	++ (0.5,0) coordinate (copy0)
	+ (1,0.5) node (X1) {}
	+(1,0) node (X2) {}
	+(1,-0.5) node (X3) {};
	\draw (X) -- (copy0) to [bend left] (X1) (copy0) to (X2) (copy0) to [bend right] (X3);
	\end{tikzpicture}\label{eq:ccom1}
\end{align}

\begin{align}
	\begin{tikzpicture}[scale=0.8]
	\path (0,0) node (X) {}
	++(0.5,0) coordinate (copy0)
	+ (1,0.5) node (S) {}
	+(1,-0.5) node (X1) {};
	\draw (X) -- (copy0) to [bend right] (X1);
	\draw[-{Rays [n=8]}] (copy0) to [bend left] (S);
	\end{tikzpicture}
	= 
	\begin{tikzpicture}[scale=0.8]
	\path (0,0) node (X) {}
	++(0.5,0) coordinate (copy0)
	+ (1,-0.5) node (S) {}
	+(1,0.5) node (X1) {};
	\draw (X) -- (copy0) to [bend left] (X1);
	\draw[-{Rays [n=8]}] (copy0) to [bend right] (S);
	\end{tikzpicture}
	=
	\begin{tikzpicture}[scale=0.8]
	\path (0,0) node (X) {}
	++ (1,0) node (X1) {};
	\draw (X) -- (X1);
	\end{tikzpicture}\label{eq:ccom2}
\end{align}

\begin{align}
	\begin{tikzpicture}[scale=0.8]
	\path (0,0) node (X) {$\RV{X}$}
	++(0.5,0) coordinate (copy0)
	+ (1,0.5) node (X2) {$\RV{X}$}
	+(1,-0.5) node (X1) {$\RV{X}$};
	\draw (X) -- (copy0) to [bend right] (X1);
	\draw (copy0) to [bend left] (X2);
	\end{tikzpicture}
=
	\begin{tikzpicture}[scale=0.8]
	\path (0,0) node (X) {}
	++(0.5,0) coordinate (copy0)
	+ (1.2,0.5) node (X2) {}
	+(1.2,-0.5) node (X1) {};
	\draw (X) -- (copy0) .. controls (0.75,0.4) .. (X1.west);
	\draw (copy0) .. controls (0.75,-0.4) .. (X2.west);
	\end{tikzpicture}
\label{eq:ccom3}
\end{align}

The discard map $\stopper{0.2}$ can ``fall through'' any Markov kernel:

\begin{align}
\begin{tikzpicture}
\path (0,0) node (X) {}
++(0.7,0) node[kernel] (A) {$\kernel{A}$}
++(0.7,0) node (S) {};
\draw (X) -- (A);
\draw[-{Rays [n=8]}] (A) -- (S);
\end{tikzpicture}
= 
\begin{tikzpicture}
\path (0,0) node (X) {}
++(0.7,0) node (S) {};
\draw[-{Rays [n=8]}] (X) -- (S);
\end{tikzpicture}\label{eq:termobj1}
\end{align}

Combining \ref{eq:ccom2} and \ref{eq:termobj1} we can derive the following: integrating $\kernel{A}:X\to \Delta(\mathcal{Y})$ with respect to $\mu\in\Delta(\mathcal{X})$ and then discarding the output of $\kernel{A}$ leaves us with $\mu$:

\begin{align}
\begin{tikzpicture}
\path (0,0) node[dist] (mu) {$\mu$}
++ (1,0) coordinate (copy0)
+ (1.4,0.5) node (X) {}
++ (0.7,-0.5) node[kernel] (A) {$\kernel{A}$}
++(0.7,0) node (Y) {};
\draw (mu)--(copy0);
\draw (copy0) to [bend left] (X);
\draw[-{Rays [n=8]}] (copy0) to [bend right] (A) (A) -- (Y);
\end{tikzpicture}
= 
\begin{tikzpicture}
\path (0,0) node[dist] (mu) {$\mu$}
++ (1,0) coordinate (copy0)
+ (1.2,0.5) node (X) {}
++ (0.4,-0.3) coordinate (A)
++(0.1,0) node (Y) {};
\draw (mu)--(copy0);
\draw (copy0) to [bend left] (X);
\draw[-{Rays [n=8]}] (copy0) to [bend right] (A) (A) -- (Y);
\end{tikzpicture}
=
\begin{tikzpicture}
\path (0,0) node[dist] (mu) {$\mu$}
++ (1,0) node (X) {};
\draw (mu)--(X);
\end{tikzpicture}
\end{align}

In elementary notation, this is equivalent to the fact that, for all $B\in \mathcal{X}$, $\int_B \kernel{A}(x;B)d\mu(x) = \mu(B)$.

The following additional properties hold for $\stopper{0.2}$ and $\splitter{0.1}$:

\begin{align}
\begin{tikzpicture}
\path (0,0) node (XY) {$X\times Y$}
++ (1.5,0) node (Z) {};
\draw[-{Rays [n=8]}] (XY) -- (Z);
\end{tikzpicture} &=
\begin{tikzpicture}
\path (0,0) node (X) {$X$} 
++ (1,0) node (X1) {}
(0,-0.3) node (Y) {$Y$}
++ (1,0) node (Y1) {};
\draw[-{Rays [n=8]}] (X) -- (X1);
\draw[-{Rays [n=8]}] (Y) -- (Y1);
\end{tikzpicture}
\end{align}
\begin{align}
\begin{tikzpicture}
\path (0,0) node (XY) {$X\times Y$}
++ (1.2,0) coordinate (copy0)
++(1.2,0.3) node (XY1) {$X \times Y$}
++(0,-0.6) node (XY2) {$X\times Y$};
\draw (XY) -- (copy0) to [bend left] (XY1);
\draw (XY) -- (copy0) to [bend right] (XY2);
\end{tikzpicture} &=
\begin{tikzpicture}
\path (0,0) node (XY) {$X$}
++ (1.,0) coordinate (copy0)
++(1.,0.5) node (XY1) {$X$}
++(0,-1) node (XY2) {$X$}
(0,-0.3) node (F) {$Y$}
++(1.,0) coordinate (copy1)
++(1.,0.5) node (F1) {$Y$}
++(0,-1) node (F2) {$Y$};
\draw (XY) -- (copy0) to [bend left] (XY1);
\draw (copy0) to [bend right] (XY2);
\draw (F) -- (copy1) to [bend left] (F1);
\draw (copy1) to [bend right] (F2);
\end{tikzpicture}
\end{align}

A key fact that \emph{does not} hold in general is

\begin{align}
 \begin{tikzpicture}
\path (0,0) node (E) {}
++ (0.7,0) node[kernel] (A) {$\kernel{A}$}
++(0.7,0) coordinate (copy0)
++(0.5,0.3) node (F1) {}
+(0,-0.6) node (F2) {};
\draw (E) -- (A) -- (copy0) to [bend left] (F1);
\draw (copy0) to [bend right] (F2);
\end{tikzpicture} 
=
\begin{tikzpicture}
\path (0,0) node (E) {}
++(0.5,0) coordinate (copy0)
++(0.7,0.3) node[kernel] (A1) {$\kernel{A}$}
+(0,-0.6) node[kernel] (A2) {$\kernel{A}$}
++(0.75,0) node (F1) {}
+(0,-0.6) node (F2) {};
\draw (E) -- (copy0) to [bend left] (A1) (A1) -- (F1);
\draw (copy0) to [bend right] (A2) (A2) -- (F2);
\end{tikzpicture}
\label{eq:copy_commutes}
\end{align}

In fact, it holds only when $\kernel{A}$ is a \emph{deterministic} kernel.

\begin{definition}[Deterministic Markov kernel]
A \emph{deterministic} Markov kernel $\kernel{A}:E\to \Delta(\mathcal{F})$ is a kernel such that $\kernel{A}_x(B)\in\{0,1\}$ for all $x\in E$, $B\in\mathcal{F}$.
\end{definition}

\begin{theorem}[Copy map commutes for deterministic kernels \citep{fong_causal_2013}]
Equation \ref{eq:copy_commutes} holds iff $\kernel{A}$ is deterministic.
\end{theorem}


\subsection{Random Variables}\label{ssec:random_variables}

The summary of this section is:
\begin{itemize}
\item Random variables are usually defined as measurable functions on a \emph{probability space}
\item It's possible to define them as measurable functions on a \emph{Markov kernel space} instead
\item It is useful to label wires with random variable names instead of names of spaces
\end{itemize}

Probability theory is primarily concerned with the behaviour of \emph{random variables}. This behaviour can be analysed via a collection of probability measures and Markov kernels representing joint, marginal and conditional distributions of random variables of interest. In the framework developed by Kolmogorov, this collection of joint, marginal and conditional distributions is modeled by a single underlying \emph{probability space}, and random variables by measurable functions on the probability space. 

We use the same approach here, with a couple of additions. We are interested in variables whose outcomes depend both on random processes and decisions. Suppose that given a particular distribution over decision variables, a probability distribution over the decision variables and random variables is obtained. Such a model is described by a Markov kernel rather than a probability distribution. We therefore investigate \emph{Markov kernel spaces}.

In the graphical notation that we are using, random variables can be thought of as a means of assigning unambiguous names to each wire in a set of diagrams. In order to do this, it is necessary to suppose that all diagrams in the set describe properties of an \emph{ambient Markov kernel} or \emph{ambient probability measure}. Consider the following example with the ambient probability measure $\mu\in\Delta(\mathcal{X}\otimes\mathcal{X})$. Suppose we have a Markov kernel $\kernel{K}:X\to \Delta(\mathcal{X})$ such that the following holds:

\begin{align}
\begin{tikzpicture}
\path (0,0) node[dist] (m) {$\mu$}
++ (0.7,0.15) node (E) {$X$}
++ (0,-0.3) node (F) {$X$};
\draw ($(m.east) + (0,0.15)$) -- (E);
\draw ($(m.east) + (0,-0.15)$) -- (F);
\end{tikzpicture} = \begin{tikzpicture}
\path (0,0) node[dist] (m) {$\mu$}
++ (0.7,0.15) coordinate (copy0)
+(0,-0.3) node (Fs) {}
++ (1.2,0) node (E) {$X$}
++(-0.7,-0.3) node[kernel] (K) {$\kernel{K}$}
++(0.7,0) node (F) {$X$};
\draw ($(m.east) + (0,0.15)$) -- (E);
\draw (copy0) to [bend right] (K) (K) -- (F);
\draw[-{Rays [n=8]}] ($(m.east) + (0,-0.15)$) -- (Fs);
\end{tikzpicture}\label{eq:disint_example}
\end{align}

Suppose that we also assign the names $\RV{X}_1$ to the upper output wire and $\RV{X}_2$ to the lower output wire in the diagram of $\mu$:

\begin{align}
\begin{tikzpicture}
\path (0,0) node[dist] (m) {$\mu$}
++ (0.7,0.15) node (E) {$\RV{X}_1$}
++ (0,-0.3) node (F) {$\RV{X}_2$};
\draw ($(m.east) + (0,0.15)$) -- (E);
\draw ($(m.east) + (0,-0.15)$) -- (F);
\end{tikzpicture}
\end{align}

Then it seems sensible to call $\kernel{K}$ ``the probability of $\RV{X}_2$ given $\RV{X}_1$''. We will make this precise so that it matches the usual notion of the probability of one variable given another (see \citet{cinlar_probability_2011} for a definition of this usual notion). 

\begin{definition}[Probability space, Markov kernel space]
A \emph{Markov kernel space} $(\kernel{K},\Omega,\mathcal{F},D,\mathcal{D})$ is a Markov kernel $\kernel{K}:D\to \Delta(\mathcal{D}\otimes\mathcal{F})$, called the \emph{ambient kernel}, along with the sample space $(\Omega,\mathcal{F})$ and the domain $(D,\mathcal{D})$. We suppose that $\kernel{K}$ is such that there exists a \emph{fundamental kernel} $\kernel{K}_0$ satisfying

\begin{align}
\prob{K} := \begin{tikzpicture}
\path (0,0) node (O) {}
++(0.5,0) coordinate (copy0)
++ (0.5,0) node[kernel] (m) {$\kernel{K}_0$}
++ (0.7,0.) node (E) {}
++(0,-0.45) node (G) {};
\draw (O) -- (m) -- (E);
\draw (copy0) to [bend right] (G);
\end{tikzpicture}
\end{align}

For brevity, we will omit the $\sigma$-algebras in further definitions of Markov kernel spaces: $(\kernel{K},\Omega,D)$.

A \emph{probability space} $(\prob{P},\Omega,\mathcal{F})$ is a probability measure $\prob{P}:\Delta(\Omega)$, which we call the \emph{ambient measure}, along with the \emph{sample space} $\Omega$ and the \emph{events} $\mathcal{F}$. A probability space is equivalent to a Markov kernel space with domain $D=\{*\}$ - note that $\Omega\times \{*\}\cong \Omega$.
\end{definition}

\begin{definition}[Random variable]\label{def:random_variable}
Given a Markov kernel space $(\kernel{K},\Omega,D)$, a random variable $\RV{X}$ is a measurable function $\Omega\times D\to E$ for arbitrary measurable $E$.
\end{definition}

\begin{definition}[Domain variable]\label{def:domain_variable}
Given a Markov kernel space $(\kernel{K},\Omega,D)$, the \emph{domain variable} $\RV{D}:\Omega\times D\to D$ is the distinguished random variable $\RV{D}:(x,d)\mapsto d$.
\end{definition}

Unlike random variables on probability spaces, random variables on Markov kernel spaces do not generally have unique marginal distributions. An analogous operation of \emph{marginalisation} can be defined, but the result is generally a Markov kernel. We will define marginalisation via coupled tensor products.

\begin{definition}[Coupled tensor product $\utimes$]\label{def:ctensor}
Given two Markov kernels $\kernel{M}$ and $\kernel{N}$ or functions $f$ and $g$ with shared domain $E$, let $\kernel{M}\utimes\kernel{N}:=\splitter{0.1}(\kernel{M}\otimes\kernel{N})$ and $f\utimes g:=\splitter{0.1}(f\otimes g)$ where these expressions are interpreted using standard product notation. Graphically:

\begin{align}
\kernel{M}\utimes\kernel{N}&:=\begin{tikzpicture}
\path (0,0) node (E) {$E$}
++(0.5,0) coordinate (copy0)
+ (0.5,0.3) node[kernel] (M) {$\kernel{M}$}
+(1.2,0.3) node (X) {$\RV{X}$}
+ (0.5,-0.3) node[kernel] (N) {$\kernel{N}$}
+(1.2,-0.3) node (Y) {$\RV{Y}$};
\draw (E) -- (copy0) to [bend left] (M) (copy0) to [bend right] (N);
\draw (M) -- (X) (N) -- (Y);
\end{tikzpicture}\\
f\utimes g&:= \begin{tikzpicture}[scale=1.2]\path (0,0) node (E) {$E$}
++(0.5,0) coordinate (copy0)
+ (0.5,0.3) node[expectation] (M) {$f$}
+ (0.5,-0.3) node[expectation] (N) {$g$};
\draw (E) -- (copy0) to [bend left] (M) (copy0) to [bend right] (N);
\end{tikzpicture}
\end{align}
The operation denoted by $\utimes$ is associative (Lemma \ref{lem:utimes_assoc}), so we can without ambiguity write $f\utimes g\utimes h=(f\utimes g)\utimes h = f\utimes(g\utimes h)$ for finite groups of functions or Markov kernels sharing a domain. 

The notation $\utimes_{i\in [N]} f_i$ is taken to mean $f_1\utimes f_2\utimes ...\utimes f_N$.
\end{definition}

\begin{lemma}[$\utimes$ is associative]\label{lem:utimes_assoc}
For Markov kernels $\kernel{L}:E\to \delta(\mathcal{F})$, $\kernel{M}:E\to \delta(\mathcal{G})$ and $\kernel{N}:E\to \delta(\mathcal{H})$, $(\kernel{L}\utimes\kernel{M})\utimes\kernel{N}=\kernel{L}\utimes(\kernel{M}\utimes\kernel{N})$.
\end{lemma}

\begin{proof}

\begin{align}
	\kernel{L}\utimes(\kernel{M}\utimes\kernel{N}) &= 
	\begin{tikzpicture}[scale=0.8]
	\path (0,0) node (X) {$E$} 
	++ (0.8,0) coordinate (copy0)
	+ (1.5,0.5) node[kernel] (X1) {$\kernel{L}$} + (2.5,0.5) node (F) {$F$}
	++ (0.5,-0.5) coordinate (copy1)
	+(1,0.3) node[kernel] (X2) {$\kernel{M}$} + (2,0.3) node (G) {$G$}
	+(1,-0.5) node[kernel] (X3) {$\kernel{N}$} + (2,-0.5) node (H) {$H$};
	\draw (X) -- (copy0) to [bend left] (X1) (copy0) to [bend right] (copy1) to [bend left] (X2) (copy1) to [bend right] (X3);
	\draw (X1) -- (F) (X2) -- (G) (X3) -- (H);
	\end{tikzpicture}\\
	&=
	\begin{tikzpicture}[scale=0.8]
	\path (0,0) node (X) {$E$} 
	++ (0.8,0) coordinate (copy0)
	+ (1.5,0.7) node[kernel] (X1) {$\kernel{L}$} + (2.5,0.7) node (F) {$F$}
	+ (0.5,0.3) coordinate (copy1)
	++ (0.5,-0.5) coordinate (next)
	+(1,0.5) node[kernel] (X2) {$\kernel{M}$} + (2,0.5) node (G) {$G$}
	+(1,-0.5) node[kernel] (X3) {$\kernel{N}$} + (2,-0.5) node (H) {$H$};
	\draw (X) -- (copy0) to [bend left] (copy1) (copy0) to [bend right] (X3);
	\draw (copy1) to [bend left] (X1) (copy1) to [bend right] (X2);
	\draw (X1) -- (F) (X2) -- (G) (X3) -- (H);
	\end{tikzpicture}\\
	&= (\kernel{L}\utimes\kernel{M})\utimes\kernel{N}
\end{align}
This follows directly from Equation \ref{eq:ccom1}.
\end{proof}

\begin{definition}[Marginal distribution, marginal kernel]\label{def:marginal_distribution}
Given a probability space $(\prob{P},\Omega,\mathcal{F})$ and the random variable $\RV{X}:\Omega\to G$ the \emph{marginal distribution} of $\RV{X}$ is the probability measure $\prob{P}^{\RV{X}}:= \prob{P}\kernel{F}^{\RV{X}}$.

See Lemma \ref{lem:pushf_funk} for the proof that this matches the usual definition of marginal distribution.

Given a Markov kernel space $(\kernel{K},\Omega,\mathcal{F},D,\mathcal{D})$ and the random variable $\RV{X}:\Omega\to G$, the \emph{marginal kernel} is $\kernel{K}^{\RV{X}|\RV{D}}:=\kernel{K}\kernel{F}^{\RV{X}}$.
\end{definition}

\begin{definition}[Joint distribution, joint kernel]\label{def:joint_distribution}
Given a probability space $(\prob{P},\Omega,\mathcal{F})$ and the random variables $\RV{X}:\Omega\to G$ and $\RV{Y}:\Omega\to H$, the \emph{joint distribution} of $\RV{X}$ and $\RV{Y}$, $\prob{P}^{\RV{X}\RV{Y}}\in \Delta(\mathcal{G}\otimes\mathcal{H})$, is the marginal distribution of $\RV{X}\utimes\RV{Y}$. That is, $\prob{P}^{\RV{X}\RV{Y}}:=\prob{P} \kernel{F}^{\RV{X}\utimes\RV{Y}}$

This is identical to the definition in \citet{cinlar_probability_2011} if we note that the random variable $(\RV{X},\RV{Y}):\omega\mapsto (\RV{X}(\omega),\RV{Y}(\omega))$ (\c{C}inlar's definition) is precisely the same thing as $\RV{X}\utimes\RV{Y}$.

Analogously, the joint kernel $\kernel{K}^{\RV{X}\RV{Y}|\RV{D}}$ is the product $\kernel{K}\kernel{F}^{\RV{X}\utimes\RV{Y}}$.
\end{definition}

Joint distributions and kernels have a nice visual representation, as a result of Lemma \ref{lem:jdist_cprod} which follows.

\begin{lemma}[Product marginalisation interchange]\label{lem:jdist_cprod}
Given two functions, the kernel associated with their coupled product is equal to the coupled product of the kernels associated with each function.

Given $\RV{X}:\Omega\to G$ and $\RV{Y}:\Omega\to H$, $\kernel{F}^{\RV{X}\utimes\RV{Y}}=\kernel{F}^\RV{X}\utimes\kernel{F}^\RV{Y}$
\end{lemma}

\begin{proof}
For $a\in \Omega$, $B\in \mathcal{G}$, $C\in \mathcal{H}$,
\begin{align}
\kernel{F}^{\RV{X}\utimes\RV{Y}} (a;B\times C) &= \delta_{\RV{X}(a),\RV{Y}(a)}(B\times C)\\
									   &= \delta_{\RV{X}(a)}(B)\delta_{\RV{Y}(a)}(C)\\
									   &= (\delta_{\RV{X}(a)}\otimes\delta_{\RV{Y}(a)})(B\times C)\\
									   &= \kernel{F}^{\RV{X}}\utimes\kernel{F}^{\RV{Y}}
\end{align}
Equality follows from the monotone class theorem.
\end{proof}

\begin{corollary}\label{corr:rewrite_joint_dist}
Given a Markov kernel space $(\kernel{K}, \Omega, D)$ and random variables $\RV{X}:\Omega\times D\to X$, $\RV{Y}:\Omega\times D\to Y$, the following holds:

\begin{align}
\begin{tikzpicture}
\path (0,0) node (O) {$D$}
++(1,0) node[kernel] (K) {$\kernel{K}^{\RV{X}\RV{Y}|\RV{D}}$}
++ (1,0.15) node (X) {$X$}
+(0,-0.3) node (Y) {$Y$};
\draw (O) -- (K);
\draw ($(K.east) + (0,0.15)$) -- (X);
\draw ($(K.east) + (0,-0.15)$) -- (Y);
\end{tikzpicture}=
\begin{tikzpicture}
\path (0,0) node (O) {$D$}
++ (0.7, 0) node[kernel] (K) {$\kernel{K}$}
++ (0.6,0) coordinate (copy0)
++ (0.4,0.25) node[kernel] (X) {$\kernel{F}^{\RV{X}}$}
+(0,-0.5) node[kernel] (Y) {$\kernel{F}^{\RV{Y}}$}
++(0.7,0) node (Xo) {$X$}
+(0,-0.5) node (Yo) {$Y$};
\draw (O) -- (K) -- (copy0);
\draw (copy0) to [bend left] (X) (X) -- (Xo);
\draw (copy0) to [bend right] (Y) (Y) -- (Yo);
\end{tikzpicture}
\end{align}
\end{corollary}

We will now define wire labels for ``output'' wires.

\begin{definition}[Wire labels - joint kernels]\label{def:wl_jprob}
Suppose we have a Markov kernel space $(\kernel{K},\Omega,D)$, random variables $\RV{X}:\Omega\times D\to X$, $\RV{Y}:\Omega\times D\to Y$ and a Markov kernel $\kernel{L}:D\to \Delta(\mathcal{X}\times\mathcal{Y})$. The following \emph{output labelling} of $\mathbf{L}$:

\begin{align}
\begin{tikzpicture}
\path (0,0) node (A) {$D$}
++ (0.7,0) node[kernel] (m) {$\kernel{L}$}
++ (0.7,0.15) node (E) {\color{blue}$\RV{X}$}
++ (0,-0.3) node (F) {\color{blue}$\RV{Y}$};
\draw (A) -- (m);
\draw ($(m.east) + (0,0.15)$) -- (E);
\draw ($(m.east) + (0,-0.15)$) -- (F);
\end{tikzpicture}
\end{align}

is \emph{valid} iff

\begin{align}
\kernel{L} = \kernel{K}_{\RV{X}\RV{Y}|\RV{D}}\label{eq:labels_express_joint}
\end{align}

and

\begin{align}
\begin{tikzpicture}
\path (0,0) node (A) {$D$}
++ (1,0) node[kernel] (m) {$\kernel{L}$}
++ (1,0.15) node (E) {\color{blue}$\RV{X}$}
++ (0,-0.3) node (F) {};
\draw (A) -- (m) ($(m.east) + (0,0.15)$) -- (E);
\draw[-{Rays [n=8]}] ($(m.east) + (0,-0.15)$) -- (F);
\end{tikzpicture} = \kernel{K}^{\RV{X}|\RV{D}}\label{eq:labels_express_marginal_upper}
\end{align}

and

\begin{align}
\begin{tikzpicture}
\path (0,0) node (A) {$D$}
++ (1,0) node[kernel] (m) {$\kernel{L}$}
++ (1,0.15) node (E) {}
++ (0,-0.3) node (F) {\color{blue}$\RV{Y}$};
\draw (A) -- (m);
\draw[-{Rays [n=8]}] ($(m.east) + (0,0.15)$) -- (E);
\draw ($(m.east) + (0,-0.15)$) -- (F);
\end{tikzpicture} = \kernel{K}^{\RV{Y}|\RV{D}}\label{eq:labels_express_marginal_lower}
\end{align}

The second and third conditions are nontrivial: suppose $\RV{X}$ takes values in some product space $Range(\RV{X}) = W\times Z$, and $\RV{Y}$ takes values in $Y$. Then we could have $\kernel{L}=\kernel{K}^{\RV{X}\RV{Y}|\RV{D}}$ and draw the diagram

\begin{align}
\begin{tikzpicture}
\path (0,0) node (A) {$D$}
++ (0.7,0) node[kernel] (m) {$\kernel{L}$}
++ (1,0.15) node (E) {$W$}
++ (0.3,-0.3) node (F) {$Z\times Y$};
\draw (A) -- (m);
\draw ($(m.east) + (0,0.15)$) -- (E);
\draw ($(m.east) + (0,-0.15)$) -- (F);
\end{tikzpicture}\label{eq:cannot_marginalise}
\end{align}

For \emph{this} diagram, properties \ref{eq:labels_express_marginal_upper} and \ref{eq:labels_express_marginal_lower} do not hold, even though \ref{eq:labels_express_joint} does.

\end{definition}

\begin{lemma}[Output label assignments exist]
Given Markov kernel space $(\kernel{K},\Omega,D)$, random variables $\RV{X}:\Omega\times D\to X$ and $\RV{Y}:\Omega\times D\to Y$ then there exists a diagram of $\kernel{L}:=\kernel{K}^{\RV{X}\RV{Y}|\RV{D}}$ with a valid output labelling assigning ${\color{blue}\RV{X}}$ and ${\color{blue}\RV{Y}}$ to the output wires.
\end{lemma}

\begin{proof}
By definition, $\kernel{L}$ has signature $D\to \Delta(\mathcal{X}\otimes\mathcal{Y})$. Thus, by the rule that tensor product spaces can be represented by parallel wires, we can draw

\begin{align}
\begin{tikzpicture}
\path (0,0) node (A) {$D$}
++ (0.7,0) node[kernel] (m) {$\kernel{L}$}
++ (0.7,0.15) node (E) {$X$}
++ (0,-0.3) node (F) {$Y$};
\draw (A) -- (m);
\draw ($(m.east) + (0,0.15)$) -- (E);
\draw ($(m.east) + (0,-0.15)$) -- (F);
\end{tikzpicture}
\end{align}

By Corollary \ref{corr:rewrite_joint_dist}, we have

\begin{align}
\begin{tikzpicture}
\path (0,0) node (A) {$D$}
++ (0.7,0) node[kernel] (m) {$\kernel{L}$}
++ (0.7,0.15) node (E) {$X$}
++ (0,-0.3) node (F) {$Y$};
\draw (A) -- (m);
\draw ($(m.east) + (0,0.15)$) -- (E);
\draw ($(m.east) + (0,-0.15)$) -- (F);
\end{tikzpicture} = \begin{tikzpicture}
\path (0,0) node (O) {$D$}
++ (0.7, 0) node[kernel] (K) {$\kernel{K}$}
++ (0.6,0) coordinate (copy0)
++ (0.4,0.25) node[kernel] (X) {$\kernel{F}^{\RV{X}}$}
+(0,-0.5) node[kernel] (Y) {$\kernel{F}^{\RV{Y}}$}
++(0.7,0) node (Xo) {$X$}
+(0,-0.5) node (Yo) {$Y$};
\draw (O) -- (K) -- (copy0);
\draw (copy0) to [bend left] (X) (X) -- (Xo);
\draw (copy0) to [bend right] (Y) (Y) -- (Yo);
\end{tikzpicture}
\end{align}

Therefore 

\begin{align}
\begin{tikzpicture}
\path (0,0) node (O) {$D$}
++ (0.7, 0) node[kernel] (K) {$\kernel{K}$}
++ (0.6,0) coordinate (copy0)
++ (0.4,0.25) node[kernel] (X) {$\kernel{F}^{\RV{X}}$}
+(0,-0.5) node[kernel] (Y) {$\kernel{F}^{\RV{Y}}$}
++(0.7,0) node (Xo) {$X$}
+(0,-0.5) node (Yo) {};
\draw (O) -- (K) -- (copy0);
\draw (copy0) to [bend left] (X) (X) -- (Xo);
\draw[-{Rays[n=8]}] (copy0) to [bend right] (Y) (Y) -- (Yo);
\end{tikzpicture} &= \kernel{K}\kernel{F}^{\RV{X}}\\
				 &= \kernel{K}^{\RV{X}|\RV{D}}
\end{align}

\begin{align}
\begin{tikzpicture}
\path (0,0) node (O) {$D$}
++ (0.7, 0) node[kernel] (K) {$\kernel{K}$}
++ (0.6,0) coordinate (copy0)
++ (0.4,0.25) node[kernel] (X) {$\kernel{F}^{\RV{X}}$}
+(0,-0.5) node[kernel] (Y) {$\kernel{F}^{\RV{Y}}$}
++(0.7,0) node (Xo) {}
+(0,-0.5) node (Yo) {$Y$};
\draw (O) -- (K) -- (copy0);
\draw[-{Rays[n=8]}] (copy0) to [bend left] (X) (X) -- (Xo);
\draw (copy0) to [bend right] (Y) (Y) -- (Yo);
\end{tikzpicture} &= \kernel{K}\kernel{F}^{\RV{Y}}\\
				 &= \kernel{K}^{\RV{Y}|\RV{D}}
\end{align}
\end{proof}

In all further work, wire labels will be used without special colouring.

\begin{definition}[Disintegration]\label{def:disintegration}
Given a probability space $(\prob{P},\Omega,\mathcal{F})$, and random variables $\RV{X}$ and $\RV{Y}$, we say that $\kernel{M}:E\to \Delta(\mathcal{F})$ is a \emph{$\RV{Y}$ on $\RV{X}$ disintegration} of $\prob{P}$ iff
\begin{align}
\begin{tikzpicture}
\path (0,0) node[dist] (m) {$\prob{P}^{\RV{X}\RV{Y}}$}
++ (1,0.15) node (E) {$\RV{X}$}
++ (0,-0.3) node (F) {$\RV{Y}$};
\draw ($(m.east) + (0,0.15)$) -- (E);
\draw ($(m.east) + (0,-0.15)$) -- (F);
\end{tikzpicture} = \begin{tikzpicture}
\path (0,0) node[dist] (m) {$\prob{P}^{\RV{X}}$}
++ (0.7,0.15) coordinate (copy0)
+(0.2,-0.3) node (T) {}
++ (1.2,0) node (E) {$\RV{X}$}
++(-0.7,-0.3) node[kernel] (K) {$\kernel{M}$}
++(0.7,0) node (F) {$\RV{Y}$};
\draw ($(m.east) + (0,0.15)$) -- (E);
\draw (copy0) to [bend right] (K) (K) -- (F);
\draw[-{Rays [n=8]}] ($(m.east) + (0,-0.15)$) -- (T);
\end{tikzpicture}\label{eq:ordinary_disint}
\end{align}
$\kernel{M}$ is a version of $\prob{P}^{\RV{Y}|\RV{X}}$, ``the probability of $\RV{Y}$ given $\RV{X}$''. Let $\prob{P}^{\{\RV{Y}|\RV{X}\}}$ be the set of all kernels that satisfy \ref{eq:ordinary_disint} and $\prob{P}^{\RV{Y}|\RV{X}}$ an arbitrary member of $\prob{P}^{\RV{Y}|\RV{X}}$.

Given a Markov kernel space $(\kernel{K},\Omega,D)$ and random variables $\RV{X}:\Omega\times D\to X$, $\RV{Y}:\Omega\times D\to Y$, $\kernel{M}:D\times E\to \Delta(\mathcal{F})$ is a \emph{$\RV{Y}$ on $\RV{DX}$ disintegration} of $\kernel{K}^{\RV{YX}|\RV{D}}$ iff

\begin{align}
\begin{tikzpicture}
\path (0,0) node (O) {}
++ (1,0) node[kernel] (m) {$\kernel{K}^{\RV{YX}|\RV{D}}$}
++ (1,0.15) node (E) {$\RV{X}$}
++ (0,-0.3) node (F) {$\RV{Y}$};
\draw (O) -- (m) ($(m.east) + (0,0.15)$) -- (E);
\draw ($(m.east) + (0,-0.15)$) -- (F);
\end{tikzpicture} = \begin{tikzpicture}
\path (0,0) node (O) {}
++ (0.3,0) coordinate (copy1)
++ (1,0) node[kernel] (m) {$\kernel{K}^{\RV{YX}|\RV{D}}$}
++ (1,0.15) coordinate (copy0)
+(0.2,-0.3) node (T) {}
++ (1.2,0) node (E) {$\RV{X}$}
++(-0.7,-0.3) node[kernel] (K) {$\kernel{M}$}
++(0.7,0) node (F) {$\RV{Y}$};
\draw (O) -- (m) ($(m.east) + (0,0.15)$) -- (E);
\draw (copy0) to [bend right] (K) (K) -- (F);
\draw (copy1) to [out=290,in=180] ($(K.west) + (0,-0.15)$);
\draw[-{Rays [n=8]}] ($(m.east) + (0,-0.15)$) -- (T);
\end{tikzpicture}\label{eq:def_k_disint}
\end{align}

Write $\kernel{K}^{\{\RV{Y}|\RV{XD}\}}$ for the set of kernels satisfying \ref{eq:def_k_disint} and $\kernel{K}^{\RV{Y}|\RV{XD}}$ for an arbitrary member of $\kernel{K}^{\{\RV{Y}|\RV{XD}\}}$.
\end{definition}

\begin{definition}[Wire labels -- input]\label{def:wl_disint}

An input wire is \emph{connected} to an output wire if it is possible to trace a path from the start of the input wire to the end of the output wire without passing through any boxes, erase maps or right facing triangles.

If an input wire is connected to an output wire and that output wire has a valid label $\RV{X}$, then it is valid to label the input wire with $\RV{X}$.

For example, if the following are valid output labels with respect to $(\prob{P},\Omega)$:

\begin{align}
\begin{tikzpicture}
\path (0,0) node (A) {}
++ (0.7,0) coordinate (copy0)
++ (0.7,0) node[kernel] (m) {$\kernel{L}$}
++ (0.7,0) node (E) {\color{blue}$\RV{X}$}
++ (0,-0.3) node (F) {\color{blue}$\RV{Y}$};
\draw (A) -- (m) -- (E);
\draw (copy0) to [out=-60,in=180] (F);
\end{tikzpicture}\label{dia:kernel_l}
\end{align}

i.e. if $\kernel{L}\in \prob{P}^{\{\RV{X}\RV{Y}|\RV{Y}\}}$, then the following is a valid input label:


\begin{align}
\begin{tikzpicture}
\path (0,0) node (A) {\color{blue}$\RV{Y}$}
++ (0.7,0) coordinate (copy0)
++ (0.7,0) node[kernel] (m) {$\kernel{L}$}
++ (0.7,0) node (E) {\color{blue}$\RV{X}$}
++ (0,-0.3) node (F) {\color{blue}$\RV{Y}$};
\draw (A) -- (m) -- (E);
\draw (copy0) to [out=-60,in=180] (F);
\end{tikzpicture}
\end{align}

An input wire in a diagram for $\kernel{M}$ may be labeled $\RV{X}$ \emph{if and only if} copy and identity maps can be inserted to yield a diagram in which the input wire labeled $\RV{X}$ is connected to an output wire with valid label $\RV{X}$.

So, if $\kernel{M}\in \prob{P}^{\{\RV{X}|\RV{Y}\}}$, then it is straightforward to show that

\begin{align}
\begin{tikzpicture}
\path (0,0) node (A) {}
++ (0.7,0) coordinate (copy0)
++ (0.7,0) node[kernel] (m) {$\kernel{M}$}
++ (0.7,0) node (E) {\color{blue}$\RV{X}$}
++ (0,-0.3) node (F) {\color{blue}$\RV{Y}$};
\draw (A) -- (m) -- (E);
\draw (copy0) to [out=-60,in=180] (F);
\end{tikzpicture} \in \prob{P}^{\{\RV{X}\RV{Y}|\RV{Y}\}} \label{eq:const_from_m}
\end{align}

and hence the output labels are valid. Diagram \ref{eq:const_from_m} is constructed by taking the product of the copy map with $\kernel{M}\otimes\textbf{Id}$. Thus it is valid to label $\kernel{M}$ with

\begin{align}
\begin{tikzpicture}
\path (0,0) node (A) {\color{blue}$\RV{Y}$}
++ (0.7,0) node[kernel] (m) {$\kernel{M}$}
++ (0.7,0) node (E) {\color{blue}$\RV{X}$};
\draw (A) -- (m) -- (E);
\end{tikzpicture}
\end{align}
\end{definition}

\begin{lemma}[labeling of disintegrations]
Given a kernel space $(\kernel{K},\Omega,D)$, random variables $\RV{X}$ and $\RV{Y}$, domain variable $\RV{D}$ and disintegration $\kernel{L}\in \kernel{K}^{\{\RV{Y}|\RV{X}\RV{D}\}}$, there is a diagram of $\kernel{L}$ with valid input labels ${\color{blue} \RV{X}}$ and ${\color{blue} \RV{D}}$ and valid output label ${\color{blue} \RV{Y}}$.
\end{lemma}

\begin{proof}
Note that for any variable $\RV{W}:\Omega\times D\to W$ and the domain variable $\RV{D}:\Omega\times D\to D$ we have by definition of $\kernel{K}$:
\begin{align}
\begin{tikzpicture}
\path (0,0) node (O) {}
++ (1,0) node[kernel] (m) {$\kernel{K}^{\RV{WD}|\RV{D}}$}
++ (1,0.15) node (E) {$\RV{W}$}
++ (0,-0.3) node (F) {$\RV{D}$};
\draw (O) -- (m) ($(m.east) + (0,0.15)$) -- (E);
\draw ($(m.east) + (0,-0.15)$) -- (F);
\end{tikzpicture} &= \begin{tikzpicture}
\path (0,0) node (O) {}
++ (0.3,0) coordinate (copy1)
++ (1,0) node[kernel] (m) {$\kernel{K}_{0}$}
++ (0.7,0) coordinate (copy0)
+ (0,-0.5) coordinate (copy2)
++ (0.7,0.3) node[kernel] (Fx) {$\kernel{F}^\RV{W}$}
++(0,-0.8) node[kernel] (Fd) {$\kernel{F}^{\RV{D}}$}
++(0.7,0) node (D) {$\RV{D}$}
++ (0,0.8) node (X) {$\RV{W}$};
\draw (O) -- (m) -- (copy0);
\draw (copy0) to [bend left] ($(Fx.west)+(0,0.1)$) (copy0) to [bend right] ($(Fd.west)+(0,0.1)$);
\draw (copy1) to [out=290,in=180] (copy2);
\draw (copy2) to [bend left] ($(Fx.west)+(0,-0.1)$) (copy2) to [bend right] ($(Fd.west)+(0,-0.1)$);
\draw (Fx) -- (X) (Fd) -- (D);
\end{tikzpicture}\\
&= \begin{tikzpicture}
\path (0,0) node (O) {}
++ (0.3,0) coordinate (copy1)
++ (1,0) node[kernel] (m) {$\kernel{K}_{0}$}
++ (0.7,0) coordinate (copy0)
+ (0,-0.5) coordinate (copy2)
++ (0.7,0.3) node[kernel] (Fx) {$\kernel{F}^\RV{W}$}
++(0,-0.8) coordinate (Fd)
++(0.7,0) node (D) {$\RV{D}$}
++ (0,0.8) node (X) {$\RV{W}$};
\draw (O) -- (m) -- (copy0);
\draw (copy0) to [bend left] ($(Fx.west)+(0,0.1)$);
\draw (copy1) to [out=290,in=180] (copy2) -- (D);
\draw (copy2) to [bend left] ($(Fx.west)+(0,-0.1)$);
\draw (Fx) -- (X) (Fd) -- (D);
\end{tikzpicture}\\
&= \begin{tikzpicture}
\path (0,0) node (O) {}
++ (0.3,0) coordinate (copy1)
+ (0.2,0) coordinate (copy3)
++ (1,0) node[kernel] (m) {$\kernel{K}_{0}$}
++ (0.7,0) coordinate (copy0)
+ (0,-0.5) coordinate (copy2)
++ (0.7,-0.1) node[kernel] (Fx) {$\kernel{F}^\RV{W}$}
++(0,-0.5) coordinate (Fd)
++(0.7,0) node (D) {$\RV{D}$}
++ (0,0.5) node (X) {$\RV{W}$};
\draw (O) -- (m);
\draw (m) to [out=0,in=180]  ($(Fx.west)+(0,0.1)$);
\draw (copy1) to [out=290,in=180] (D);
\draw (copy3) to [out=290,in=180] ($(Fx.west)+(0,-0.1)$);
\draw (Fx) -- (X);
\end{tikzpicture}\\
&= \begin{tikzpicture}
\path (0,0) node (O) {}
++ (0.3,0) coordinate (copy1)
++ (1,0) node[kernel] (m) {$\kernel{K}$}
++ (0.7,0) coordinate (copy0)
+ (0,-0.5) coordinate (copy2)
++ (0.7,-0.) node[kernel] (Fx) {$\kernel{F}^\RV{W}$}
++(0,-0.5) coordinate (Fd)
++(0.7,0) node (D) {$\RV{D}$}
++ (0,0.5) node (X) {$\RV{W}$};
\draw (O) -- (m);
\draw (m) to [out=0,in=180]  ($(Fx.west)+(0,0.0)$);
\draw (copy1) to [out=290,in=180] (D);
\draw (Fx) -- (X);
\end{tikzpicture}\\
&=\begin{tikzpicture}
\path (0,0) node (O) {}
++ (0.3,0) coordinate (copy1)
++ (1,0) node[kernel] (m) {$\kernel{K}^{\RV{W}|\RV{D}}$}
++(0,-0.5) coordinate (Fd)
++(1,0) node (D) {$\RV{D}$}
++ (0,0.5) node (X) {$\RV{W}$};
\draw (O) -- (m) -- (X);
\draw (copy1) to [out=290,in=180] (D);
\end{tikzpicture}
\end{align}



\end{proof}

The existence of disintegrations of standard measurable probability spaces is well known.

\begin{theorem}[Disintegration existence - probability space]\label{th:disintegration_exist}
Given a probability measure $\mu\in \Delta(\mathcal{X}\otimes \mathcal{Y})$, if $(F,\mathcal{F})$ is standard then a disintegration $\kernel{K}:X\to \Delta(\mathcal{Y})$ exists \citep{cinlar_probability_2011}.
\end{theorem}

In particular, if for all $x\in X$, $\prob{P}^{\RV{X}}(\RV{X}\in\{x\})>0$, then $\prob{P}^{\RV{Y}|\RV{X}}_x(\RV{Y}\in A) = \frac{\prob{P}^{\RV{X}\RV{Y}}(\RV{Y}\in A \And \RV{X}\in \{x\})}{\prob{P}^{\RV{X}}(\RV{X}\in\{x\})}$.

For Markov kernel spaces, we make the simplifying assumption that the domain space $D$ is a discrete space. Given this assumption, there exists a positive definite probability $\mu\in \Delta(\mathcal{D})$. That is, for every $d\in D$, $\mu(\{d\})>0$. Given this assumption, for every Markov kernel space $(\kernel{K},\Omega,D)$ there is a probability space $(\prob{P},\Omega\times D)$ such that $\kernel{K}$ can be uniquely defined as a disintegration of $\prob{P}$. For uncountable $D$, even if it is standard measurable, this is not possible \citep{hajek_what_2003}.


\begin{definition}[Relative probability space]

\todo[inline]{better name}

Given a Markov kernel space $(\kernel{K},\Omega,D)$ and a positive definite measure $\mu\in \Delta(\mathcal{D})$, $(\mu\kernel{K},\Omega\times D)$ is a \emph{relative} probability space.

For any random variable $\RV{X}:\Omega\times D\to X$ on $(\kernel{K},\Omega,D)$, its relative on $(\mu\kernel{K},\Omega\times D)$ is given by the same measurable function, and we give it the same name $\RV{X}$.
\end{definition}


\begin{lemma}[Agreement of disintegrations]\label{lem:agree_disint}
Given a Markov kernel space $(\kernel{K},\Omega,D)$, any relative probability space $(\mu\prob{K},\Omega\times D)$ and any random variables $\RV{X}:\Omega\times D\to X$, $\RV{Y}:\Omega\times D\to Y$, $\kernel{K}^{\{\RV{Y}|\RV{X}\RV{D}\}}=(\mu\prob{K})^{\{\RV{Y}|\RV{X}\RV{D}\}}$ (note that this set equality).
\end{lemma}

\begin{proof}
Define $\prob{P}:=\mu\kernel{K}$ and let $\kernel{M}$ be an arbitrary version of $\kernel{K}^{\{\RV{Y}|\RV{X}\RV{D}\}}$. Then
\begin{align}
\begin{tikzpicture}
\path (0,0) node[dist,inner sep=0 pt] (m) {$\prob{P}^{\RV{X}\RV{Y}\RV{D}}$}
++ (1,0.3) node (E) {$\RV{X}$}
++ (0,-0.3) node (F) {$\RV{Y}$}
++ (0,-0.3) node (D) {$\RV{D}$};
\draw ($(m.east) + (0,0.3)$) -- (E);
\draw ($(m.east) + (0,0)$) -- (F);
\draw ($(m.east) + (0,-0.3)$) -- (D);
\end{tikzpicture} &= \begin{tikzpicture}
\path (0,0) node[dist] (O) {$\mu$}
+ (0.75,0) coordinate (copy0)
++ (1.5,0) node[kernel] (m) {$\kernel{K}^{\RV{XY}|\RV{D}}$}
++ (1,0.15) node (E) {$\RV{X}$}
++ (0,-0.3) node (F) {$\RV{Y}$}
++ (0,-0.3) node (D) {$\RV{D}$};
\draw (O) -- (m) ($(m.east) + (0,0.15)$) -- (E);
\draw ($(m.east) + (0,-0.15)$) -- (F);
\draw (copy0) to [out=-60,in=180] (D);
\end{tikzpicture}\\
 &= \begin{tikzpicture}\path (0,0) node[dist] (O) {$\mu$}
++ (0.3,0) coordinate (copy1)
++ (1,0) node[kernel] (m) {$\kernel{K}^{\RV{X}|\RV{D}}$}
++ (1,0.15) coordinate (copy0)
++ (1.2,0) node (E) {$\RV{X}$}
++(-0.7,-0.3) node[kernel] (K) {$\kernel{M}$}
++(0.7,0) node (F) {$\RV{Y}$}
++(0,-0.3) node (D) {$\RV{D}$};
\draw (O) -- (m) ($(m.east) + (0,0.15)$) -- (E);
\draw (copy0) to [bend right] ($(K.west) + (0,0.1)$) (K) -- (F);
\draw (copy1) to [out=-45,in=180] ($(K.west) + (0,-0.1)$);
\draw (copy1) to [out=-90,in=180] (D);
\end{tikzpicture}\\
 &= \begin{tikzpicture}
\path (0,0) node[dist] (m) {$\prob{P}^{\RV{X}\RV{D}}$}
++ (0.7,0.15) coordinate (copy0)
+ (0,-0.3) coordinate (copy1)
+(0.2,-0.3) node (T) {}
++ (1.2,0) node (E) {$\RV{X}$}
++(-0.7,-0.3) node[kernel] (K) {$\kernel{M}$}
++(0.7,0) node (F) {$\RV{Y}$}
++ (0,-0.3) node (D) {$\RV{D}$};
\draw ($(m.east) + (0,0.15)$) -- (E);
\draw (copy0) to [bend right] ($(K.west) + (0,0.1)$) (K) -- (F);
\draw ($(m.east) + (0,-0.15)$) -- (copy1) -- ($(K.west) + (0,0)$);
\draw (copy1) to [out = -60, in=180] (D);
\end{tikzpicture}
\end{align}

Thus $\kernel{M}\in \prob{P}^{\{\RV{Y}|\RV{X}\RV{D}\}}$.

Let $\kernel{N}$ be an arbitrary version of $\prob{P}^{\{\RV{Y}|\RV{X}\RV{D}\}}$. To show that $\kernel{N}\in \kernel{K}^{\{\RV{Y}|\RV{X}\RV{D}\}}$, we will show for all $d\in D$

\begin{align}
	\prob{Q} &:= \begin{tikzpicture}
\path (0,0) node[dist] (D) {$\delta_{d}$}
++ (0.7,0) coordinate (copy0)
++(0.7,0) node[kernel] (K) {$\kernel{K}^{\RV{X}|\RV{D}}$}
++(0.5,0) coordinate (copy1)
++(0.8,0) node[kernel] (N) {$\kernel{N}$}
++(1,0) node (Y) {$\RV{Y}$}
++(0,-0.3) node (X) {$\RV{X}$}
++(0,-0.3) node (Do) {$\RV{D}$};
\draw (D) -- (K) -- (N) -- (Y);
\draw (copy0) to [out=-90,in=180] (Do);
\draw (copy1) to [out=-45,in=180] (X);
\draw (copy0) to [out=90,in=180] ($(N.west)+(0,0.15)$);
\end{tikzpicture}\\
 &= \kernel{K}^{\RV{X}\RV{Y}\RV{D}|\RV{D}}_d\label{eq:prob_disint_in_kernel_disint}
\end{align}



For $A\in\sigalg{X}$,$B\in\sigalg{Y}$, $d\in D$, we have $\prob{Q}(A\times B\times \emptyset)=0=\kernel{K}^{\RV{X}\RV{Y}\RV{D}|\RV{D}}_d(A\times B\times \emptyset$, and for $\{d\}\in\sigalg{D}$ we have $\mu(\{d\})>0$ so:

\begin{align}
\prob{Q}(A\times B\times \{d\}) &= \int_{X^2} \int_X \int_{D^3} \kernel{N}_{d'',x'}(A) \textbf{Id}_{x''}(B) \textbf{Id}_{d'''} (\{d\}) d\splitter{0.1}_d(d',d'',d''') d\kernel{K}^{\RV{X}|\RV{D}}_{d'}(x)d\splitter{0.1}_x(x',x'')\\
							&= \delta_d(\{d\}) \int_X \kernel{N}_{d,x}(A) \delta_x(B) d\kernel{K}^{\RV{X}|\RV{D}}_d(x)\\
							&= \frac{1}{\mu(\{d\})} \int_{\{d\}} d\mu(d') \int_X \kernel{N}_{d,x}(A) \delta_x(B) d\kernel{K}^{\RV{X}|\RV{D}}_d(x)\\
							&= \frac{1}{\mu(\{d\})} \int_D\int_X \kernel{N}_{d,x}(A) \delta_{d'}(\{d\}) \delta_x(B) d\kernel{K}^{\RV{X}|\RV{D}}_d(a) d\mu(d')\\
							&= \frac{1}{\mu(\{d\})} \int_D\int_X \kernel{N}_{d,x}(A) \delta_{d'}(\{d\}) \delta_x(B) d\kernel{K}^{\RV{X}|\RV{D}}_{d'}(a) d\mu(d')\\
							&= \frac{1}{\mu(\{d\})} \prob{P}^{\RV{X}\RV{Y}\RV{D}}(A\times B\times \{d\})\\
							&= \frac{1}{\mu(\{d\})} \int_D \kernel{K}_{d'}^{\RV{X}\RV{Y}\RV{D}|\RV{D}}(A\times B\times \{d\})d\mu(d')\\
							&= \frac{1}{\mu(\{d\})} \int_D \kernel{K}_{d'}{\RV{X}\RV{Y}|\RV{D}}(A\times B) \delta_{d'}(\{d\})d\mu(d')\\
							&= \kernel{K}_{d}^{\RV{X}\RV{Y}|\RV{D}}(A\times B)\\
							&= \kernel{K}_d^{\RV{X}\RV{Y}|\RV{D}}(A\times B) \delta_d(\{d\})\\
							&= \int_D \kernel{K}_{d'}^{\RV{X}\RV{Y}} (A\times B) \delta_{d''}(\{d\}) d\splitter{0.1}_d(d',d'')\\
							&= \kernel{K}_d^{\RV{X}\RV{Y}\RV{D}|\RV{D}}(A\times B\times \{d\})
\end{align}


Equality follows from the monotone class theorem. Thus $\kernel{N}\in \kernel{K}^{\{\RV{Y}|\RV{X}\RV{D}\}}$.
\end{proof}

Thus any kernel conditional probability $\kernel{K}^{\RV{Y}|\RV{X}\RV{D}}$ can equally well be considered a regular conditional probability $\prob{P}^{\RV{Y}|\RV{X}\RV{D}}$ for a related probability space $(\prob{P},\Omega\times D)$ under the obvious identification of random variables, provided $D$ is countable. Note that any conditional probability $\prob{P}^{\RV{Y}|\RV{X}}$ that is \emph{not} conditioned on $\RV{D}$ is undefined in the kernel space $(\kernel{K},\Omega,D)$.

\subsubsection{Conditional Independence}

\begin{definition}[Kernels constant in an argument]
	Given a kernel $(\kernel{K},\Omega,D)$ and random variables $\RV{Y}$ and $\RV{X}$, we say a verstion of the disintegration $\kernel{K}^{\RV{Y}|\RV{X}\RV{D}}$ is constant in $\RV{D}$ if for all $x\in X$, $d,d'\in D$, $\kernel{K}^{\RV{Y}|\RV{X}\RV{D}}_{(x,d)} = \kernel{K}^{\RV{Y}|\RV{X}\RV{D}}_{(x,d')}$.

\end{definition}

\begin{definition}[Domain Conditional Independence]
Given a kernel space $(\kernel{K},\Omega,D)$, relative probability space $(\prob{P},\Omega\times D)$, variables $\RV{X}$,$\RV{Y}$ and domain variable $\RV{D}$, $\RV{X}$ is \emph{conditionally independent} of $\RV{D}$ given $\RV{Y}$, written $\RV{X}\CI_{\kernel{K}} \RV{D}|\RV{Y}$ if any of the following equivalent conditions hold:

\begin{itemize}
	\item $\prob{P}^{\RV{X}\RV{D}|\RV{Y}} = \prob{P}^{\RV{X}|\RV{Y}}\prob{P}^{\RV{D}|\RV{Y}}$
	\item For any version of $\prob{P}^{\{\RV{X}|\RV{Y}\}}$, $\prob{P}^{\RV{X}|\RV{Y}}\otimes\stopper{0.1}_D$ is a version of  $\kernel{K}^{\{\RV{X}|\RV{Y}\RV{D}\}}$
	\item There exists a version of $\kernel{K}^{\{\RV{X}|\RV{Y}\RV{D}\}}\text{ constant in }\RV{D}$
\end{itemize}
\end{definition}

\begin{theorem}[Definitions are equivalent]
(1)$\implies$(2):
By Lemma \ref{lem:agree_disint}, $\prob{P}^{\{\RV{Y}|\RV{X}\RV{D}\}}=\kernel{K}^{\{\RV{Y}|\RV{X}\RV{D}\}}$. Thus it is sufficient to show that $\prob{P}^{\RV{X}|\RV{Y}}\otimes\stopper{0.1}$ is a version of $\prob{P}^{\{\RV{X}|\RV{Y}\RV{D}\}}$.

\begin{align}
\begin{tikzpicture}
	\path (0,0) node[dist] (Pxd) {$\prob{P}^{\RV{X}\RV{D}}$}
	+ (0.7,0.1) coordinate (copy0)
	+ (0.7,-0.1) coordinate (copy1)
	++ (1.5,0) node[kernel] (Pyxd) {$\prob{P}^{\RV{X}|\RV{Y}}$}
	++(1,0) node (Y) {$\RV{Y}$}
	+(0,0.3) node (D) {$\RV{D}$}
	+(0,0.6) node (X) {$\RV{X}$};
	\draw ($(Pxd.east) + (0,0.1)$) -- ($(Pyxd.west)+(0,0.1)$);
	\draw ($(Pxd.east) + (0,-0.1)$) -- (copy1);
	\draw[-{Rays[n=8]}] (copy1) to [out=-80,in=180] ($(Pyxd.south)+(0,-0.3)$);
	\draw (copy0) to [out=80,in=180] (X);
	\draw (copy1) to [out=80,in=180] (D);
	\draw (Pyxd) -- (Y);
\end{tikzpicture} &= \begin{tikzpicture}
	\path (0,0) node[dist] (Pxd) {$\prob{P}^{\RV{X}\RV{D}}$}
	+ (0.7,0.1) coordinate (copy0)
	+ (0.7,-0.1) coordinate (copy1)
	++ (1.5,0) node[kernel] (Pyxd) {$\prob{P}^{\RV{X}|\RV{Y}}$}
	++(1,0) node (Y) {$\RV{Y}$}
	+(0,0.3) node (D) {$\RV{D}$}
	+(0,0.6) node (X) {$\RV{X}$};
	\draw ($(Pxd.east) + (0,0.1)$) -- ($(Pyxd.west)+(0,0.1)$);
	\draw ($(Pxd.east) + (0,-0.1)$) -- (copy1);
	\draw (copy0) to [out=80,in=180] (X);
	\draw (copy1) to [out=80,in=180] (D);
	\draw (Pyxd) -- (Y);
\end{tikzpicture} \\
 &= \begin{tikzpicture}
	\path (0,0) node[dist] (Pxd) {$\prob{P}^{\RV{X}}$}
	+ (0.7,-0.2) coordinate (copy1)
	++ (1.5,-0.2) node[kernel] (Pyxd) {$\prob{P}^{\RV{X}|\RV{Y}}$}
	+ (0,0.4) node[kernel] (Pdx) {$\prob{P}^{\RV{X}|\RV{D}}$}
	++(1,0) node (Y) {$\RV{Y}$}
	+(0,0.4) node (D) {$\RV{D}$}
	+(0,0.8) node (X) {$\RV{X}$};
	\draw ($(Pxd.east) + (0,-0.2)$) -- ($(Pyxd.west)+(0,0)$);
	\draw (copy1) to [out=90,in=180] (X);
	\draw (copy1) to [out=80,in=180] (Pdx);
	\draw (Pdx) -- (D);
	\draw (Pyxd) -- (Y);
\end{tikzpicture} 
\end{align}
\end{theorem}


\begin{lemma}[Diagrammatic consequences of labels]

In general, diagram labels are ``well behaved'' with regard to the application of any of the special Markov kernels: identities \ref{eq:identity}, swaps \ref{eq:swap}, discards \ref{eq:discard} and copies \ref{eq:copy} as well as with respect to the coherence theorem of the CD category. They are not ``well behaved'' with respect to composition.

Fix some Markov kernel space $(\kernel{K},\Omega,D)$ and random variables $\RV{X}$, $\RV{Y}$, $\RV{Z}$ taking values in $X,Y,Z$ respectively. $\mathrm{Sat:}$ indicates that a labeled diagram satisfies definitions \ref{def:wl_jprob} and \ref{def:wl_disint} with respect to $(\mathscr{K},\Omega,D)$ and $\RV{X}$, $\RV{Y}$, $\RV{Z}$.  The following always holds:

\begin{align}
\mathrm{Sat:}
\begin{tikzpicture}
\path (0,0) node (A) {$\RV{X}$}
++(0.8,0) node (X) {$\RV{X}$};
\draw (A) -- (X);
\end{tikzpicture}
\end{align}

and the following implications hold:
\begin{align}
\mathrm{Sat:}\;\begin{tikzpicture}
\path (0,0) node (Z) {$\RV{Z}$} 
++ (0.7,0) node[kernel] (M) {$\kernel{K}$}
++ (0.7,0.15) node (X) {$\RV{X}$}
++(0,-0.3) node (Y) {$\RV{Y}$};
\draw (Z) -- (M) ($(M.east) + (0,0.15)$) -- (X);
\draw ($(M.east) + (0,-0.15)$) -- (Y);
\end{tikzpicture} &\implies \mathrm{Sat:}\; \begin{tikzpicture}
\path (0,0) node (Z) {$\RV{Z}$} 
++ (0.7,0) node[kernel] (M) {$\kernel{K}$}
++ (0.7,0.15) node (X) {$\RV{X}$}
++(0,-0.3) node (Y) {};
\draw (Z) -- (M) ($(M.east) + (0,0.15)$) -- (X);
\draw[-{Rays [n=8]}] ($(M.east) + (0,-0.15)$) -- (Y);
\end{tikzpicture}\\
\mathrm{Sat:}\;\begin{tikzpicture}
\path (0,0) node (Z) {$\RV{Z}$} 
++ (0.7,0) node[kernel] (M) {$\kernel{K}$}
++ (0.7,0.15) node (X) {$\RV{X}$}
++(0,-0.3) node (Y) {$\RV{Y}$};
\draw (Z) -- (M) ($(M.east) + (0,0.15)$) -- (X);
\draw ($(M.east) + (0,-0.15)$) -- (Y);
\end{tikzpicture} &\implies \mathrm{Sat:}\; \begin{tikzpicture}
\path (0,0) node (Z) {$\RV{Z}$} 
++ (0.7,0) node[kernel] (M) {$\kernel{K}$}
++ (0.7,0.15) node (X) {$\RV{Y}$}
++(0,-0.3) node (Y) {$\RV{X}$};
\draw (Z) -- (M) ($(M.east) + (0,0.15)$) to [out = 0, in = 180] (Y);
\draw ($(M.east) + (0,-0.15)$) to [out = 0, in = 180] (X);
\end{tikzpicture}\\
\mathrm{Sat:}\begin{tikzpicture}
\path (0,0) node (Z) {$\RV{Z}$} 
++ (0.7,0) node[kernel] (M) {$\mathrm{L}$}
++(0.6,0) node (X1) {$\RV{X}$};
\draw (Z) -- (M) (M)--(X1);
\end{tikzpicture}
&\implies \mathrm{Sat:}\begin{tikzpicture}
\path (0,0) node (Z) {$\RV{Z}$} 
++ (0.7,0) node[kernel] (M) {$\mathrm{L}$}
++ (0.7,0) coordinate (copy0)
++(0.5,0.2) node (X1) {$\RV{X}$}
++(0,-0.4) node (X2) {$\RV{X}$};
\draw (Z) -- (M) (M) -- (copy0) to [bend left] (X1);
\draw (copy0) to [bend right] (X2);
\end{tikzpicture}\\
\mathrm{Sat:}\begin{tikzpicture}
\path (0,0) node (X) {$\RV{Z}$}
++ (0.7,0) node[kernel] (K) {$\kernel{K}$}
++(0.7,0) node (Y) {$\RV{Y}$};
\draw (X) -- (K) -- (Y);
\end{tikzpicture} &\implies \mathrm{Sat:}
\begin{tikzpicture}
\path (0,0) node (A) {$\RV{Z}$}
++(0.5,0) coordinate (copy0)
+(1.2,0.3) node (X) {$\RV{Z}$}
++(0.5,-0.3) node[kernel] (K) {$\kernel{K}$}
+(0.7,0) node (Y) {$\RV{Y}$};
\draw (A) -- (copy0) to [bend left] (X);
\draw (copy0) to [bend right] (K) (K) -- (Y);
\end{tikzpicture}\label{eq:splitter_preserves_name}
\end{align}
\end{lemma}


\begin{proof}
\begin{itemize}
	\item $\mathrm{Id}_X$ is a version of $\prob{P}_{\RV{X}|\RV{X}}$ for all $\prob{P}$; $\prob{P}_{\RV{X}}\mathrm{Id}_X = \prob{P}_{\RV{X}}$
	\item $\kernel{K}\mathrm{Id}\otimes \stopper{0.2})(w;A) = \int_{X\times Y} \delta_x(A) \mathds{1}_Y(y) d\kernel{K}_w(x,y) = \kernel{K}_w(A\times Y) = \prob{P}_{\RV{X}|\RV{Z}}(w;A)$
	\item $\int_{X\times Y} \delta_{\mathrm{swap(x,y)}}(A\times B)d\kernel{K}_w(x,y) = \prob{P}_{\RV{Y}\RV{X}|\RV{Z}}(w;A\times B)$
	\item $\kernel{K}\splitter{0.1} (w;A\times B) = \int_{X} \delta_{x,x}(A\times B) d\kernel{K}_w(x) = \prob{P}_{\RV{X}\RV{X}|\RV{Z}} (w;A\times B)$
\end{itemize}
\ref{eq:splitter_preserves_name}: Suppose $\kernel{K}$ is a version of $\prob{P}_{\RV{Y}|\RV{Z}}$. Then
\begin{align}
\prob{P}_{\RV{Z}\RV{Y}} &= \begin{tikzpicture}
\path (0,0) node[dist] (m) {$\prob{P}_{\RV{Z}}$}
++ (0.7,0.15) coordinate (copy0)
++ (1.2,0) node (E) {$\RV{Z}$}
++(-0.7,-0.3) node[kernel] (K) {$\kernel{K}$}
++(0.7,0) node (F) {$\RV{Y}$};
\draw ($(m.east) + (0,0.15)$) -- (E);
\draw (copy0) to [bend right] (K) (K) -- (F);
\end{tikzpicture}\\
\prob{P}_{\RV{Z}\RV{Z}\RV{Y}} &= \begin{tikzpicture}
\path (0,0) node[dist] (m) {$\prob{P}_{\RV{Z}}$}
++ (0.7,0.15) coordinate (copy0)
+ (0.5,0) coordinate (copy1)
+ (1.2,0.3) node (Xm) {$\RV{Z}$}
++ (1.2,0) node (E) {$\RV{Z}$}
++(-0.7,-0.3) node[kernel] (K) {$\kernel{K}$}
++(0.7,0) node (F) {$\RV{Y}$};
\draw ($(m.east) + (0,0.15)$) -- (E);
\draw (copy0) to [bend right] (K) (K) -- (F);
\draw (copy1) to [bend left] (Xm);
\end{tikzpicture}\\
&= \begin{tikzpicture}
\path (0,0) node[dist] (m) {$\prob{P}_{\RV{Z}}$}
+ (0.5,0.15) coordinate (copy1)
++ (0.7,0.15) coordinate (copy0)
+ (1.2,0.3) node (Xm) {$\RV{Z}$}
++ (1.2,0) node (E) {$\RV{Z}$}
++(-0.7,-0.3) node[kernel] (K) {$\kernel{K}$}
++(0.7,0) node (F) {$\RV{Y}$};
\draw ($(m.east) + (0,0.15)$) -- (E);
\draw (copy0) to [bend right] (K) (K) -- (F);
\draw (copy1) to [bend left] (Xm);
\end{tikzpicture}
\end{align}
Therefore $\splitter{0.1}(\mathrm{Id}_X\otimes\kernel{K})$ is a version of $\prob{P}_{\RV{Z}\RV{Y}|\RV{Z}}$ by \ref{def:labeled_disint} 
\end{proof}

The following property, on the other hand, does \emph{not} generally hold:
\begin{align}
\mathrm{Sat:}\begin{tikzpicture}
\path (0,0) node (X) {$\RV{Z}$}
++ (0.7,0) node[kernel] (K) {$\kernel{K}$}
++(0.7,0) node (Y) {$\RV{Y}$};
\draw (X) -- (K) -- (Y);
\end{tikzpicture},
\begin{tikzpicture}
\path (0,0) node (X) {$\RV{Y}$}
++ (0.7,0) node[kernel] (K) {$\kernel{L}$}
++(0.7,0) node (Y) {$\RV{X}$};
\draw (X) -- (K) -- (Y);
\end{tikzpicture}
 &\implies \mathrm{Sat:}
\begin{tikzpicture}
\path (0,0) node (X) {$\RV{Z}$}
++ (0.7,0) node[kernel] (K) {$\kernel{K}$}
++(0.7,0) node[kernel] (L) {$\kernel{L}$}
++(0.7,0) node (X1) {$\RV{X}$};
\draw (X) -- (K) -- (Y) -- (L) -- (X1);
\end{tikzpicture}\label{eq:composition}
\end{align}

Consider some ambient measure $\prob{P}$ with $\RV{Z}=\RV{X}$ and $\prob{P}_{\RV{Y}|\RV{X}}=x\mapsto \mathrm{Bernoulli}(0.5)$ for all $z\in Z$. Then $\prob{P}_{\RV{Z}|\RV{Y}}=y\mapsto \prob{P}_{\RV{Z}}$, $\forall y\in Y$ and therefore $\prob{P}_{\RV{Y}|\RV{Z}}\prob{P}_{\RV{Z}|\RV{Y}}=x\mapsto \prob{P}_{\RV{Z}}$ but $\prob{P}_{\RV{Z}|\RV{X}} = x\mapsto \delta_x\neq \kernel{\prob{P}_{\RV{Y}|\RV{Z}}\prob{P}_{\RV{Z}|\RV{Y}}}$.





% ***************************************************************************************************


% \subsubsection{Alternative approach}

% It seems like there should be a more direct way of giving meaning to wire labels than going via a probability space as above, particularly as for our purposes it necessitates the limitation of $D$ to a countable set and the introduction of $\gamma^*$ to support embedding consequence mappings in probability spaces. 

% An alternative approach could be to begin by defining coherence rules for diagram labels similar to \ref{eq:identity_labels} -- \ref{eq:splitter_preserves_name}. We could then 

% \todo[inline]{This should probably be a category somehow, but I don't think categories of random variables have actually been worked out}


% \begin{definition}[Labelled diagram]
% Given a measurable space $E$, suppose it has some factorisation $E=A\times B$. An assignment of \emph{labels} is a map from a countable label set $L$ to an enumeration the chosen factorisation of $E$, $F:=[|\{A,B\}|]$. 

% Given a Markov kernel $\kernel{K}:E\to \Delta(\mathcal{F})$, an abstract labelled diagram $\mathfrak{D}$ is the triple $(\mathrm{Dia}(\mathfrak{D}),\mathrm{In}(\mathfrak{D}),\mathrm{Out}(\mathfrak{D})$ where $\mathrm{Dia}(\mathfrak{D})$ is a string diagram encoding $\kernel{K}$, along with ``input label assignments'' $\mathrm{In}(\mathfrak{D})$ and ``output label assignments'' $\mathrm{Out}(\mathfrak{D})$. We represent $\mathfrak{D}$ with a diagram for $\kernel{K}$ featuring $|\mathrm{In}(\mathfrak{D})|$ input wires and $|\mathrm{Out}(\mathfrak{D})|$ output wires, with each input and output wire labelled with a label from $L$. Define $\mathrm{Ker}(\mathfrak{D}):=\kernel{K}$ to be the .

% Given two diagrams $\mathfrak{D}_1$ and $\mathfrak{D}_2$ with $|\mathrm{In}(\mathfrak{D}_2)|=|\mathrm{Out}(\mathfrak{D}_1)|$ (i.e. the number of of output wires of $\mathfrak{D}_1$ is the same as the number of input wires of $\mathfrak{D}_2$) then write $\mathfrak{D}_1 \text{--} \mathfrak{D}_2$ for the diagram formed by connecting corresponding wires of $\mathfrak{D}_1$ and $\mathfrak{D}_2$ preserving the relevant labels.
% \end{definition}

% \begin{example}[Labelled diagrams]
% If we have
% \begin{align}
% \mathfrak{D}_1 &:= \begin{tikzpicture}
% \path (0,0) node (A) {$\RV{X}$}
% ++(0.6,0) node[kernel] (K) {$\RV{K}$}
% ++(0.6,0.15) node (B) {$\RV{Y}$}
% ++(0,-0.3) node (C) {$\RV{Z}$};
% \draw (A) -- (K) ($(K.east) + (0,0.15)$) -- (B) ($(K.east) + (0,-0.15)$) -- (C);
% \end{tikzpicture}\\
% \mathfrak{D}_2 &:= \begin{tikzpicture}
% \path (0,0.15) node (A) {$\RV{Y}$}
% + (0,-0.3) node (B) {$\RV{Z}$}
% ++(0.7,-0.15) node[kernel] (K) {$\RV{M}$}
% ++(0.6,0) node (C) {$\RV{W}$};
% \draw (A) -- ($(K.west)+(0,0.15)$) (B) -- ($(K.west) + (0,-0.15)$) (K) -- (C);
% \end{tikzpicture}
% \end{align}

% then 
% \begin{align}
% \mathrm{In}(\mathfrak{D}_2) = \begin{cases}
% \RV{V}\mapsto \text{``input wire 1''}\\
% \RV{W}\mapsto \text{``input wire 2''}
% \end{cases}
% \end{align}
% and $\mathfrak{D}_1 \text{--} \mathfrak{D}_2$ is the diagram
% \begin{align}
% \begin{tikzpicture}
% \path (0,0) node (A) {$\RV{X}$}
% ++(0.6,0) node[kernel] (K) {$\RV{K}$}
% ++(0.6,0.15) node (B) {}
% ++(0,-0.3) node (C) {}
% ++(0.7,0.15) node[kernel] (K1) {$\RV{M}$}
% ++(0.6,0) node (C1) {$\RV{W}$};
% \draw (A) -- (K);
% \draw ($(K.east) + (0,0.15)$) -- ($(K1.west)+(0,0.15)$) ($(K.east) + (0,-0.15)$) -- ($(K1.west) + (0,-0.15)$) (K1) -- (C1);
% \end{tikzpicture}
% \end{align}
% Note that we 
% \end{example}

% \begin{definition}[Namespace]
% A \emph{labelled kernel space} $N$ is a collection of labelled string diagrams $\{\mathfrak{A,B,C,...}\}$ such that the following rules hold:
% \begin{enumerate}
% 	\item \textbf{Composition:} $\mathfrak{A}\text{--}\mathfrak{B}\in N$ iff $\mathfrak{A}\in N$ and $\mathfrak{B}\in N$ and $\mathrm{Out}(\mathfrak{A})=\mathrm{In}(\mathfrak{B})$
% 	\item \textbf{Unison:} for $\mathfrak{A},\mathfrak{B}\in N$, if $\mathrm{In}(\mathfrak{A})=\mathrm{In}(\mathfrak{B})$ and $\mathrm{Out}(\mathfrak{A})=\mathrm{Out}(\mathfrak{B})$ then $\mathrm{Ker}(\mathfrak{A}) = \mathrm{Ker}(\mathfrak{B})$
% \end{enumerate}
% In addition, 

% 	\item \textbf{Copied inputs:} if $\mathfrak{A}\in N$ then we also have $\mathfrak{B}\in N$ such that $\mathrm{Ker}(\mathfrak{B}) = \mathrm{Id}_{\mathrm{In}(\mathfrak{A})} \utimes \mathrm{Ker}(\mathfrak{A})$

% \end{definition}

% have to deal with the case where we want to work with some Markov kernel $\kernel{M}:D\to \Delta(\mathcal{E})$ but are uncommitted as to whether this is a

% We will begin by defining wire names in the context of \emph{joint probability distributions}, which will then yield an unambiguous meaning for wire names in the context of string diagrams representing probability measaures. We then add a number of coherence rules to derive wire names for general Markov kernels. We first define analogues of \ref{def:joint_distribution} and \ref{def:disintegration} for Markov kernels. 

% \begin{definition}[Pushforward map, joint map]
% Given a kernel space $(\kernel{K},\Omega,E,\mathcal{F},\mathcal{A})$ and a random variable $\RV{X}:\Omega\to G$, the pushforward map is $\kernel{K}\kernel{F}^{\RV{X}}$.

% Given $\RV{Y}:\Omega\to H$ in addition, the joint map of $\RV{X}$ and $\RV{Y}$ is $\kernel{K}\kernel{F}^{\RV{X}\utimes\RV{Y}}$. 
% \end{definition}

% \begin{definition}[Kernel disintegration]
% Given a markov kernel $\kernel{K}:E\to\Delta(\mathcal{F})$, we say that $\kernel{L}$ is a disintegration of $\kernel{K}$ if
% \begin{align}
% \begin{tikzpicture}
% \path (0,0) node (A) {}
% ++(0.7,0)  node[kernel] (m) {$\kernel{K}$}
% ++ (0.7,0.15) node (E) {}
% ++ (0,-0.3) node (F) {};
% \draw (A) -- (m) ($(m.east) + (0,0.15)$) -- (E);
% \draw ($(m.east) + (0,-0.15)$) -- (F);
% \end{tikzpicture} = \begin{tikzpicture}
% \path (0,0) node (A) {}
% ++(0.7,0)  node[kernel] (m) {$\kernel{K}$}
% ++ (0.7,0.15) coordinate (copy0)
% ++ (1.2,0) node (E) {}
% ++(-0.7,-0.3) node[kernel] (K) {$\kernel{L}$}
% ++(0.7,0) node (F) {};
% \draw (A) -- (m) ($(m.east) + (0,0.15)$) -- (E);
% \draw (copy0) to [bend right] (K) (K) -- (F);
% \end{tikzpicture}
% \end{align}
% \end{definition}

% \begin{lemma}[Joint distributions from coupled tensor products]\label{lem:rvg_jd}
% Given a probability space $\langle E,\mathcal{E},\mu \rangle$ and a finite set of random variables $G = \{\RV{X}_i|i\in [n]\}$, the joint distribution of $G$ is given by $\mu (\utimes_{i\in [n]} \kernel{F}^{\RV{X}_i})$.
% \end{lemma}

% \begin{proof}
% This follows directly from Definition \ref{def:joint_distribution} and Lemma \ref{lem:pushf_funk}.
% \end{proof}

% When we define a joint probability distribution $\mu_{\RV{X}\RV{Y}}$ on some produce space $F\times G$, we implicitly define an association between the random variable $\RV{X}$ and the first factor of the product space $F\times G$, and similarly for $\RV{Y}$ and the second factor. Consider two random variables $\RV{X}_1:E\to X$ and $\RV{X}_2:E\to X$ and a ``unusually named'' joint distribution $\mu_{??}$ over $X\times X$. I cannot uniquely associate the elements of a tuple $(a,b)\in X\times X$ with values of $\RV{X}_1$ or $\RV{X}_2$ - to define this association, I need to either specify it separately to $\mu$ or use a standard joint probability notation such as $\mu_{\RV{X}_1\RV{X}_2}$ or $\mu(\RV{X}_1,\RV{X}_2)$.

% The first purpose of wire names is to unambiguously refer to particular wires in a given diagram. For example, suppose we have some $\mu\in \Delta(\mathcal{X}\times\mathcal{X})$, and we label wires with \emph{spaces} rather than names:

% \begin{align}
% \begin{tikzpicture}
% \path (0,0) node[dist] (M) {$\mu$}
% ++ (0.7,0.15) node (X) {$X$}
% ++ (0,-0.3) node (Y) {$X$};
% \draw ($(M.east)+(0,0.15)$) -- (X);
% \draw ($(M.east)+(0,-0.15)$) -- (Y);
% \end{tikzpicture}\label{eq:space_names}
% \end{align}

% Given just only the diagram \ref{eq:space_names}, we cannot easily refer to ``the top wire'' or ``the bottom wire''. Trying to say something like ``the probability of the bottom wire conditional on the top wire'' is very confusing, and we really do need to be able to talk about such conditional probabilities (see \ref{pgph:disint}). Giving wires unique names solves this, but (as suggested by this example), there is another desirable property of wire names: they should function as de-facto random variables, so that ``the probability of the bottom wire conditional on the top wire'' actually refers to a conditional probability defined in terms of random variables. Suppose we have a probability space $\langle E,\mathcal{E}, \mu\rangle$, and for arbitrary random variables $\RV{X}:E\to X$ and $\RV{Y}:E\to Y$ write the joint distribution $\mu_{\RV{X}\RV{Y}}$. We want wire labels to correspond to random variables in the sense that

% \begin{align}
% \mu_{\RV{X}\RV{Y}}:=\begin{tikzpicture}
% \path (0,0) node[smalldist,inner sep=-2.5pt] (M) {$\mu_{\RV{X}\RV{Y}}$}
% ++ (0.7,0.15) node (X) {$\RV{X}$}
% ++ (0,-0.3) node (Y) {$\RV{Y}$};
% \draw ($(M.east)+(0,0.15)$) -- (X);
% \draw ($(M.east)+(0,-0.15)$) -- (Y);
% \end{tikzpicture}\label{eq:wire_labels_des}
% \end{align}

% That is, the correspondence between the product space $X\times Y$ and the values taken by the random variables $\RV{X}$ and $\RV{Y}$ implicit in the definition of $\mu_{\RV{X}\RV{Y}}$ is reflected by the wire names on the right hand side of \ref{eq:wire_labels_des}. Given this definition, if we have some finite set of random variables $S=\{\RV{X},\RV{Y},...,\RV{Z}\}$ such that $\mu_{\RV{X}\RV{Y}...\RV{Z}}=\mu$, then we must be able to represent $\mu$ in a diagram with $|S|$ output wires (as the joint distribution is by definition on an appropriate product space), and we should label the wires of this diagram with $\RV{X},\RV{Y},...\RV{Z}$. Note that $\{\mathrm{Id}_E\}$ always satisfies this criterion, thus we can always draw $\mu$ with a single output wire labeled $\mathrm {Id}_E$.

% \begin{align}
% \mu = \begin{tikzpicture}
% \path (0,0) node[dist] (M) {$\mu$}
% ++ (0.7,0) node (X) {$\mathrm{Id}_E$};
% \draw (M) -- (X);
% \end{tikzpicture}\label{eq:space_names}
% \end{align}

% In general, if we have $E=X\times Y$, we can define $\RV{X}:X\times Y\to X$ and $\RV{Y}:X\times Y\to Y$ by the projection maps $\RV{X}:(x,y)\mapsto x$, $\RV{Y}:(x,y)\mapsto y$ and $\mu_{\RV{X}\RV{Y}}=\mu$ and we can write (see Lemma \ref{lem:cpm_ident}):

% \begin{align}
% \mu = \begin{tikzpicture}
% \path (0,0) node[dist] (M) {$\mu$}
% ++ (0.7,0.15) node (X) {$\RV{X}$}
% ++ (0,-0.3) node (Y) {$\RV{Y}$};
% \draw ($(M.east)+(0,0.15)$) -- (X);
% \draw ($(M.east)+(0,-0.15)$) -- (Y);
% \end{tikzpicture}\label{eq:wire_names}
% \end{align}

% If we take \ref{eq:space_names} to \emph{define} $\RV{X}$ and $\RV{Y}$, then Equation \ref{eq:wire_labels_des} compels the following labels for products involving $\mu$:

% \begin{align}
% \mu_{\RV{X}} &= \begin{tikzpicture}
% \path (0,0) node[dist] (M) {$\mu$}
% ++ (0.7,0.15) node (X) {$\RV{X}$}
% ++ (0,-0.3) node (Y) {};
% \draw ($(M.east)+(0,0.15)$) -- (X);
% \draw[-{Rays [n=8]}] ($(M.east)+(0,-0.15)$) -- (Y);
% \end{tikzpicture}\label{eq:labels_under_tensor}\\
% \mu_{\RV{Y}\RV{X}} &= \begin{tikzpicture}
% \path (0,0) node[dist] (M) {$\mu$}
% ++ (1.2,0.25) node (X) {$\RV{Y}$}
% ++ (0,-0.5) node (Y) {$\RV{X}$};
% \draw ($(M.east)+(0,0.15)$) -- (Y);
% \draw ($(M.east)+(0,-0.15)$) -- (X);
% \end{tikzpicture}\label{eq:labels_under_swap}\\
% \mu_{\RV{X}\RV{Y}\RV{X}\RV{Y}} &= \begin{tikzpicture}
% \path (0,0) node[dist] (M) {$\mu$}
% ++ (0.7,0.15) coordinate (copy0)
% + (0,-0.3) coordinate (copy1)
% ++ (0.7,0.15) node (X) {$\RV{X}$}
% ++ (0,-0.3) node (Y) {$\RV{Y}$}
% ++(0,-0.3) node (X1) {$\RV{X}$}
% ++(0,-0.3) node (Y1) {$\RV{Y}$};
% \draw ($(M.east)+(0,0.15)$) -- (copy0) to [bend left] (X);
% \draw (copy0) to [bend right] (X1);
% \draw ($(M.east)+(0,-0.15)$) -- (copy1) to [bend left] (Y);
% \draw (copy1) to [bend right] (Y1);
% \end{tikzpicture}\label{eq:copy_labels}
% \end{align}

% This illustrates the logic of the representations of the identity \ref{eq:identity}, the swap \ref{eq:swap} and the copy maps \ref{eq:copy}: their representations visually preserve the identities of wires that should be identified according to Equation \ref{eq:wire_labels_des}. Note that the presence of a copy map as in \ref{eq:copy_labels} is the \emph{only} time when a diagram will feature identical labels on wires.

% \paragraph{Free Random Variables}\label{par:frvs}

% We are interested in working with general Markov kernels, not just probability measures, and we will thus not always have an ambient probability space as in \ref{eq:wire_labels_des} to ground our wire labels.

% \begin{definition}[Free Random Variables]
% Given an ambient Markov kernel $\kernel{M}:E\to \Delta(\mathcal{F})$, a free random variable $\RV{X}$ is a measurable function on $\mathcal{E}\otimes\mathcal{F}$.
% \end{definition}

% The equivalent of marginal and joint distributions for free random variables are marginal and joint \emph{maps}.

% \begin{definition}[Marginal, joint maps]
% Given a Markov kernel $\kernel{M}:E\to \Delta(\mathcal{F})$, note that given $\gamma\in \Delta(\mathcal{E})$, by our definitions, $\gamma \prec \kernel{M}:=\gamma\splitter{0.1}(\mathrm{Id}_E\otimes\kernel{M})$ is a probability measure on $\Delta(\mathcal{E}\otimes\mathcal{F})$ and

% \begin{align}
% \gamma\prec\kernel{M} = \begin{tikzpicture}
% \path (0,0) node[dist] (E) {$\gamma$}
% ++(0.5,0) coordinate (copy0)
% + (0.5,0.3) node[kernel] (M) {$\kernel{M}$}
% +(1.2,0.3) node (X) {$\RV{E}$}
% + (0.5,-0.3) coordinate (N)
% +(1.2,-0.3) node (Y) {$\RV{F}$};
% \draw (E) -- (copy0) to [bend left] (M) (copy0) to [bend right] (N);
% \draw (M) -- (X) (N) -- (Y);
% \end{tikzpicture}
% \end{align}

% Given $\kernel{M}:E\to \Delta(\mathcal{F})$ and a free random variable $\RV{X}$, the marginal map $\kernel{M}_{\RV{X}}:E\to \Delta(\mathcal{X})$ is the unique Markov kernel such that for all $\gamma\in \Delta(\mathcal{E})$, $\gamma(\kernel{M}_{\RV{X}})=(\gamma \prec \kernel{M})_{\RV{X}}$ where the right hand side is an ordinary marginal distribution. Similarly, given free random variables $\RV{X}$, $\RV{Y}$, the joint map $\kernel{M}_{\RV{X}\RV{Y}}:E\to \Delta(\mathcal{X}\otimes\mathcal{Y})$ is the unique Markov kernel such that $\gamma\in \Delta(\mathcal{E})$, $\gamma(\kernel{M}_{\RV{X}\RV{Y}})=(\gamma \prec \kernel{M})_{\RV{X}\RV{Y}}$.
% \end{definition}

% We are now placed to impose a criterion on wire labels equivalent to \ref{eq:wire_labels_des}: given the ambient Markov kernel $\kernel{M}:E\to \Delta(\mathcal{F})$, wire labels in diagrams representing joint distributions must correspond in the sense of \ref{eq:wire_labels_des}:

% \begin{align}
% \kernel{M}_{\RV{X}\RV{Y}}:=\begin{tikzpicture}
% \path (0,0) node (E) {}
% ++(0.7,0) node[kernel] (M) {$\kernel{M}_{\RV{X}\RV{Y}}$}
% ++ (1,0.15) node (X) {$\RV{X}$}
% ++ (0,-0.3) node (Y) {$\RV{Y}$};
% \draw (E) -- (M);
% \draw ($(M.east)+(0,0.15)$) -- (X);
% \draw ($(M.east)+(0,-0.15)$) -- (Y);
% \end{tikzpicture}
% \end{align}

% In addition, we impose the requirement of identity \ref{eq:identity}, swap \ref{eq:swap} and copy maps \ref{eq:copy} - ``output'' wires that are connected to ``input'' wires with no boxes in between share names.

% Wire labels for general kernels behave similarly to those for probability measures. Suppose $E=X\times Y$ and $F=W\times Z$ and define $\RV{X},\RV{Y},\RV{W},\RV{Z}$ as projection maps from $X\times Y\times W\times Z$ to their respective spaces with $\kernel{M}$ as before. Then the following labels are compelled:

% \begin{align}
% \mathrm{Id}_E \prec \kernel{M} &= \begin{tikzpicture}[baseline={([yshift=-0.6cm]current bounding box.north)}]
% \path (0,0.15) node (W) {$\RV{W}$}
% +(0,-0.3) node (Z) {$\RV{Z}$}
% ++ (0.5,0) coordinate (copy0)
% ++ (0,-0.3) coordinate (copy1)
% ++ (0.7,0.15) node[kernel] (M) {$\kernel{M}$}
% ++ (0.7,0.15) node (X) {$\RV{X}$}
% ++ (0,-0.3) node (Y) {$\RV{Y}$}
% ++(0,-0.3) node (W1) {$\RV{W}$}
% ++(0,-0.3) node (Z1) {$\RV{Z}$};
% \draw ($(M.east)+(0,0.15)$) -- (X);
% \draw ($(M.east)+(0,-0.15)$) -- (Y);
% \draw (W) -- ($(M.west)+(0,0.15)$);
% \draw (Z) -- ($(M.west)+(0,-0.15)$);
% \draw (copy0) to [bend right] (W1) (copy1) to [bend right] (Z1);
% \end{tikzpicture}\\
% \kernel{M} &= \begin{tikzpicture}[baseline={([yshift=-0.6cm]current bounding box.north)}]
% \path (0,0.15) node (W) {$\RV{W}$}
% +(0,-0.3) node (Z) {$\RV{Z}$}
% ++ (0.7,-0.15) node[kernel] (M) {$\kernel{M}$}
% ++ (0.7,0.15) node (X) {$\RV{X}$}
% ++ (0,-0.3) node (Y) {$\RV{Y}$};
% \draw ($(M.east)+(0,0.15)$) -- (X);
% \draw ($(M.east)+(0,-0.15)$) -- (Y);
% \draw (W) -- ($(M.west)+(0,0.15)$);
% \draw (Z) -- ($(M.west)+(0,-0.15)$);
% \end{tikzpicture}\\
% \kernel{M}_{\RV{X}} &= \begin{tikzpicture}[baseline={([yshift=-0.6cm]current bounding box.north)}]
% \path (0,0.15) node (W) {$\RV{W}$}
% +(0,-0.3) node (Z) {$\RV{Z}$}
% ++ (0.5,0) coordinate (copy0)
% ++ (0,-0.3) coordinate (copy1)
% ++ (0.7,0.15) node[kernel] (M) {$\kernel{M}$}
% ++ (0.7,0.15) node (X) {$\RV{X}$}
% ++ (0,-0.3) node (Y) {}
% ++(0,-0.3) node (W1) {}
% ++(0,-0.3) node (Z1) {};
% \draw ($(M.east)+(0,0.15)$) -- (X);
% \draw[-{Rays [n=8]}] ($(M.east)+(0,-0.15)$) -- (Y);
% \draw (W) -- ($(M.west)+(0,0.15)$);
% \draw (Z) -- ($(M.west)+(0,-0.15)$);
% \draw[-{Rays [n=8]}] (copy0) to [bend right] (W1);
% \draw[-{Rays [n=8]}] (copy1) to [bend right] (Z1);
% \end{tikzpicture}\\
% &=\begin{tikzpicture}[baseline={([yshift=-0.6cm]current bounding box.north)}]
% \path (0,0.15) node (W) {$\RV{W}$}
% +(0,-0.3) node (Z) {$\RV{Z}$}
% ++(0.7,-0.15) node[kernel] (M) {$\kernel{M}$}
% ++ (1,0.15) node (X) {$\RV{X}$}
% ++ (0,-0.3) node (Y) {};
% \draw (W) -- ($(M.west)+(0,0.15)$);
% \draw (Z) -- ($(M.west)+(0,-0.15)$);
% \draw ($(M.east)+(0,0.15)$) -- (X);
% \draw[-{Rays [n=8]}] ($(M.east)+(0,-0.15)$) -- (Y);
% \end{tikzpicture}
% \end{align}

% As well as properties analogous to Equations \ref{eq:copy_labels} and \ref{eq:labels_under_swap}.

% \paragraph{Derivations of label properties}


% \begin{proof}
% For all $B\in \mathcal{F}$:
% \begin{align}
%   (\RV{X})_\# \mu \kernel{A} (B) &= \mu \kernel{A} (\RV{X}^{-1}(B))\\
%   								   &= \int_{F} \delta_{\RV{X}(a)} (B) d\mu\kernel{A} (a)\\
%   								   &= \mu\kernel{A}\kernel{F}^{\RV{X}}(B)
% \end{align}
% \end{proof}





% \begin{lemma}[Coupled projection maps are equal to the identity]\label{lem:cpm_ident}
% Suppose $E$ is a finite Cartesian product: $E=\prod_{i\in[n]} A_i$. Let $\pi_i:E\to A_i$ be the projection map $(a_1,..,a_i,..,a_n)\mapsto a_i$. Then $\utimes_{i\in [n]} \pi_i = \mathrm{Id}_E$ where $\mathrm{Id}_E$ is the identity function on $E$.
% \end{lemma}

% \begin{proof}
% Define $\pi_{[m]}:E\to \prod_{i\in [m]} A_i$ by $(a_1,...,a_{m},...,a_n)\mapsto (a_1,...,a_{m})$. Suppose $\utimes_{i\in [n-1]}\pi_i = \pi_{[n-1]}$. Then by associativity of $\utimes$, $\utimes_{i\in [n]}\pi_i = \pi_{[n-1]}\utimes \pi_{n}$ and for all $(a_1,...,a_n)\in E$, $\pi_{[n-1]}\utimes \pi_{n}(a_1,...,a_n) = (\pi_{[n-1]}(a_1,...,a_n),\pi_{n}(a_1,...,a_n))=(a_1,...,a_{n-1},a_n)=\pi_{[n]}(a_1,...,a_n)$.

% Also, $\utimes_{i\in [1]} \pi_i = \pi_1$, thus $\utimes_{i\in [n]}\pi_i = \pi_{[n]}$. But $\pi_{[n]}=\mathrm{Id}_E$.
% \end{proof}


% \begin{corollary}
% If we have a probability space $\langle E,\mathcal{E},\mu\rangle$ where $E=\prod_{i\in[n]} A_i$ and $\RV{X}_i:=\pi_i$, then $\mu_{\utimes_{i\in [n]} \RV{X}_i} = \mu$.
% \end{corollary}

% \begin{lemma}[A projection is the identity tensored with the erase map]
% Let $\pi_X:X\times Y\to X$ be the projection $\pi_X:(x,y)\mapsto x$. Then $\kernel{F}^{\pi_x} = \kernel{F}^{\mathrm{Id}_X} \otimes \stopper{0.2}$
% \end{lemma}

% \begin{proof}
% $\kernel{F}^{\pi_X}$ is, by definition, the Markov kernel $(x,y)\mapsto \delta_x$, which is equivalent to $\kernel{F}^{\mathrm{Id}_X} \otimes \stopper{0.2}$.
% \end{proof}

% \begin{corollary}
% For any $\kernel{M}:E\to \Delta(\mathcal{X}\otimes\mathcal{Y})$,

% \begin{align}
% \begin{tikzpicture}
% \path (0,0) node (E) {$\RV{E}$}
% ++ (0.7,0) node[kernel] (M) {$\kernel{M}$}
% ++ (1,0) node[kernel] (FX) {$\kernel{F}^{\pi_X}$}
% ++ (1,0) node (X) {$\RV{X}$};
% \draw (E) -- (M) (FX) -- (X);
% \draw ($(M.east)+(0,0.15)$) -- ($(FX.west)+(0,0.15)$);
% \draw ($(M.east)+(0,-0.15)$) -- ($(FX.west)+(0,-0.15)$);
% \end{tikzpicture} &=
% \begin{tikzpicture}
% \path (0,0) node (E) {$\RV{E}$}
% ++ (0.7,0) node[kernel] (M) {$\kernel{M}$}
% ++ (0.7,0.15) node (X) {$\RV{X}$}
% +(0,-0.3) node (Y) {};
% \draw (E) -- (M);
% \draw ($(M.east)+(0,0.15)$) -- (X);
% \draw[-{Rays [n=8]}] ($(M.east)+(0,-0.15)$) -- (Y);
% \end{tikzpicture}
% \end{align}
% \end{corollary}
% If all our reasoning were done using diagrams, wire identities would be clear from the diagrams. However, because we reason using a combination of diagrams, product notation, elementary notation and English words, it is necessary to be able to refer to particular wires in a diagram. We label wires with what we call \emph{meta variables}. Given a diagram representing a probability measure (rather than a general Markov kernel) that defines a number of meta variables, we can take the marginal distribution of a given meta variable or find the conditional distribution of one set of meta variables given another. In addition, we can perform analogous operations to marginalisation and finding the conditional distribution of meta variables for general Markov kernels.

% \paragraph{Meta variables and probability measures}


% Given a probability measure $\mu\in \Delta(\mathcal{X}\otimes\mathcal{Y})$, a \emph{meta variable} is a label attached to an output wire of the diagram in some valid representation. By convention, we name meta variables with sans-serif letters matching the name of the underlying space, though this is only a convention. For example, the following two diagrams define meta variables $\RV{X}$, $\RV{Y}$ and $\RV{W}$:
% \begin{align}
% &\begin{tikzpicture}
% \path (0,0) node[dist] (M) {$\mu$}
% ++ (0.7,0.15) node (X) {$\RV{X}$}
% ++ (0,-0.3) node (Y) {$\RV{Y}$};
% \draw ($(M.east)+(0,0.15)$) -- (X);
% \draw ($(M.east)+(0,-0.15)$) -- (Y);
% \end{tikzpicture} &
% \begin{tikzpicture}
% \path (0,0) node[dist] (M) {$\mu$}
% ++ (0.7,0) node (W) {$\RV{W}$};
% \draw (M) -- (W);
% \end{tikzpicture}
% \end{align}

% Define random variables $\RV{X}':X\times Y\to X$ and $\RV{Y}':X\times Y\to Y$ by the projections $\RV{X}':(x,y)\mapsto x$ and $\RV{Y}':(x,y)\mapsto y$. We can identify the meta variable $\RV{X}$ with the concrete random variable $\RV{X}'$ and likewise identify $\RV{Y}$ with the random variable $\RV{Y}'$. If we denote the marginal of $\RV{X}'$ by $\mu(\RV{X}')$ and the joint distribution of $(\RV{X}',\RV{Z}')$ by $\mu(\RV{X}',\RV{Z}')$ for any $\RV{Z}':X\times Y\to Z$, then what we mean by identifying $\RV{X}'$ with $\RV{X}$ is that

% \begin{align}
%  \mu(\RV{X}') &= \begin{tikzpicture}
% \path (0,0) node[dist] (M) {$\mu$}
% ++ (0.7,0.15) node (X) {$\RV{X}$}
% ++ (0,-0.3) node (Y) {};
% \draw ($(M.east)+(0,0.15)$) -- (X);
% \draw[-{Rays [n=8]}] ($(M.east)+(0,-0.15)$) -- (Y);
% \end{tikzpicture}\label{eq:x_marginal}\\
%  \mu(\RV{X}',\RV{Z}') &= \begin{tikzpicture}
% \path (0,0) node[dist] (M) {$\mu$}
% ++ (0.6,0.15) coordinate (copy0)
% + (1.2,0.3) node (X) {$\RV{X}$}
% +(0.5,-0.3) node[kernel] (FZ) {$\kernel{F}^{\RV{Z}'}$}
% +(1.2,-0.3) node (Z) {$\RV{Z}$};
% \draw ($(M.east)+(0,0.15)$) -- (copy0) to [bend left] (X);
% \draw ($(M.east)+(0,-0.15)$) -- (FZ);
% \draw (copy0) to [bend right] ($(FZ.west)+(0,0.15)$) (FZ) -- (Z);
% \end{tikzpicture}\label{eq:xz_joint}
% \end{align}

% Strictly, given a proability measure with labeled output wires, we can define corresponding random variables via projection maps and we can find the arbitrary marginal and joint distributions of these corresponding random variables by performing appropriate operations on the diagrams as in \ref{eq:x_marginal} and \ref{eq:xz_joint}.

% Note that in order to construct the diagrams of \ref{eq:x_marginal} and \ref{eq:xz_joint} we exploit two of the wire naming conditions described above: identity maps preserve wire names and copy maps make two copies of a wire with the same name. We can visualise these rules 

% - for details, see Example \ref{ex:wires_to_RVs}. 

% What makes $\RV{X}$ and $\RV{Y}$ ``meta'' variables is that we define rules that allow them to refer to the ``same'' wire given certain transformations of $\mu$.



% We insist that meta variables obey a number of consistency properties which ensure that wires that are intuitively the same between diagrams receive the same label.

% We take a slightly nonstandard view of random variables. Random variables are typically defined to be measurable functions on a probability space $\langle \Omega, \mathcal{F}, \mu\rangle$ \citep{cinlar_probability_2011}. With this definition, a random variable $\RV{X}:\Omega\to E$ has a canonical probability measure given by the pushforward of $\mu$: for all $A\in\mathscr{E}$ $\RV{X}_\# \mu(A) = \mu(\RV{X}^{-1}(A))$.

% We take a random variable to be a measurable function on a \emph{kernel space} $\langle F, G, \mathcal{F}\otimes\mathcal{G},\kernel{M}\rangle$ where $\kernel{M}:G\to \Delta(\mathcal{F})$ is a Markov kernel. A random variable $\RV{X}:F\to X$ has a probability distribution only relative to some argument measure $\nu\in\Delta(\mathcal{G})$. Because of this, we cannot in general unambiguously talk about ``the'' distribution of a given random variable; in general we have only conditional probabilities (which we define in Paragraph \ref{pgph:disint}). This approach mirrors to some extent the approach suggested by \citet{hajek_what_2003}, which also takes conditional probability to be fundamental.

% This choice is largely pragmatic - it is helpful to make statements about the properties of random variables on a kernel space without quantifying over prior distributions. However, there is a connection between this choice and the philosophical field of decision theory. In particular, we use random variables to model both stochastic quantities and quantities that depend on the decision maker's choices. We treat the uncertainty associated with choosing and that associated with stochasticity as different -- we do \emph{not} suppose that uncertainty over which choice will be made should itself be modeled using probability. \emph{Evidential decision theory}, as defended by \citet{jeffrey_logic_1981}, proposes that is is proper to consider choices to be random variables, though it  doing so rigorously may necessitate a theory that allows for the assignment of probabilities to the outcomes of mathematical deliberations such as the theory of \emph{logical induction} introduced in \citet{garrabrant_logical_2017}. Understanding the relationship between choices and stochastic processes is a deep, interesting and difficut question, and one we sidestep by presuming that we can address nearly all common decision problems while disregarding modelling whatever process gives rise to choices. The resulting decision theory is structurally similar to \emph{causal decision theory} \citep{lewis_causal_1981}.

% This definition of random variables permits the convention of identifying every output wire of a string diagram with a random variable. 

% \begin{example}[Wire names to random variables]\label{ex:wires_to_RVs}
% Suppose we have a Markov kernel $\kernel{A}:X\to \Delta(\mathcal{Y}\otimes\mathcal{Y})$:
% \begin{align}
% \begin{tikzpicture}
% \path (0,0) node (A) {$X$}
% ++(0.75,0) node[kernel] (B) {$\kernel{A}$}
% ++(0.75,0.15) node (C) {$\RV{Y}_1$}
% +(0,-0.3) node (D) {$\RV{Y}_2$};
% \draw (A) -- (B) ($(B.east) + (0,0.15)$) -- (C) ($(B.east)+(0,-0.15)$) -- (D);
% \end{tikzpicture}
% \end{align}

% Define $\RV{Y}'_1:Y\times Y\to Y$ by the projection map $\RV{Y}'_1:(y_1,y_2)\mapsto y_1$ and $\RV{Y}'_2:Y\times Y\to Y$ by the projection $\RV{Y}'_2:(y_1,y_2)\mapsto y_2$. Given any prior $\mu\in \Delta(\mathcal{X})$, let $(\RV{Y}'_1)_\# \mu \kernel{A}$ be the pushforward of $\RV{Y}'_1$ by $\mu\kernel{A}$. Then $\kernel{F}^{\RV{Y}'_1}:Y\times Y\to Y$ will be given by $a\mapsto \delta_{\RV{Y}'_1(a)}$.

% Define $\Pi_{\RV{Y}_1}:Y\times Y\to \Delta(\mathcal{Y})$ by $\Pi_{\RV{Y}_1}=\mathrm{Id}_Y\otimes \stopper{0.2}$. $\Pi_{\RV{Y}_1}$ is the Markov kernel that marginalises over the second argument; i.e. it marginalises over the wire named $\RV{Y}_2$. Graphically:

% \begin{align}
% \kernel{A}\Pi_{\RV{Y}_1} = 
% \begin{tikzpicture}
% \path (0,0) node (A) {$X$}
% ++(0.75,0) node[kernel] (B) {$\kernel{A}$}
% ++(1,0.15) node (C) {$\RV{Y}_1$}
% +(-0.25,-0.3) node (D) {};
% \draw (A) -- (B) ($(B.east) + (0,0.15)$) -- (C);
% \draw[-{Rays[ n=8]}] ($(B.east)+(0,-0.15)$) -- (D);
% \end{tikzpicture}
% \end{align}

% Note that for all $(y_1,y_2)\in Y\times Y$, $(\Pi_1)_{y_1,y_2} = \delta_{y_1} = \delta_{\RV{Y}'_1(y_1,y_2)}$. That is, $\Pi_{\RV{Y}_1}=F_{\RV{Y}'_1}$ and so $(\RV{Y}'_1)_\# \mu\kernel{A} = \mu\kernel{A}\Pi_{\RV{Y}_1}$. 

% Furthermore, define the joint distribution of $\RV{Y}'_1$ and $\RV{Y}'_2$ by $(\RV{Y}'_{1}\utimes\RV{Y}'_{2})_\# \mu\kernel{A}(B\times C) = \mu\kernel{A}(\RV{Y}_1^{\prime-1}(B)\cap \RV{Y}_2^{\prime-1}(C))$ for all $B,C\in \mathcal{Y}$. Then, defining $\Pi_{\RV{Y}_1\otimes\RV{Y}_2} = \mathrm{Id}_Y\otimes \mathrm{Id}_Y = \mathrm{Id}_{Y\times Y}$:

% \begin{align}
%  (\RV{Y}'_1\utimes\RV{Y}'_2)_\# \mu\kernel{A}(B\times C) &= \mu\kernel{A}(\RV{Y}_1^{\prime-1}(B)\cap \RV{Y}_2^{\prime-1}(C))\\
%  													   &= \int_{Y\times Y} \delta_{\RV{Y}'_1(y_1,y_2)}(B) \delta_{\RV{Y}'_2(y_1,y_2)}(C) d\mu\kernel{A}(y_1,y_2)\\
%  													   &= \int_{B\times C} d\mu\kernel{A}(y_1,y_2)\\
%  													   &= \mu\kernel{A}(B\times C)\\
%  													   &= \mu\kernel{A}\Pi_{\RV{Y}_1\otimes\RV{Y}_2} (B\times C) \label{eq:identity}
% \end{align}

% That is, for any prior $\mu$, the joint distribution of $\RV{Y}'_1$ and $\RV{Y}'_2$ under $\mu\kernel{A}$ is ``carried'' by the wires labeled $\RV{Y}_1$ and $\RV{Y}_2$, and the marginal distribution of $\RV{Y}_1$ is ``carried'' by the wire named $\RV{Y}_1$ alone. It's in this sense that we identify the random variable $\RV{Y}'_1$ with $\RV{Y}_1$. We will henceforth drop the distinction between a wire name and its associated random variable, letting $\RV{Y}_1$ denote both the wire and the random variable previously named $\RV{Y}'_1$.

% In general, given a Markov kernel with output space $\prod_{i\in [n]} X_i$, we can identify the $j$-th output wire with the random variable given by the projection map $\pi_j:(x_1,...x_j,...x_n)\mapsto x_j$. 
% \end{example}


% \begin{definition}[Coupled tensor product $\utimes$]
% Given two Markov kernels $\kernel{M}$ and $\kernel{N}$ or functions $f$ and $g$ with shared domain $E$, let $\kernel{M}\utimes\kernel{N}:=\splitter{0.1}(\kernel{M}\otimes\kernel{N})$ and $f\utimes g:=\splitter{0.1}(f\otimes g)$ where these expressions are interpreted using standard product notation. Graphically:

% \begin{align}
% \kernel{M}\utimes\kernel{N}&:=\begin{tikzpicture}
% \path (0,0) node (E) {$E$}
% ++(0.5,0) coordinate (copy0)
% + (0.5,0.3) node[kernel] (M) {$\kernel{M}$}
% +(1.2,0.3) node (X) {$\RV{X}$}
% + (0.5,-0.3) node[kernel] (N) {$\kernel{N}$}
% +(1.2,-0.3) node (Y) {$\RV{Y}$};
% \draw (E) -- (copy0) to [bend left] (M) (copy0) to [bend right] (N);
% \draw (M) -- (X) (N) -- (Y);
% \end{tikzpicture}\\
% f\utimes g&:= \begin{tikzpicture}[scale=1.2]\path (0,0) node (E) {$E$}
% ++(0.5,0) coordinate (copy0)
% + (0.5,0.3) node[expectation] (M) {$f$}
% + (0.5,-0.3) node[expectation] (N) {$g$};
% \draw (E) -- (copy0) to [bend left] (M) (copy0) to [bend right] (N);
% \end{tikzpicture}
% \end{align}

% \end{definition}

% \begin{proof}
% This follows from the commutativity of the swap map (Equation \ref{eq:ccom1}):

% \begin{align}
% 	(\kernel{L}\utimes\kernel{M})\utimes\kernel{N} &= \splitter{0.1}(\splitter{0.1}(\kernel{L}\otimes\kernel{M})\otimes\kernel{N})\\
% 									   &= \begin{tikzpicture}[yscale=2,xscale=1.5]
% 									   \path (0,0) node (E) {$E$}
% 									   ++(0.5,0) coordinate (copy0)
% 									   +(0.5,0.25) coordinate (copy1)
% 									   +(0.5,-0.25) coordinate (Z0)
% 									   (copy1) ++(0.5,0.15) node[kernel] (X) {$\kernel{L}$}
% 									   ++ (0,-0.3) node[kernel] (Y) {$\kernel{M}$}
% 									   (Z0) ++ (0.5,0) node[kernel] (Z) {$\kernel{N}$};
% 									   \draw (E) -- (copy0) to [bend left] (copy1) (copy0) to [bend right] (Z0) -- (Z);
% 									   \draw (copy1) to [bend left] (X) (copy1) to [bend right] (Y);
% 									   \draw (X) -- +(0.5,0) (Y) -- +(0.5,0) (Z) -- +(0.5,0);
% 									   \end{tikzpicture}\\
% 									   &= \begin{tikzpicture}[yscale=2,xscale=1.5]
%    									   \path (0,0) node (E) {$E$}
% 									   ++(0.5,0) coordinate (copy0)
% 									   +(0.5,-0.25) coordinate (copy1)
% 									   +(0.5,0.25) coordinate (X0)
% 									   (X0) ++(0.5,0) node[kernel] (X) {$\kernel{L}$}
% 									   ++ (0,-0.3) node[kernel] (Y) {$\kernel{M}$}
% 									   (copy1) ++ (0.5,-0.25) node[kernel] (Z) {$\kernel{N}$};
% 									   \draw (E) -- (copy0) to [bend right] (copy1) (copy0) to [bend left] (X0) -- (X);
% 									   \draw (copy1) to [bend right] (Z) (copy1) to [bend left] (Y);
%    									   \draw (X) -- +(0.5,0) (Y) -- +(0.5,0) (Z) -- +(0.5,0);
% 									   \end{tikzpicture}\\
% 									   &= \kernel{L}\utimes(\kernel{M}\utimes\kernel{N})
% \end{align}
% \end{proof}


% \begin{definition}[Joint distribution]\label{def:joint_distribution}
% Given $\mu\in \Delta(\mathcal{E})$ and $\RV{X}:E\to X$ and $\RV{Y}:E\to Y$, the \emph{joint distribution} of $\RV{X}$ and $\RV{Y}$ is the pushforward measure of $\mu$ by the random variable $\RV{X}\utimes\RV{Y}$ on $\mathcal{X}\otimes\mathcal{Y}$.

% This is identical to the definition in, for example, \citet{cinlar_probability_2011} if we note that the random variable $(\RV{X},\RV{Y}):\omega\mapsto (\RV{X}(\omega),\RV{Y}(\omega))$ (\c{C}inlar's definition) is equivalent to $\RV{X}\utimes\RV{Y}$.
% \end{definition}

% \begin{lemma}[Joint distributions from coupled tensor products]\label{lem:rvg_jd}
% Given a probability space $\langle E,\mathcal{E},\mu \rangle$ and a finite set of random variables $G = \{\RV{X}_i|i\in [n]\}$, the joint distribution of $G$ is given by $\mu (\utimes_{i\in [n]} \kernel{F}^{\RV{X}_i})$.
% \end{lemma}

% \begin{proof}
% This follows directly from Definition \ref{def:joint_distribution} and Lemma \ref{lem:pushf_funk}.
% \end{proof}

% %!TEX root = main.tex

\section{Notes on category theoretic probability and string diagrams}

Category theoretic treatments of probability theory often start with \emph{probability monads} (for a good overview, see \citep{jacobs_probability_2018}). A monad on some category $C$ is a functor $T:C\to C$ along with natural transformations called the unit $\eta:1_C\to T$ and multiplication $\mu:T^2\to T$. Roughly, functors are maps between categories that preserve identity and composition structure and natural transformations are "maps" between functors that also preserve composition structure. The monad unit is similar to the identity element of a monoid in that application of the identity followed by multiplication yields the identity transformation. The multiplication transformation is also (roughly speaking) associative.

An example of a probability monad is the discrete probability monad given by the functor $\mathcal{D}:\textbf{Set}\to\textbf{Set}$ which maps a countable set $X$ to the set of functions from $X\to [0,1]$ that are probability measures on $X$, denoted $\mathcal{D}(X)$. $\mathcal{D}$ maps a measurable function $f$ to $\mathcal{D}f:X\to \mathcal{D}(X)$ given by $\mathcal{D}f:x\mapsto \delta_{f(x)}$. The unit of this monad is the map $\eta_X:X\to \mathcal{D}(X)$ given by $\eta_X:x\mapsto \delta_x$ (which is equivalent to $\mathcal{D} 1_X$) and multiplication is $\mu_X:\mathcal{D}^2(X)\to \mathcal{D}(X)$ where $\mu_X:\Omega\mapsto \sum_{\phi} \Omega(\phi) \phi$.

For continuous distributions we have the Giry monad on the category $\textbf{Meas}$ of mesurable spaces given by the functor $\mathcal{G}$ which maps a measurable space $X$ to the set of probability measures on $X$, denoted $\mathcal{G}(X)$. Other elements of the monad (unit, multiplication and map between morphisms) are the ``continuous'' version of the above.

Of particular interest is the Kleisli category of the monads above. The Kleisli $C_T$ category of a monad $T$ on category $C$ is the category with the same objects and the morphisms $X\to Y$ in $C_T$ is the set of morphisms $X\to TY$ in $C$. Thus the morphisms $X\to Y$ in the Kleisli category $\textbf{Set}_{\mathcal{D}}$ are morphisms $X\to \mathcal{D}(Y)$ in $\textbf{Set}$, i.e. stochastic matrices, and in the Kleisli category $\textbf{Meas}_{\mathcal{G}}$ we have Markov kernels. Composition of arrows in the Kleisli categories correspond to Matrix products and ``kernel products'' respectively.

Both $\mathcal{D}$ and $\mathcal{G}$ are known to be \emph{commutative} monads, and the Kleisli category of a commutative monad is a symmetric monoidal category.

Diagrams for symmetric monoidal categories consist of wires with arrows, boxes and a couple of special symbols. The identity object (which we identify with the set $\{*\}$) is drawn as nothing at all  $\{*\}:=\boxed{\hspace{2em}}$ and identity maps are drawn as bare wires:

\begin{align}
	\mathrm{Id}_X:=\begin{tikzpicture}[->]
	\path (0,0) node (A) {\hspace{1em}$X$}
	++(0,0.5) coordinate (B); 
	\draw (A.center) -- (B);
	\end{tikzpicture}
\end{align}

We draw Kleisli arrows from the unit (i.e. probability distributions) $\mu:\{*\}\to X$ as triangles and Kleisli arrows $\kappa:X\to Y$ (i.e. Markov kernels $X\to \Delta(\mathcal{Y})$) as boxes. We draw the Kleisli arrow $\mathds{1}_{X}:X\to\{*\}$ (which is unique for each $X$) as below
\begin{align}
	\mu:=
	\begin{tikzpicture}[->]
		\path (0,0) node[dist] (A) {$\mu$}
		++(0,0.5) node (B) {\hspace{2em}$X$};
		\draw (A)--(B.center);
	\end{tikzpicture}
	\hspace{2cm} \kappa:=
	\begin{tikzpicture}[->]
		\path (0,0) node[kernel] (A) {$\kappa$}
		++(0,0.5) node(B) {\hspace{2em}$Y$};
		\draw (A)--(B.center);
	\end{tikzpicture}
\end{align}

The product of objects in \textbf{Meas} is given by $(X,\mathcal{X})\cdot(Y,\mathcal{Y})=(X\times Y,\mathcal{X}\otimes\mathcal{Y})$, which we will often write as just $X\times Y$. Horizontal juxtaposition of wires indicates this product, and horizontal juxtaposition also indicates the tensor product of Kleisli arrows. Let $\kappa_1:X\to W$ and $\kappa_2:Y\to Z$:
\begin{align}
 (X\times Y,\mathcal{X}\otimes\mathcal{Y}) :=
 \begin{tikzpicture}[->]
	\path (0,0) node (A) {\hspace{1em}$X$}
	+(0,0.5) coordinate (B)
	++(0.5,0) node (C) {\hspace{1em}$Y$}
	+(0,0.5) coordinate (D); 
	\draw (A.center) -- (B);
	\draw (C.center) -- (D);
 \end{tikzpicture}
 \hspace{2cm} \kappa_1\otimes \kappa_2:=
 \begin{tikzpicture}
	\path (0,0) node (A) {\hspace{1em}$X$}
	+(0,0.5) node[kernel] (B) {$\kappa_1$}
	+(0,1) node (C) {\hspace{1em}$W$}
	++(0.8,0) node (D) {\hspace{1em}$Y$}
	+(0,0.5) node[kernel] (E) {$\kappa_2$}
	+(0,1) node (F) {\hspace{1em}$Z$}; 
	\draw (A.center) -- (B);
	\draw[->] (B) -- (C.center);
	\draw (D.center) --(E);
	\draw[->] (E) -- (F.center);
 \end{tikzpicture}
\end{align}

Composition of arrows is achieved by ``wiring'' boxes together. For $\kappa_1:X\to Y$ and $\kappa_2:Y\to Z$ we have


\begin{align}
\kappa_1\kappa_2(x;A)=\int_Y \kappa_2(y;A) \kappa_1(x;dy):=\begin{tikzpicture}
	\path (0,0) node (C) {\hspace{1em}$X$}
	++(0,0.5) node[kernel] (A) {$\kappa_1$}
	++(0,0.75) node[kernel] (B) {$\kappa_2$}
	++(0,0.5) node (D) {\hspace{1em}$Z$};
	\draw (C.center) -- (A);
	\draw (A) -- (B);
	\draw[->] (B) -- (D.center);
\end{tikzpicture}
\end{align}


Symmetric monoidal categoris have the following coherence theorem\citep{selinger_survey_2010}:

\begin{theorem}[Coherence (symmetric monoidal)]
	A well-formed equation between morphisms in the language of symmetric monoidal categories follows from the axioms of symmetric monoidal categories ifand only if it holds, up to isomorphism of diagrams, in the graphical language.
\end{theorem}

Isomorphism of diagrams for symmetric monoidal categories (somewhat informally) is any planar deformation of a diagram including deformations that cause wires to cross. We consider a diagram for a symmetric monoidal category to be well formed only if all wires point upwards.


In fact the Kleisli categories of the probability monads above have (for each object) unique \emph{copy}: $X\to X\times X$ and \emph{erase}: $X\to\{*\}$ maps that satisfy the \emph{commutative comonoid axioms} that (thanks to the coherence theorem above) can be stated graphically. These differ from the copy and erase maps of \emph{finite product} or \emph{cartesian} categories in that they do not necessarily respect composition of arrows.


\begin{align}
	\text{Erase} = \mathds{1}_X := 
	\begin{tikzpicture}
	    \path (0,0) coordinate (A)
	    ++ (0,0.5) node(B) {\textbf{*}};
	    \draw (A) -- (B.center);
	\end{tikzpicture}
	\text{Copy} = x\mapsto \delta_{x,x} := 
	\begin{tikzpicture}[scale=0.8]
	\path (0,0) coordinate (A) 
	++ (0,0.5) coordinate (B)
	+ (0.5,0.5) coordinate (C)
	+ (-0.5,0.5) coordinate (D);
	\draw (A) -- (B);
	\draw[->] (B) -- (C);
	\draw[->] (B) -- (D);
	\end{tikzpicture}
\end{align}

\begin{align}
	\begin{tikzpicture}[scale=0.8]
	\path (0,0) coordinate (A) 
	++ (0,0.5) coordinate (B)
	+ (0.5,0.5) coordinate (C)
	++ (-0.5,0.5) coordinate (D)
	+(0.5,0.5) coordinate (E)
	+(-0.5,0.5) coordinate (F);
	\draw (A) -- (B);
	\draw[->] (B) -- (C);
	\draw (B) -- (D);
	\draw[->] (D) -- (E);
	\draw[->] (D) -- (F);
	\end{tikzpicture}
	=
	\begin{tikzpicture}[scale=0.8]
	\path (0,0) coordinate (A) 
	++ (0,0.5) coordinate (B)
	+ (-0.5,0.5) coordinate (C)
	++ (0.5,0.5) coordinate (D)
	+(0.5,0.5) coordinate (E)
	+(-0.5,0.5) coordinate (F);
	\draw (A) -- (B);
	\draw[->] (B) -- (C);
	\draw (B) -- (D);
	\draw[->] (D) -- (E);
	\draw[->] (D) -- (F);
	\end{tikzpicture}
	:=
		\begin{tikzpicture}[scale=0.8]
	\path (0,0) coordinate (A) 
	++ (0,0.5) coordinate (B)
	+ (-0.5,0.5) coordinate (C)
	+ (0,0.5) coordinate (D)
	+(0.5,0.5) coordinate (E);
	\draw (A) -- (B);
	\draw[->] (B) -- (C);
	\draw[->] (B) -- (D);
	\draw[->] (B) -- (E);
	\end{tikzpicture}\label{eq:ccom1}
\end{align}

\begin{align}
	\begin{tikzpicture}[scale=0.8]
	\path (0,0) coordinate (A) 
	++ (0,0.5) coordinate (B)
	+ (0.5,0.5) coordinate (C)
	+ (-0.5,0.5) node (D) {\textbf{*}};
	\draw (A) -- (B);
	\draw[->] (B) -- (C);
	\draw (B) -- (D.center);
	\end{tikzpicture}
	= 
	\begin{tikzpicture}[scale=0.8]
	\path (0,0) coordinate (A) 
	++ (0,0.5) coordinate (B)
	+ (0.5,0.5) node (C) {\textbf{*}}
	+ (-0.5,0.5) coordinate (D) ;
	\draw (A) -- (B);
	\draw (B) -- (C.center);
	\draw[->] (B) -- (D);
	\end{tikzpicture}
	=
	\begin{tikzpicture}[scale=0.8]
	\path (0,0) coordinate (A) 
	++ (0,1) coordinate (B);
	\draw[->] (A) -- (B);
	\end{tikzpicture}\label{eq:ccom2}
\end{align}

\begin{align}
\begin{tikzpicture}[scale=0.8]
	\path (0,0) coordinate (A) 
	++ (0,0.5) coordinate (B)
	+ (0.5,0.5) coordinate (C)
	+ (-0.5,0.5) coordinate (D)
	+(0.5,1) coordinate (E)
	+(-0.5,1) coordinate (F);
	\draw (A) -- (B);
	\draw (B) to [bend right] (C);
	\draw (B) to [bend left] (D);
	\draw[->] (C) to  (F);
	\draw[->] (D) to  (E);	
\end{tikzpicture}
=
\begin{tikzpicture}[scale=0.8]
	\path (0,0) coordinate (A) 
	++ (0,0.5) coordinate (B)
	+ (0.5,0.5) coordinate (C)
	+ (-0.5,0.5) coordinate (D);
	\draw (A) -- (B);
	\draw[->] (B) -- (C);
	\draw[->] (B) -- (D);
\end{tikzpicture}\label{eq:ccom3}
\end{align}

Finally, $\{*\}$ is a terminal object in the Kleisli categories of either probability monad. This means that the map $X\to\{*\}$ is unique for all objects $X$, and as a consequence for all objects $X,Y$ and all $\kappa:X\to Y$ we have
\begin{align}
\begin{tikzpicture}
 \path (0,0) node (A) {\hspace{1em}$X$}
 ++(0,0.5) node[kernel] (B) {$\kappa$}
 ++(0,0.5) node (C) {\textbf{*}};
 \draw (A.center) -- (B);
 \draw (B) -- (C.center);
\end{tikzpicture}
=
\begin{tikzpicture}
 \path (0,0) node (A) {\hspace{1em}$X$}
  ++(0,0.5) node (C) {\textbf{*}};
 \draw (A.center) -- (C.center);
\end{tikzpicture}\label{eq:termobj}
\end{align}
This is equivalent to requiring for all $x\in X$ $\int_Y \kappa(x;dy)=1$. In the case of $\textbf{Set}_\mathcal{D}$, this condition is what differentiates a stochastic matrix from a general positive matrix (which live in a larger category than $\textbf{Set}_\mathcal{D}$).

Thus when manipulating diagrams representing Markov kernels in particular (and, importantly, not more general symmetric monoidal categories) diagram isomorphism also includes applications of \ref{eq:ccom1}, \ref{eq:ccom2}, \ref{eq:ccom3} and \ref{eq:termobj}.

A particular property of the copy map in $\textbf{Meas}_{\mathcal{G}}$ (and probably $\textbf{Set}_\mathcal{D}$ as well) is that it commutes with Markov kernels iff the markov kernels are deterministic \citep{fong_causal_2013}.

\subsection{Disintegration and Bayesian inversion}

\emph{Disintegration} is a key operation on probability distributions (equivalently arrows $\{*\}\to X$) in the categories under discussion. It corresponds to ``finding the conditional probability'' (though conditional probability is usually formalised in a slightly different way).

Given a distribution $\mu:\{*\}\to X\otimes Y$, a disintegration $c:X\to Y$ is a Markov kernel that satisfies
\begin{align}
	\begin{tikzpicture}
	 	\path (0,0) node[dist] (A) {$\mu$}
	 	+ (0,0.7) node (B) {$X$ $Y$};
	 	\draw ($(A.north) +(-0.2,0)$) -- ($(B.south) + (-0.2,0)$);
	 	\draw ($(A.north) +(0.2,0)$) -- ($(B.south) + (0.2,0)$);
	\end{tikzpicture}
	=
	\begin{tikzpicture}
	\path (0,0) node[dist] (A) {$\mu$}
	+ (0.2,0.5) node (Z) {\textbf{*}}
	++ (-0.2,0.5) coordinate (B)
	+ (0.5,0.5) node[kernel] (C) {$c$}
	+ (-0.5,0.5) coordinate (D)
	+(0.5,1.2) node (E) {$Y$}
	+(-0.5,1.2) node (F) {$X$};
	\draw ($(A.north) +(-0.2,0)$) -- (B);
	\draw ($(A.north) +(0.2,0)$) -- (Z.center);
	\draw (B) -- (C);
	\draw (B) -- (D);
	\draw (D) -- (F);
	\draw (C) -- (E);
	\end{tikzpicture}
\end{align}

Disintegrations always exist in $\textbf{Set}_{\mathcal{D}}$ but not in $\textbf{Meas}_{\mathcal{G}}$. The do exist in the latter if we restrict ourselves to standard measurable spaces. If $c_1$ and $c_2$ are disintegrations $X\to Y$ of $\mu$, they are equal $\mu$-A.S. In fact, this equality can be strengthened somewhat - they are equal almost surely with respect to any distribution that shares the ``$X$-marginal'' of $\mu$.


Given $\sigma:\{*\}\to X$ and a channel $c:X\to Y$, a Bayesian inversion of $(\sigma,c)$ is a channel $d:Y\to X$ such that
\begin{align}
	\begin{tikzpicture}
	\path (0,0) node[dist] (A) {$\sigma$}
	++ (0,0.5) coordinate (B)
	+ (0.5,0.5) node[kernel] (C) {$c$}
	+ (-0.5,0.5) coordinate (D)
	+(0.5,1.2) node (E) {$Y$}
	+(-0.5,1.2) node (F) {$X$};
	\draw (A) -- (B);
	\draw (B) -- (C);
	\draw (B) -- (D);
	\draw (D) -- (F);
	\draw (C) -- (E);
	\end{tikzpicture}
	=
	\begin{tikzpicture}
	\path (0,0) node[dist] (A) {$\sigma$}
	++ (0,0.5) node[kernel] (Z) {$c$}
	++ (0,0.5) coordinate (B)
	+ (-0.5,0.5) node[kernel] (C) {$d$}
	+ (0.5,0.5) coordinate (D)
	+(0.5,1.2) node (E) {$Y$}
	+(-0.5,1.2) node (F) {$X$};
	\draw (A) -- (Z);
	\draw (Z) -- (B);
	\draw (B) -- (C);
	\draw (B) -- (D);
	\draw (D) -- (E);
	\draw (C) -- (F);
	\end{tikzpicture}
\end{align}

We can obtain disintegrations from Bayesian inversions and vise-versa.

\citet{clerc_pointless_2017} offer an alternative view of Bayesian inversion which they claim doesn't depend on standard measurability conditions, but there is a step in their proof I didn't follow.

\subsection{Generalisations}

\citet{cho_disintegration_2019} make use of a larger ``CD'' category by dropping \ref{eq:termobj}. I'm not completely clear whether you end up with arrows being ``Markov kernels for general measures'' or something else (can we have negative arrows?). This allows for the introduction of ``observables'' or ``effects'' of the form 
\begin{tikzpicture}
 \path (0,0) coordinate (A)
 ++ (0,0.5) node[expectation] (B) {$f$};
 \draw (A) -- (B);
\end{tikzpicture}.

\citet{jacobs_causal_2019} make use of an embedding of $\textbf{Set}_\mathcal{D}$ in $\textbf{Mat}(\mathbb{R}^+)$ with morphisms all positive matrices (I'm not totally clear on the objects, or how they are self-dual - this doesn't seem to be exactly the same as the category of finite dimensional vector spaces). This latter category is compact closed, which - informally speaking - supports the same diagrams as symmetric monoidal categories with the addition of ``upside down'' wires.

\subsection{Key questions for Causal Theories}


\paragraph{generalised disintegrations}: Of key importance to our work is generalising the notion of disintegration (and possibly Bayesian inversion) to general kernels $X\to Y$ rather than restricting ourselves to probability distributions $\{*\}\to Y$.

Given $\kappa:D\to X\times Y$, a kernel $c:D\times X\to Y$ is a \emph{generalised disintegration} (``g-disintegration'') of $\kappa$  from $D\times X$ to $Y$ if the following holds:

\begin{align}
	\begin{tikzpicture}
	 	\path (0,0) coordinate (Z)
	 	++(0,0.5) node[kernel] (A) {\hspace{1em}$\kappa$}
	 	+ (0,0.7) node (B) {$X$ $Y$};
	 	\draw (Z) -- (A);
	 	\draw ($(A.north) +(-0.2,0)$) -- ($(B.south) + (-0.2,0)$);
	 	\draw ($(A.north) +(0.2,0)$) -- ($(B.south) + (0.2,0)$);
	\end{tikzpicture}
	=
	\begin{tikzpicture}
	\path (0,0) coordinate (X)
	++(0,0.1) coordinate (W)
	++ (0,0.5) node[kernel] (A) {\hspace{1em}$\kappa$}
	+ (0.2,0.5) node (Z) {\textbf{*}}
	++ (-0.2,0.5) coordinate (B)
	+ (0.9,0.5) node[kernel] (C) {$c$}
	+ (-0.5,0.5) coordinate (D)
	+(0.9,1.2) node (E) {$Y$}
	+(-0.5,1.2) node (F) {$X$};
	\draw (X) -- (W);
	\draw (W) to [bend right] (C);
	\draw (W) -- (A);
	\draw ($(A.north) +(-0.2,0)$) -- (B);
	\draw ($(A.north) +(0.2,0)$) -- (Z.center);
	\draw (B) -- (C);
	\draw (B) -- (D);
	\draw (D) -- (F);
	\draw (C) -- (E);
	\end{tikzpicture}
\end{align}

If we introduce let $Z=\{*\}$ and draw attention to the isomorphism $\{*\}\times X\times Y\cong X\times Y$, we can see that $\kappa:D\to X\times Y$ itself is a g-distintegration $\kappa$ from $D$ to $X\times Y$.

\begin{theorem}
For all $\kappa:D\to X\times Y$, if $D$ is countable and $X\times Y$ is standard measurable, a g-disintegration of $\kappa$ exists in the following three directions: from $D\to X\times Y$, from $D\times X\to Y$ and from $D\times Y\to X$.
\end{theorem}

\begin{proof}
For the first direction, we note that $\kappa$ satisfies the condition.

For all $y\in D$ we have a disintegration $c_y:X\to Y$ of $\delta_y \kappa$ by standard measurability of $X\times Y$. Define $c:D\times X\to Y$ by $c:(y,x)\mapsto c_y(x)$. Clearly, $c(y,x)$ is a probability distribution on $Y$ for all $(y,x)\in D\times X$. It remains to show $c(\cdot)^{-1}(B)$ is measurable for all $B\in \mathcal{B}([0,1])$. But $c(\cdot)^{-1}(B) = \cap_{y\in D} c_y(\cdot)^{-1}(B)$. The right hand side is measurable by measurability of $c_y(\cdot)^{-1}(B)$ and the properties of a $\sigma$-algebra.

The final direction follows from symmetry.
\end{proof}

\paragraph{Conjecture:} This can be generalised to any $\kappa$ that is determined by its values on a countable set of points along with some notion of continuity. This seems likely to be true. In a more general setting, I think I could find a counterexample, but the converse also seems unlikely.

The extension of \emph{conditional independence} to g-disintegrations becomes a directional relationship. Suppose we have $\kappa:D\to X\times Y$ and a disintegration $c:D\times X\to Y$. We say $Y$ is directionally conditionally independent (DCI) of $D$ given $X$ if
\begin{align}
 \begin{tikzpicture}
 
 \end{tikzpicture}
\end{align}

Generalised disintegrations facilitate the following construction of a ``graphical model'':

Suppose we have two causal theories, $\mathscr{T}^*$ and $\mathscr{T}$ both with signature $E\times D\rightarrowtriangle E$, and $\mathscr{T}$ is a decision randomised version of $\mathscr{T}^*$ (i.e. $\mathscr{T}=\{(\lambda\kappa,\mu)|(\kappa,\mu)\in\mathscr{T}^*\}$ for some $\lambda:D\to D$. We will construct a graphical model from $\mathscr{T}^*$ and $\mathscr{T}$ in three steps:

First, we assume \emph{reproducibility} in the stronger theory $\mathscr{T}^*$. That is, for all $(\kappa,\mu)\in \mathscr{T}^*$ we suppose there exists $\gamma\in \Delta(\mathcal{D})$ such that $\gamma\kappa=\mu$. 

\todo{I don't think reproducibility is quite the right assumption, but it is good enough for now}

Second, we will assume certain \emph{generalised conditional independences} hold for the stronger theory $\mathscr{T}^*$ (we have not defined these, but they are the obvious generalisation of standard conditional independence lifted to g-disintegrations). Because we're constructing a graphical model, we will assume these are a ``DAG-compatible'' set, though we are under no obligation to do so. I conjecture we can illustrate these independences graphically. Suppose we have random variables $\RV{X}:E\to X$, $\RV{Y}:E\to Y$ and $\RV{Z}:E\to Z$, and we assume we have at least the generalised CIs implied by the following diagram for all $(\kappa,\mu)\in \mathscr{T}^*$:
\begin{align}
\kappa=\begin{tikzpicture}
\path (0,0) node (A) {$E$}
	++(0,0.2) coordinate (B)
	++ (0,0.3) node[kernel] (C) {\hspace{1em}$\kappa$}
	+ (0,0.5) node (D) {\textbf{*}}
	+ (0.2,0.5) node (E) {\textbf{*}}
	++ (-0.2,0.5) coordinate (F)
	+ (-0.5,0.5) node (G) {$X$}
	++ (0.5,0.5) node[kernel] (H) {$c_{Y|X}$}
	+ (-0.5,0.8) node (I) {$Y$}
	++(0.5,0.8) node[kernel] (J) {$c_{Z|Y}$}
	++(0,0.8) node (K) {$Z$};
	\draw (A) -- (B);
	\draw (B) -- (C);
	\draw (C) -- (D.center);
	\draw (C) -- (E.center);
	\draw (C) -- (F);
	\draw (F) -- (G);
	\draw (F) -- (H);
	\draw (H) -- (I);
	\draw (H) -- (J);
	\draw (J) -- (K);
\end{tikzpicture}
\end{align}


The above diagram is typed incorrectly, but we can always construct a kernel $\kappa_{\RV{XYZ}}$ that maps to $X\times Y\times Z$.
% \section{Causal Decision Problems}

We will work with a slightly restricted version of a statistical decision problem where the sequence of observations is fixed in advance.

\begin{definition}[Statistical Decision Problem]
A statistical decision problem (SDP) is a tuple $\langle (\Omega,\mathcal{F},\mu), (D,\mathcal{D}), \RV{X}, \mathscr{H}, \ell\rangle$. 

$(\Omega,\mathcal{F},\mu)$ is a probability space given by the environment where the distribution $\mu$ is unknown, $(D,\mathcal{D})$ is a measurable space of decisions, $\RV{X}:\Omega\to X$ is a random variable representing a fixed sequence of observations, $\mathscr{H}\subseteq\Delta(\mathcal{F})$ is a hypothesis class known to contain $\mu$ and $\ell:\mathscr{H}\times D\to [0,\infty)$ is a loss.

The aim of a statistical decision problem is to find a decision kernel $J:X\to \Delta(\mathcal{D})$ minimising the risk:
\begin{align}
    R(J,\mu) = \int_{\Omega}\int_D \ell(\mu,y) \mu J(dy)
\end{align}
\end{definition}

Here $\mu J$ is the measure-kernel product defined in \ref{def:kernel_product}. As $\mu$ is generally unknown, it is not possible to do this directly. Given a prior $\pi$ on 


A causal decision problem extends a statistical decision problem in that, while an SDP provides us with a loss $D\times \Delta(\mathcal{F})\to [0,\infty)$ that evaluates (decision, state) pairs, a CDP gives us a loss from $\Delta(\mathcal{F})\to [0,\infty)$ that evaluates only consequences. Evaluating the quality of a decsion in an CDP therefore requires an object of the type $D\to \Delta(\mathcal{F})$, which we call a \emph{consequence kernel}.

\begin{definition}[Consequence kernel]
Given a measurable space $(\Omega,\mathcal{F})$ and a measurable decision set $(D,\mathcal{D})$, a consequence kernel is a Markov kernel $\kappa:D \to \Delta(\mathcal{F})$.
\end{definition}

\begin{definition}[Causal Decision Problem]
A causal decision problem (CDP) is a tuple $\langle (\Omega,\mathcal{F},\kappa^*), (D,\mathcal{D}), L, \mathbf{K} \rangle$. The triple $(\Omega,\mathcal{F},\kappa^*)$ is given by the environment, where $\Omega$ is the sample space and $\mathcal{F}$ is a $\sigma$-algebra on $\Omega$. We posit that a true consequence kernel $\kappa^*$ exists which is only known to belong to the class $\mathbf{K}\subset \Delta(\mathcal{F})^D$.

The measurable set $D$ represents the decisions available, and $L:\Delta(\mathcal{F})\to [0,\infty)$ is the loss.

The objective of a causal decision problem is to choose a stochastic decision $\theta\in \Delta(\mathcal{D})$ such that $L(\theta \kappa^*)$ is minimal.
\end{definition}

In a causal decision problem, the class $\mathbf{K}$ represents the given background knowledge about how the world works. Without an informative class $\mathbf{K}$ it is clearly impossible to determine if any decision is preferred to any other decision.

A statistical causal decision problem is a causal decision problem where we are provided some data which we may use to make a decision. A generalised statistical causal decision problem features a causal prospect $\mathscr{T}$ that plays a similar role to to the class of possible consequences $\mathbf{K}$ for a regular causal decision problem.

\begin{definition}[Generalised Causal Theory]\label{def:gen_causal_theory}
Given measurable space $(\Omega,\mathcal{F})$ and $(X,\mathcal{X})$, a random variable $\RV{X}:\Omega\to X$ and a decision set $D$, a generalised causal theory is a Markov kernel $\tau:X\times D \to \Delta(\mathcal{F})$. The map $x\mapsto \tau(\cdot;x,\cdot)$ maps the set $X$ to a consequence kernel.
\end{definition}

\begin{definition}[Generalised Causal Prospect]\label{def:gen_causal_prospect}
A generalised causal prospect is a set of generalised causal theories.
\end{definition}

\begin{definition}[Generalised Statistical Causal Decision Problem]\label{def:gen_scdp}
A generalised statistical causal decision problem (GSCDP) is a tuple $\langle (\Omega,\mathcal{F},\kappa^*,\mu), (D,\mathcal{D}), L, \RV{X}, \mathscr{T} \rangle$.

The elements $(\Omega, \mathcal{F},\kappa^*), (D,\mathcal{D}), L$ are in common with a causal decision problem. $\RV{X}:\Omega\to X$ is a random variable distributed according to $\RV{X}_*\mu$ where $\mu\in \mathscr{H}\subset \Delta(\mathcal{F})$. $\mathscr{T}$ is a generalised causal prospect containing theories $X\times D\to \Delta(\mathcal{F})$.

The objective of a GSCDP is to find a decision kernel $\phi:X\to \Delta(D)$ such that $L(\mu\phi\kappa^*)$ is minimal.

If a GSCDP is realisable, given arbitray $A\in \mathcal{X}$ such that $\mu(A)>0$, the true kernel $\kappa^*$ is assumed to belong to the set $\{\frac{1}{\mu(A)}\int_A \tau(\cdot;x,\cdot)\mu(dx)|\tau\in\mathscr{T}\}$. That is, we assume the causal prospect is sufficiently large to contain a theory mapping to the true kernel with probability 1.
\end{definition}

A GSCDP can be reduced to a regular CDP. The intuition is that a decision kernel for a statistical causal decision problem can be considered to be an ordinary decision for a regular causal decision problem. If the set of available decision kernels is convex and closed, each stochastic decision in the regular causal decision problem corresponds to a decision kernel in the GSCDP, so each problem presents exactly the same set of options.

\begin{theorem}[Reduction from GSCDP to CDP]\label{th:gscdp_to_cdp}
Given a GSCDP $Q=\langle (\Omega,\mathcal{F},\kappa^*,\mu), (D,\mathcal{D}), L, \RV{X}, \mathscr{T} \rangle$, suppose we further have a set of available decision functions $\Phi\subset \Delta(\mathcal{F})^D$ along with some $\sigma$-algebra $\mathcal{E}$ and a hypothesis class $\mathscr{H}\ni \mu$. Then $Q$ can be reduced to a regular causal decision problem $\langle (\Omega,\mathcal{F},\kappa^{*\prime}), (D',\mathcal{D}'), L, \mathbf{K}' \rangle$ with the identification $(D',\mathcal{D}') = (\Phi,\mathcal{E})$, $\kappa^{*\prime}:\phi\mapsto \mu\phi \kappa^*$, $\mathbf{K}' = \{\phi\mapsto \nu\phi\frac{1}{\nu(A)}\int_A\tau(\cdot;x,\cdot)\nu(dx)|\tau\in\mathscr{T},\nu\in \mathscr{H},\forall A:\nu(A)>0\}$.

If $\Phi$ is convex and closed, then each $\theta\in \Delta(\mathcal{E})$ can be identified with some $\phi\in \Phi$ such that $\theta\kappa^{*\prime} = \mu\phi\kappa^*$.
\end{theorem}

\begin{proof}
For the reduction, we must show that $\kappa^*\in \{\frac{1}{\nu(A)}\int_A \tau(\cdot;x,\cdot)\nu(dx)|\tau\in \mathscr{T},\nu\in\mathscr{H},\forall A:\nu(A)>0\}$ implies $\kappa^{*\prime}\in \mathbf{K}'$. This is trivial by the definition of $\mathscr{H}$, $\mathbf{K}'$ and $\kappa^{*\prime}$.

For the identification, we require for each $\theta\in \Delta(\mathcal{E})$ there is a corresponding $\phi\in \Phi$ such that $L(\mu\phi\kappa^*)=L(\theta\kappa^{*\prime})$. Observe that given any $\theta\in \Delta(\mathcal{E})$ by convexity and closure there exists $\phi_0\in \Phi$ such that $\phi_0 = \int_\Phi \phi \theta(d\phi)$. Therefore $\theta\kappa^{*\prime}=\int_\Phi \mu \phi \kappa^* \theta(d\phi) = \mu \int_\Phi \phi \theta(d\phi) \kappa^*=\mu\phi_0\kappa^*$.
\end{proof}
% \section{Causal Bayesian Networks}

A causal Bayesian network can be understood to be a causal theory. The CBN convention is to call the elements of the decision space $D$ ``interventions'' and denote then with $do()$ notation. Given a random variable $\RV{X}^i$ on the output space $(F,\mathcal{F})\to (X^i,\mathcal{X}^i)$, we identify the intervention $do(\RV{X}^i=x)$ with an element $x\in X^i$ and the absence of any do intervention with the special ``passive'' element $*$.

\begin{definition}[Causal Bayesian Network]\label{def:CBN}
The definition here follows \cite{pearl_causality:_2009}.

Consider a directed acyclic graph $\mathcal{G}$ with edges $\mathbf{V}=\{\RV{X}^i|i\in [N]\}$, a measurable space $(F,\mathcal{F})$ and a set of random variables $\RV{X}^i:F\to X^i$ and let $X=\cup_{i\in[N]} X^i$. For each $x\in X\cup\{*\}$ suppose we have an \emph{interventional distribution} $P^{x}_\RV{X}$, and let the set of all such distributions be denoted $\mathbf{P}^{X\cup\{*\}}$. Let $P^*_{\RV{X}}$ be the passive distribution given by the intervention $x = (*,...,*)$.

Given any $x\in X\cup\{*\}$ let $S\subset[N]$ be the set of all indices $i$ such that $x^i\neq *$. The graph $\mathcal{G}$ is a causal Bayesian network compatible with $\mathbf{P}^{X\cup\{*\}}$ iff for all $x\in X$ and $S\subset [N]$:
\begin{enumerate}
    \item $P^{x}_{\RV{X}}$ is compatible with $\mathcal{G}$ for all $x\in X\cup\{*\}$
    \item $P^x_{\RV{X}}(\RV{X}^S)=\delta_{x^S}(\RV{X}^S)$
    \item For $i\in S^C$, $P^x_{\RV{X}}(\RV{X}^i|\PA{\mathcal{G}}{\RV{X}^i})=P^*_\RV{X}(\RV{X}^i|\PA{\mathcal{G}}{\RV{X}^i})$, $P^x$-almost surely
\end{enumerate}
\end{definition}

Taking $(F,\mathcal{F})=(X^{\mathbb{N}},\mathcal{X}^{\mathbb{N}})$ and considering only distributions where the sequence $\RV{X}_0,\RV{X}_1,..$ is IID, the above three conditions are sufficient that, given some graph $\mathcal{G}$ and $P^*_{\RV{X}}\in \Delta(\mathcal{X})$, one obtains a unique set of interventional distributions $\mathbf{P}^{X\cup\{*\}}$ (this follows from the truncated factorisation property given by \cite{pearl_causality:_2009}). 

Taking the map $\kappa: x\mapsto P^x$ as a consequence, a graph $\mathcal{G}$ therefore defines the causal theory $P^*\mapsto \kappa$. Theorem \ref{th:cbn_MK} establishes that the object $\kappa$ is a Markov kernel if $\mathcal{G}$ has a finite number of nodes, so it is strictly correct to consider this map a causal theory.

\begin{theorem}\label{th:cbn_MK}
Given a graph $\mathcal{G}$ and a set $\mathscr{H}\subset\Delta{\mathcal{X}}$ of probability distributions compatible with $\mathcal{G}$, the map $\tau_{\mathcal{G}}:\mathscr{H}\to (D\to \mathscr{H})$ given by $\mu\mapsto (x\mapsto P^x)$ for $\mu\in \Delta(X)$ is a causal theory.
\end{theorem}

The proof is given in Appendix \ref{app:cbn_ct}.

\subsection{Supergraph equivalence}

A CBN defines a causal theory, so naturally a set of CBNs defines a causal prospect. 

It is well known that in Definition \ref{def:CBN}, condition 1 is vacuous if $\mathcal{G}$ is a fully connected graph. For a fully connected graph, therefore, conditions 2 and 3 are sufficient to define the causal theory associated with $\mathcal{G}$. 



Theorem \ref{th:sup_equiv} establishes that in fact, a set of fully connected graphs and hence conditions 2 and 3 alone are enough to specify any causal prospect (that is, any set of causal theories) that can be specified by a set of causal Bayesian networks.

This has consequences for the practice of learning causal graphs. In particular, any algorithm that learns to delete edges from a set of graphs 

\begin{theorem}[Supergraph equivalence]\label{th:sup_equiv}

\end{theorem}

\section{Single World Intervetion Graphs}

\section{Potential Outcomes}

The definition of 
% 
\section{Potential Outcomes}

Potential Outcomes is an alternative to the approach typified by Causal Bayesian Networks for formulating causal questions and hypotheses. Causal queries in the Potential Outcomes framework concern the distribution of random variables $\RV{X}_0, \RV{X}_1$ representing potential outcomes, or ``the value $\RV{X}$ would have taken if action 0 or 1 were taken respectively'' (\cite{hernan_causal_2018}). This is similar, but not the same, as the question answered by a consequence map which is ``what is the distribution of $\RV{X}$ if I take actions 0 or 1?''

A natural connection between these informal notions of potential outcomes and consequence maps is given by the notion of consequence consistency. Let $\Delta(\mathcal{Y}_\circ)$ be the space of joint distributions over real and potential outcomes of $\RV{X}$. A consequence map $\kappa:D\to \Delta(\mathcal{Y}_\circ)$ is consequence consistent if
\begin{align}
    (\delta_i\kappa)_{|\RV{X}_i} F_{\RV{X}} (w;A)=\delta_{\RV{X}_i(w)}(A) \label{eq:oc_consist}
\end{align} 
Consequence consistency is similar to the consistency condition \citep{richardson2013single}, but the latter does not involve consequences.

A causal theory that is consequence consistent need not have any particular relationship between an ``observed'' distribution $\mu\in \Delta(\mathcal{Y}_\circ)$ and an associated consequence $\kappa$; one choice to make this connection is equality of the distributions of potential outcomes $\mu F_{\RV{X}_i} = \delta_i \kappa F_{\RV{X}_i}$, $i\in D$. Example \ref{ex:nonst_distn} in Appendix \ref{app:counfac} shows that other choices may be preferred.



% %!TEX root = main.tex

\section{Coarsening and Saved Inference}

The causal theories associated with both CBN and PO models are very profligate. They define many decisions that are unlikely to be considered in a pragmatic decision problem, and in practice it is usually only possible to determine the consequences of a small subset of these decisions if it is possible to determine any at all. In additon, proponents of both theories have advocated for the universality of the ``causal effects'' they represent:

\begin{quote}
The perspective that (1) the science exists independently of how we try to learn about it and that (2) if the model used for analysis of the resulting data is approximately correct, then the resulting posterior distribution will give a fair summary of the current state of knowledge of that science seems, at least to me, consistent with common views of the scientific enterprise
[...]
The potential outcomes, together with covariates, define the science in the sense that all causal estimands are functions of these values \citep{rubin_causal_2005}
\end{quote}

\begin{quote}
By representing the domain in the form of an assembly of stable mechanisms, we have in fact created an oracle capable of answering queries about the effects of a huge set of actions and action combinations \citep{pearl_causality:_2009}
\end{quote}

We present here a somewhat speculative account of what both of these approaches are trying to achieve based on the notions of \emph{coarsening} and \emph{saved inference}. With these, we can show that it is sometimes possible to reuse the results of inference performed with a dextrous theory $\mathbf{T}$ with a more pragmatic theory $\mathbf{T}'$.

\begin{definition}[Coarsening]
A theory $\mathbf{T}:\Theta\times D\to \Delta(\mathcal{E}\otimes\mathcal{F})$ can be coarsened to a theory $\mathbf{T}':\Theta\times D'\to \Delta(\mathcal{E}\otimes\mathcal{F})$ if there exists $M:D'\to \Delta(\mathcal{D})$ such that $(\xi \otimes \mathrm{Id}_D) \mathbf{T}'  = (\xi \otimes M) \mathbf{T}$. We say that $\mathbf{T}'$ is \emph{clumsier} than $\mathbf{T}$ or $\mathbf{T}$ is \emph{more dextrous} than $\mathbf{T}'$.
\end{definition}

Given $\mathbf{T}$, an event $A\in \mathcal{E}$ with $\xi H \mathds{1}_A >0$, write the theory conditioned on $A$ as $\mathbf{T}_\xi|A:D\to \Delta(\mathcal{F})$, defined as
\begin{align}
 \mathbf{T}_\xi|A:= (\xi H (A))^{-1}\begin{tikzpicture}
\path (0,0) node[dist] (theta) {$\xi$}
      +(0,-1) node (D) {}
      ++(0.5,0) coordinate (copy0)
      ++(0.5,0) node[kernel] (H) {$H$}
      +(0,-1) node[kernel] (C) {$C$}
      ++(0.7,0) node[expectation] (E) {$\mathds{1}_A$}
      +(0,-1) node (F) {};
\draw (theta) -- (copy0);
\draw (D) -- (C) -- (F);
\draw (copy0) to [bend right] (C);
\draw (copy0) to [bend left] (H);
\draw (H) -- (E);
\end{tikzpicture}
\end{align}

Note that $\mathbf{T}_\xi|A$ along with a strategy $\gamma\in \Delta(\mathcal{D})$ is the conditional probability of $\RV{F}$ by the elementary definition - for $B\in \mathcal{F}$, $\mathbf{T}_{\xi,\gamma}|A:B\mapsto \frac{(\xi\otimes\gamma) \mathbf{T} (A,B))}{\xi H (A)}$.

\begin{theorem}\label{th:mod_extn}
Given any prior $\xi$, a strategy $\gamma$ and $A\in \mathcal{E}$ such that $\mathbf{T}_\xi|A$ is defined, then there exists $\mathbf{M}$ such that $\mathbf{T}'_{\gamma,\xi} | A = \gamma M \mathbf{T}_\xi|A$ if and only if $\mathbf{T}'$ is a coarsening of $\mathbf{T}$.
\end{theorem}

\begin{proof}
Let the coarsening from $\mathbf{T}$ to $\mathbf{T}'$ be witnessed by $\mathbf{M}:D'\to \Delta(\mathcal{D})$. For arbitrary $\xi$, $A$ such that $\xi \mathbf{H} (A)>0$ and arbitrary $\gamma$:
\begin{align}
\gamma M \mathbf{T}_\xi|A &= (\xi \mathbf{H} (A))^{-1}\begin{tikzpicture}
\path (0,0) node[dist] (theta) {$\xi$}
      +(0,-1) node[dist] (D) {$\gamma$}
      +(0.5,-1) node[kernel] (M) {$\mathbf{M}$}
      ++(1,0) coordinate (copy0)
      ++(1,0) node[kernel] (H) {$\mathbf{H}$}
      +(0,-1) node[kernel] (C) {$\mathbf{C}$}
      ++(0.7,0) node[expectation] (E) {$\mathds{1}_A$}
      +(0,-1) node (F) {};
\draw (theta) -- (copy0);
\draw (D) -- (M)--(C) -- (F);
\draw (copy0) to [bend right] (C);
\draw (copy0) to [bend left] (H);
\draw (H) -- (E);
\end{tikzpicture}\\
&= (\xi \mathbf{H} (A))^{-1}\gamma (\xi\otimes \mathbf{M}) \mathbf{T} (\mathds{1}_A\otimes \mathrm{Id}_F)\\
&= (\xi \mathbf{H} (A))^{-1}\gamma (\xi\otimes \mathrm{Id}_D) \mathbf{T}' (\mathds{1}_A\otimes \mathrm{Id}_F)\\
&= \mathbf{T}'_{\gamma,\xi} | A
\end{align}



\end{proof}

In other words, if and only if $\mathbf{T}$ can be coarsened to $\mathbf{T}'$ then we can ``save'' the results of conditioning $\mathbf{T}_\xi$ on $A$ via $\mathbf{T}_\xi|A$ and later on we can determine the effects of some strategy $\gamma$ on $\mathbf{T}'$ via $\gamma \mathbf{M} \mathbf{T}_\xi|A$.

We will return to our discussion of the the ``effect of taking the treatment'' for an example. Suppose we have $\Theta=[0,1]^2:=\Theta_1\otimes \Theta_2$ where given $(\theta_1,\theta_2)\in\Theta$ we identify $\theta_1$ with ``treatment efficacy'' and $\theta_2$ with ``treatment susceptibility''. Let $Y,W=\{0,1\}$ where $Y$ is the set of outcomes and $W$ indicates whether or not a patient took the treatment. Define a potential outcomes model $H_{PO}:\Theta\to \Delta(\mathcal{Y}^2)$, $H_W:\Theta\to \Delta(\mathcal{W})$ and $H_Y:W\times Y^2\to \Delta(\mathcal{Y})$. Furthermore, suppose we have $D=[0,1]^2$ and $C_W:\Theta_2\times D\to \Delta(\mathcal{W})$ defined by $C_W(\theta_2,d_1,d_2;A):= \theta_2(d_1\delta_1(A) (1-d_1)\delta_0(A)) + (1-\theta_2)(d_2\delta_1(A)+(1-d_2)\delta_0(A))$; that is, $d_1$ and $d_2$ parametrise the set of Markov kernels $\Theta_2\to \Delta(\mathcal{W})$. Then $\langle H_{PO},H_W,H_Y,C_W\rangle$ defines a causal theory $T$. Suppose we observe $A\in \mathcal{E}$; then for $(d_1,d_2)\in D$, $B\in \mathcal{E}$:

\begin{align}
	T_\xi|A (d_1,d_2;B) = \frac{1}{\int_\Theta (H_{PO,\theta}\otimes H_{W,\theta})H_Y(A)}\int_\Theta (H_{PO,\theta}\otimes H_{W,\theta})H_Y(A) C_{W,\theta} (d_1,d_2;B) d\xi
\end{align}

$T_\xi|A$ describes a Markov kernel from $D\to \Delta(\mathcal{E})$. However, it is unlikely to be the case that $D$ describes the actual decisions we have available - we probably don't have the ability to choose the exact relationship between treatment susceptibility and treatment taking. In fact, all that may be actually available are some decisions $D'=\{0,1\}$ where 0 represents no prescription and 1 represents prescription; these may yield uncertain relationships between $\theta_2$ and $\RV{W}$. While we might not know which relationship a decision to prescribe or not induces, we might accept that influencing this relationship is the only important consequence of this decision. That is, we might accept that for \emph{some} $M:D'\to \Delta(\mathcal{D})$ a causal theory $T':\Theta\times D'\to \Delta(\mathcal{E}^2)$ of the form $T' = (\mathrm{Id}_\Theta\otimes M)T$ is appropriate. Then, because $T'$ is an extension of $T$, rather than having to rerun our inference we can simply compute $MT_\xi|A$.

% While there are numerous differences between the Potential Outcomes and Causal Bayesian Network approach to causality, it is interesting to reflect on their different approaches to handling Theorem \ref{th:ncinco}. The CBN approach fixes a number of conditional probabilities among variables unless they are directly intervened on and combines this with a standard ``hard-intervention'' operation. Potential outcomes, in the form we discuss here, represents by potential outcomes a fixed set of properties of outcomes (``the science'', as Rubin calls it) which are then partially revealed by an assignment function which may respond in problem specific ways to decisions. As we have discussed regarding ETT, the potential outcomes approach is capable of representing decisions that are known to have certain effects but we may be uncertain as to how exactly they achieve these effects, though this is not always enough to satisfactorily represent the problem of interest (see the discussion of ITT). The CBN approach features a hard transition between fixed flexible conditional probabilities - either an intervention is not on a node, in which case its probability conditional on its parents is fixed, or it is on that node in which case the conditional probability is vastly different. This doesn't appear to be ideal for representing uncertainty over how a decision might correspond to an intervention. In fact \emph{any} CBN with hard interventions can in fact represent any causal theory if we permit decisions to correspond to unknown, state-dependent mixtures of interventions (this reflects the fact that every joint probability distribution can be achieved with the right mixture of hard interventions on every node). The fact that we lose dependence on the graph suggests that a na\"ive approach to uncertainty over the decision bias may be too general. There are many versions of CBNs with generalised interventions that may address this issue.

% It is interesting to consider whether there might be principles of causal inference that eschew the two part approach of fixing some underlying notion of ``the science'' and separately adding in some kind of decision bias. One could imagine, for example, a causal theory that posits that consequences minimise some combination of causally appropriate dissimilarity measures from the observational distribution and from a decision-dependent target distribution without any clear commitments to invariant principles of science. It's not obvious how we should construct such measures without appealing to some notion of the underlying science, but we regard it as an interesting question nonetheless.


% %!TEX root = main.tex

\section{Discussion}

We have introduced an original approach to formulating questions of causal inference and analysing approaches to causal modelling. We take cues from statistical decision theory in the realm of problem definition and make heavy use of the theory of Markov kernels for reasoning about causal theories, the central object of our approach. Our approach makes crystal clear the distinction between ``statistical'' and ``causal'' knowledge -- the former is represented by a statistical experiment and the latter by a causal theory. We can also plausibly interpret the two major existing approaches of Causal Bayesian Networks and Potenial Outcomes as tools to generate causal theories, though there are arbitrary decisions that must be made in order to do this.

Though we develop this theory in the context of ``small world decision problems'' \citep{joyce_foundations_1999}, we also make progress on the question of what causal theories are doing apart from facilitating reasoning about small world decision problems. We show that if a potentially unrealistic theory can be related to a more realistic theory by coarsening, then knowledge of consequences under the former may be informative about consequences under the latter. 

While we do not address the unique questions that can be raised with counterfactual models \citep{pearl_causality:_2009}, our approach suggests an alternative view for the relationship between counterfactual and interventional causal models. Rather than occupying different levels of a hierarchy, each yields causal theories with different kinds of rich decisions sets. It is plausible that the different sets of decisions each approach provides may be amenable to coarsening in different domains. Indeed, we see extensive discussion of counterfactual treatment effects in the econometrics literature, where decisions usually involve changing incentives which can plausibly be understood as altering the assignment function $\mathbf{W}$ in unpredictable ways \citep{angrist_mastering_2014,carneiro_evaluating_2010,imbens_identification_1994}. Causal Bayesian Networks, on the other hand, have found applications in the study of biological systems which typically feature large numbers of variables which permit a wide variety of targetted interventions \citep{sachs_causal_2005,maathuis_estimating_2009}.

While Theorem \ref{th:mod_extn} suggests that coarsening can be useful for ``reusing knowledge'' between compatible causal theories, this is only likely to be helpful if it is possible to determine that a theory $\mathbf{T}'$ is a coarsening of $\mathbf{T}$ under $\mathbf{M}$ without having to perform inference on both $\mathbf{T}$ and $\mathbf{T}'$ to satisfy ourselves that the consequences do indeed match in detail for both theories. Understanding when we can consider $\mathbf{T}'$ to be a coarsening of $\mathbf{T}$ and when it is useful to do so is an important development of the ideas presented here. Informally, we want to understand the question ``if I know my decision definitely results in $\RV{X}=x$, when do I also know it corresponds to $do(\RV{X}=x)$?''

A number of the results here are predicated on discrete spaces, a step that allows us to disregard questions of measurability. A second important direction of development is extending this theory to continuous spaces and understanding what limitations this introduces. Relatedly, the notions of conditional probability, conditioning, independence and Bayesian inversion are well understood in the context of probability measures, including in their string diagrammatic treatment \citep{cho_disintegration_2019}, but we are not aware of analogues of these notions for general Markov kernels, if they exist. They would be invaluable tools in the analysis of causal theories, which, owing to the dependence on $D$, are not naturally dealt with as probability measures.

The string diagram notation we use has a strong connection with the DAGs \citep{fong_causal_2013} used in causal graphical models as well as to influence diagrams\citep{dawid_influence_2002}, as do Markov kernels themselves. It would not be surprising if there were a deep connection between the two. 


% %!TEX root = main.tex

\section{Invariance and Capital-C Causality}

CSDT features \emph{consequences} - that is, probabilistic relations between decisions and results - but it does not feature \emph{causal effects}, which seem to be probabilistic relations between random variables on the observation space $E$ that are not necessarily disintegrations of a joint distribution. Here I propose the following notion of a causal effect in CSDT: if I have prior knowledge about the consequences of my decisions $D$ on some random variable $\RV{B}$ via $\mathbf{C}_0\mathbf{B}$ and I can extend this to the consequences on $\RV{Y}$ using some Markov kernel $\mathbf{G}_\theta:B\to \Delta(\mathcal{Y})$ via composition $\mathbf{C}_\theta \mathbf{Y} = \mathbf{C}_0\mathbf{B}\mathbf{G}_\theta$ then we'll say that $\mathbf{G}_\theta$ is the ``causal effect'' of $\RV{B}$ on $\RV{Y}$ in state $\theta$.

This notion accords with intuitions about do-interventions - if I have the option to $do(\RV{B})$ then I definitely know the effect of my decision on $\RV{B}$. Furthermore, using do interventions I assume that the effect of $do(\RV{B}=x)$ on $\RV{Y}$ is given by fixing $\RV{B}$ to $x$ and then computing $P(\RV{Y}|do(\RV{B})=x)$, a special case of the composition above if we let $\mathbf{G}_\theta=P(\RV{Y}|do(\RV{B}))$. 

However, the notion of causal effect proposed here doesn't make additional commitments that $do(\RV{B})$ does - namely, that I also know $do(\RV{B})$ has no direct effect on any other variable. Unlike the assumption of prior knowledge, the ``no direct effects'' assumption is cannot even be formulated within CSDT.

The notion of ``causal effect'' given here permits a (to my knowledge) novel set of assumptions under which this type of causal effect of $\RV{B}$ on $\RV{Y}$ can be inferred. Importantly, it avoids any assumptions of ``conditional independence of unobservables'', and is to my knowledge the only known case of ``causal identifiability'' that achieves this.

Suppose $\mathbf{C}:D\times \Theta\to\Delta(\mathcal{E})$ is the consequence of interest, and furthermore given some $\theta\in\Theta$ we have $\mathbf{A}_\theta:D\to \Delta(\mathcal{A})$ and $\mathbf{B}:E\to \Delta(\mathcal{B})$ such that the observations are distributed according to
\begin{align}
	\mathbf{H}_\theta := \begin{tikzpicture}
		\path (0,0) node[dist] (S) {$\gamma_\theta$}
		++ (0.5,0) coordinate (copy0)
		++ (0.5,0) node[kernel] (C) {$\mathbf{C}_\theta$}
		+  (0,1) node[kernel] (A) {$\mathbf{A}_\theta$}
		++ (0.5,0) coordinate (copy1)
		++ (0.5,0) node[kernel] (B) {$\mathbf{B}$}
		+  (0,0.5) coordinate (Y)
		++ (0.7,0) node (Bout) {$\RV{B}$}
		+  (0,1) node (Aout) {$\RV{A}$}
		+  (0,0.5) node (Yout) {$\RV{Y}$};
		\draw (S) -- (C) -- (B) -- (Bout);
		\draw (copy0) to [bend left] (A) (A) -- (Aout);
		\draw (copy1) to [bend left] (Y) -- (Yout);
	\end{tikzpicture}\label{eq:capital_c_observations}
\end{align}

That is, the observations $\mathbf{H}_\theta$ are the inputs and outputs of $\mathbf{C}_\theta$ where the inputs are masked by $\mathbf{A}_\theta$ and we run one copy of the outputs through a fixed $\mathbf{B}$. Without the mask, we could recover $\mathbf{C}_\theta$ from $\mathbf{H}_\theta$ via disintegration $\RV{D}\dashrightarrow \RV{Y}$.

For all $\theta$, assume $\mathbf{A}_\theta$ is $\gamma_\theta$-almost surely right invertible and that $\gamma_\theta$ is strictly positive. Together these are strong assumptions. Note that as $\theta$ is unknown, despite the fact that $\mathbf{A}_\theta$ is right invertible, we can't recover it's inputs, so we can't simply disintegrate.

This setup is something like an instrumental variables (IV) setup where $\gamma_\theta$ is a source of ``exogenous'' variation. It is stronger than a typical IV setup in that $\gamma_\theta$ is strictly positive and $\mathbf{A}_\theta$ is right invertible, but it is weaker than a typical IV setup in that we make no assumptions at all about model class and fewer assumptions about the relationships between $\RV{A}$, $\RV{B}$ and $\RV{Y}$.

It is also somewhat similar to \citet{arjovsky_invariant_2019}, itself based on \citet{peters_causal_2016}, if $\RV{A}$ is understood as an environment indicator. Under this interpretation, the assumption that $\gamma_\theta$ is strictly positive is much stronger than \citet{arjovsky_invariant_2019}, but on the other hand we do not assume any particular relationship between $\RV{B}$ and $\RV{Y}$ where \citet{arjovsky_invariant_2019} assumes a particular form of SEM. An interpretational difference is that while we regard $D$ as a set of feasible decisions, \citet{arjovsky_invariant_2019} considers the set of environments thus:
\begin{quote}
Here, the set of all environments contains all possible experimental conditions concerning oursystem of variables, both observable and hypothetical.
\end{quote}
As an aside, we could potentially interpret such a set of environments as an rich set of decisions which may be coarsened to a realistic set of decisions.

Under certain conditions, this setup allows for the extension of causal knowledge via observed data. In particular, if we find the conditional probability of $\RV{Y}$ on $\RV{A}$ and $\RV{B}$ (written $[\RV{Y}|\RV{A}\utimes\RV{B}]_\theta$) is independent of $\RV{A}$, then given prior knowledge for the effect of a decision on $\RV{B}$ we can deduce the full consequence map $\mathbf{C}_\theta$ from our prior knowledge and the disintegration $[\RV{Y}|\RV{B}]_\theta$ (see Theorem \ref{th:inv_ci}). Note that no assumptions have been made about ``causal'' relationships between $\RV{B}$ and $\RV{Y}$ - $[\RV{Y}|\RV{B}]_\theta$ is an ordinary disintegration, not a platonic Markov kernel/structural equation/FFRCISTG/whatever.

A kernel $\RV{B}$ that throws away more information is advantageous in the sense that less prior knowledge is needed to determine $\mathbf{C}_\theta$.
\begin{example}[Waste collection]
A council is deciding on how to implement a waste collection service to reduce littering $\RV{Y}$. For every possible service $d\in D$, the collection schedule $\RV{S}$ that will be achieved is known prior to any investigation (services may differ in other ways - e.g. the bin types may differ, and these differences may or may not be known in advance). In addition, the council has obtained weekly collection and littering data from a set $A$ of other councils with their own waste collection services that are known to have faced the same unknown consequence map $\mathbf{C}_\theta$. Each other council has implemented exactly one possible service $d\in D$ and enough councils were surveyed that all possible choices of service have been sampled, though they do not know which councils have implemented which systems. These conditions ensure that for each service $d$ the set of councils $A_d\subset A$ implementing that service is disjoint from the set of councils implementing any other service, and hence the unknown map from services to councils $\mathbf{A}_\theta:D\to \Delta(\mathcal{A})$ is right invertible.

It is found by the council's statisticians that the disintegration $[\RV{Y}|\RV{A}\utimes\RV{S}]_\theta$ is independent of $\RV{A}$. Thus by theorem \ref{th:inv_ci} the impact of any service $d$ on the rate of littering $\RV{Y}$ can be found via the collection schedule $\RV{S}$ that will be achieved by that service and the disintegration $[\RV{Y}|\RV{S}]_\theta$.

This is a surprisingly strong conclusion from the assumptions made. We appear to have the ability to deduce a ``causal effect'' from observational data in a context that looks remarkably similar to standard examples of when this \emph{can't} be done. In fact, the conclusion is stronger than a $do$-style causal relationship, as $do$ interventions assume we know a decision has no ``direct effect'' on any variable other than the target, whereas here we only assume the consequence of $d$ on $\RV{S}$ is known.

An objection might be that other councils' waste collection service choices are determined, in part, by some unobserved background factors. Suppose that these background factors take values in some space $K$. One means of formalising this is that $\mathbf{C}_\theta$ factorises:
\begin{align}
\begin{tikzpicture}
 \path (0,0) node (D) {$\RV{D}$}
   ++(0.8,0) node[kernel] (C) {$\mathbf{C}_\theta$}
   ++(0.7,0) node (E) {$\RV{E}$};
 \draw (D) -- (C) -- (E);
\end{tikzpicture}=
\begin{tikzpicture}
	\path (0,0) node (D) {$\RV{D}$}
	+ (0,-0.4) node[dist] (K) {$\kappa$}
	++(0.8,-0.2) node[kernel] (C) {$\mathbf{F}_\theta$}
	++ (0.7,0) node (E) {$\RV{E}$};
	\draw (D) -- ($(C.west)+(0,0.15)$) (C) -- (E);
	\draw (K) -- ($(C.west)+(0,-0.15)$);
\end{tikzpicture}
\end{align}

Where $\kappa\in \Delta(\mathcal{K})$ is a distribution on background factors and $\mathbf{F}_\theta:D\times K\to \Delta(\mathcal{E})$ maps decisions and background factors to consequences. For $d\neq d'$ we may have:
\begin{align}
\begin{tikzpicture}
	\path (0,0) node[dist] (D) {$\delta_d$}
	+ (0,-1) node (K) {$\RV{K}$}
	++(0.8,-0.5) node[kernel] (C) {$\mathbf{F}_\theta$}
	++(0.7,0) node[kernel] (S) {$\mathbf{S}$}
	++ (0.5,0) node (E) {$\RV{S}$};
	\draw (D) -- ($(C.west)+(0,0.15)$) (C) -- (S) -- (E);
	\draw (K) -- ($(C.west)+(0,-0.15)$);
\end{tikzpicture}\neq
\begin{tikzpicture}
	\path (0,0) node[dist] (D) {$\delta_{d'}$}
	+ (0,-1) node (K) {$\RV{K}$}
	++(0.8,-0.5) node[kernel] (C) {$\mathbf{F}_\theta$}
	++(0.7,0) node[kernel] (S) {$\mathbf{S}$}
	++ (0.5,0) node (E) {$\RV{S}$};
	\draw (D) -- ($(C.west)+(0,0.15)$) (C) -- (S) -- (E);
	\draw (K) -- ($(C.west)+(0,-0.15)$);
\end{tikzpicture}
\end{align}

That is, different decisions may induce different relationships between background characteristics $\RV{K}$ and collection schedules $\RV{S}$. Formally, introducing this extra complication has not violated any of our assumptions - the causal inference is still valid! Practically, we would typically need a much larger set of decisions $D$ in this case to account for the number of plausible relationships between $\RV{K}$ and $\RV{S}$, and this may give us more reason to question the assumption that $\gamma_\theta$ is strictly positive - i.e. that the set of councils implementing each decision has positive measure. Nonetheless, this appears to be quite different to existing conditions for observational causal inference: we have allowed for $\RV{K}$ to affect both $\RV{S}$ and $\RV{Y}$ but in contrast to existing approaches \textbf{we do not need to see $\RV{K}$ in order to -- sometimes -- infer the ``causal effect'' of $\RV{S}$ on $\RV{Y}$}.


\end{example}



\begin{theorem}\label{th:inv_ci}
 Suppose we have a causal theory $\mathbf{T}:\Theta\times D\to \Delta(\mathcal{E}^2)$ where for $\theta\in \Theta$ we have consequence $\mathbf{C}_\theta$ and experiment $\mathbf{H}_\theta$ given by \ref{eq:capital_c_observations}. Suppose for some $\mathbf{C}_0:D\to \Delta(\mathcal{E})$ we have $\mathbf{C}_\theta \mathbf{B} = \mathbf{C}_0 \mathbf{B}$ for all $\theta\in \Theta$.

 For all $\theta\in \Theta$ such that $[\RV{Y}|\RV{A}\utimes\RV{B}]_{\theta}$ is independent of $\RV{A}$ we have $\mathbf{C}_{\theta} =\mathbf{C}_0 \mathbf{B} [\mathbf{Y|B}]_{\theta}$
\end{theorem}

\begin{proof}
By the definition of disintegration

\begin{align}
\begin{tikzpicture}
		\path (0,0) node[dist] (S) {$\gamma_\theta$}
		++ (0.5,0) coordinate (copy0)
		++ (0.5,0) node[kernel] (C) {$\mathbf{C}_\theta$}
		+  (0,1) node[kernel] (A) {$\mathbf{A}_\theta$}
		++ (0.5,0) coordinate (copy1)
		++ (0.5,0) node[kernel] (B) {$\mathbf{B}$}
		+  (0,0.5) coordinate (Y)
		++ (0.7,0) node (Bout) {$\RV{B}$}
		+  (0,1) node (Aout) {$\RV{A}$}
		+  (0,0.5) node (Yout) {$\RV{Y}$};
		\draw (S) -- (C) -- (B) -- (Bout);
		\draw (copy0) to [bend left] (A) (A) -- (Aout);
		\draw (copy1) to [bend left] (Y) -- (Yout);
	\end{tikzpicture} &= 
	\begin{tikzpicture}
		\path (0,0) node[dist] (S) {$\gamma_\theta$}
		++ (0.5,0) coordinate (copy0)
		++ (0.5,0) node[kernel] (C) {$\mathbf{C}_\theta$}
		+  (0,1) node[kernel] (A) {$\mathbf{A}_\theta$}
		+  (0.6,1) coordinate (copy2)
		++ (.8,0) node[kernel] (B) {$\mathbf{B}$}
		++ (0.5,0) coordinate (copy1)
		++  (1,0.5) node[kernel] (Y) {$[\RV{Y}|\RV{A}\utimes \RV{B}]_\theta$}
		++ (1.3,-0.5) node (Bout) {$\RV{B}$}
		+  (0,1) node (Aout) {$\RV{A}$}
		+  (0,0.5) node (Yout) {$\RV{Y}$};
		\draw (S) -- (C) -- (B) -- (Bout);
		\draw (copy0) to [bend left] (A) (A) -- (Aout);
		\draw (copy1) to [bend left] ($(Y.west)+(0,-0.15)$) (Y) -- (Yout);
		\draw (copy2) to [bend right] ($(Y.west)+(0,0.15)$);
	\end{tikzpicture}\\
	&= 	\begin{tikzpicture}
		\path (0,0) node[dist] (S) {$\gamma_\theta$}
		++ (0.5,0) coordinate (copy0)
		++ (0.5,0) node[kernel] (C) {$\mathbf{C}_\theta$}
		+  (0,1) node[kernel] (A) {$\mathbf{A}_\theta$}
		+  (0.6,1) coordinate (copy2)
		++ (.8,0) node[kernel] (B) {$\mathbf{B}$}
		++ (0.5,0) coordinate (copy1)
		++  (1,0.5) node[kernel] (Y) {$[\RV{Y}|\RV{B}]_\theta$}
		++ (1.3,-0.5) node (Bout) {$\RV{B}$}
		+  (0,1) node (Aout) {$\RV{A}$}
		+  (0,0.5) node (Yout) {$\RV{Y}$};
		\draw (S) -- (C) -- (B) -- (Bout);
		\draw (copy0) to [bend left] (A) (A) -- (Aout);
		\draw (copy1) to [bend left] ($(Y.west)+(0,-0.15)$) (Y) -- (Yout);
	\end{tikzpicture}\label{eq:from_independence}
\end{align}
Where \ref{eq:from_independence} follows from the independence of $[\RV{Y}|\RV{A}\utimes\RV{B}]_{\theta}$ from $\RV{A}$.

In addition, we have 

\begin{align}
\begin{tikzpicture}
		\path (0,0) node[dist] (S) {$\gamma_\theta$}
		++ (0.5,0) coordinate (copy0)
		++ (0.5,0) node[kernel] (C) {$\mathbf{C}_\theta$}
		+  (0,1) coordinate (A) 
		++ (0.5,0) coordinate (copy1)
		++ (0.5,0) node[kernel] (B) {$\mathbf{B}$}
		+  (0,0.5) coordinate (Y)
		++ (0.7,0) node (Bout) {$\RV{B}$}
		+  (0,1) node (Aout) {$\RV{D}$}
		+  (0,0.5) node (Yout) {$\RV{Y}$};
		\draw (S) -- (C) -- (B) -- (Bout);
		\draw (copy0) to [bend left] (A) (A) -- (Aout);
		\draw (copy1) to [bend left] (Y) -- (Yout);
	\end{tikzpicture} = 
	\begin{tikzpicture}
		\path (0,0) node[dist] (S) {$\gamma_\theta$}
		++ (0.5,0) coordinate (copy0)
		++ (0.5,0) node[kernel] (C) {$\mathbf{C}_\theta$}
		+  (0,1) node[kernel] (A) {$\mathbf{A}_\theta$}
		+  (1,1) node[kernel] (Ainv) {$\mathbf{A}_\theta^{-1}$}
		++ (0.5,0) coordinate (copy1)
		++ (0.5,0) node[kernel] (B) {$\mathbf{B}$}
		+  (0,0.5) coordinate (Y)
		++ (0.7,0) node (Bout) {$\RV{B}$}
		+  (0,1) node (Aout) {$\RV{D}$}
		+  (0,0.5) node (Yout) {$\RV{Y}$};
		\draw (S) -- (C) -- (B) -- (Bout);
		\draw (copy0) to [bend left] (A) (A) -- (Ainv) -- (Aout);
		\draw (copy1) to [bend left] (Y) -- (Yout);
	\end{tikzpicture}\label{eq:mult_by_inverse}
\end{align}

Thus

\begin{align}
\begin{tikzpicture}
		\path (0,0) node[dist] (S) {$\gamma_\theta$}
		++ (0.5,0) coordinate (copy0)
		++ (0.5,0) node[kernel] (C) {$\mathbf{C}_\theta$}
		+  (0,1) coordinate (A) 
		++ (0.5,0) coordinate (copy1)
		++ (0.5,0) node[kernel] (B) {$\mathbf{B}$}
		+  (0,0.5) coordinate (Y)
		++ (0.7,0) node (Bout) {$\RV{B}$}
		+  (0,1) node (Aout) {$\RV{D}$}
		+  (0,0.5) node (Yout) {$\RV{Y}$};
		\draw (S) -- (C) -- (B) -- (Bout);
		\draw (copy0) to [bend left] (A) (A) -- (Aout);
		\draw (copy1) to [bend left] (Y) -- (Yout);
	\end{tikzpicture} = 
\begin{tikzpicture}
		\path (0,0) node[dist] (S) {$\gamma_\theta$}
		++ (0.5,0) coordinate (copy0)
		++ (0.5,0) node[kernel] (C) {$\mathbf{C}_\theta$}
		+  (0,1) coordinate (A)	
		+  (0.6,1) coordinate (copy2)
		++ (.8,0) node[kernel] (B) {$\mathbf{B}$}
		++ (0.5,0) coordinate (copy1)
		++  (1,0.5) node[kernel] (Y) {$[\RV{Y}|\RV{B}]_\theta$}
		++ (1.3,-0.5) node (Bout) {$\RV{B}$}
		+  (0,1) node (Aout) {$\RV{D}$}
		+  (0,0.5) node (Yout) {$\RV{Y}$};
		\draw (S) -- (C) -- (B) -- (Bout);
		\draw (copy0) to [bend left] (A) (A) -- (Aout);
		\draw (copy1) to [bend left] ($(Y.west)+(0,-0.15)$) (Y) -- (Yout);
	\end{tikzpicture}
\end{align}

From Lemma \ref{lem:eq_disints} and by the assumption of strict positivity on $\gamma_\theta$, we therefore have
\begin{align}
\begin{tikzpicture}
		\path (0,0) coordinate (S)
		++ (0.5,0) node[kernel] (C) {$\mathbf{C}_\theta$}
		+  (0,1) coordinate (A) 
		++ (0.5,0) coordinate (copy1)
		++ (0.5,0) node[kernel] (B) {$\mathbf{B}$}
		+  (0,0.5) coordinate (Y)
		++ (0.7,0) node (Bout) {$\RV{B}$}
		+  (0,0.5) node (Yout) {$\RV{Y}$};
		\draw (S) -- (C) -- (B) -- (Bout);
		\draw (copy1) to [bend left] (Y) -- (Yout);
	\end{tikzpicture} &= 
\begin{tikzpicture}
		\path (0,0) coordinate (S)
		++ (0.5,0) node[kernel] (C) {$\mathbf{C}_\theta$}
		++ (.8,0) node[kernel] (B) {$\mathbf{B}$}
		++ (0.5,0) coordinate (copy1)
		++  (1,0.5) node[kernel] (Y) {$[\RV{Y}|\RV{B}]_\theta$}
		++ (1.3,-0.5) node (Bout) {$\RV{B}$}
		+  (0,0.5) node (Yout) {$\RV{Y}$};
		\draw (S) -- (C) -- (B) -- (Bout);
		\draw (copy1) to [bend left] ($(Y.west)+(0,-0.15)$) (Y) -- (Yout);
	\end{tikzpicture}\label{eq:joint_equality}\\
	\begin{tikzpicture}
		\path (0,0) coordinate (S)
		++ (0.5,0) node[kernel] (C) {$\mathbf{C}_\theta$}
		+  (0.8,0) node (Yout) {$\RV{Y}$};
		\draw (S) -- (C) -- (Yout);
	\end{tikzpicture} &= 
\begin{tikzpicture}
		\path (0,0) coordinate (S)
		++ (0.5,0) node[kernel] (C) {$\mathbf{C}_\theta$}
		++(0.8,0) node[kernel] (B) {$\mathbf{B}$}
		++  (1,0) node[kernel] (Y) {$[\RV{Y}|\RV{B}]_\theta$}
		+  (1,0) node (Yout) {$\RV{Y}$};
		\draw (S) -- (C) -- (B) -- (Y) -- (Yout);
	\end{tikzpicture}\label{eq:consequence_equality}\\
	&= \begin{tikzpicture}
		\path (0,0) coordinate (S)
		++ (0.5,0) node[kernel] (C) {$\mathbf{C}_0$}
		++(0.8,0) node[kernel] (B) {$\mathbf{B}$}
		++  (1,0) node[kernel] (Y) {$[\RV{Y}|\RV{B}]_\theta$}
		+  (1,0) node (Yout) {$\RV{Y}$};
		\draw (S) -- (C) -- (B) -- (Y) -- (Yout);
	\end{tikzpicture}
\end{align}

Where \ref{eq:consequence_equality} follows from marginalisation of \ref{eq:joint_equality}.

\end{proof}

A non-invertible $\mathbf{A}_\theta$ means \ref{eq:mult_by_inverse} doesn't hold. Given that $\mathbf{A}_\theta$ is arbitrary apart from the fact that it is invertible, I wonder if a channel capacity lower boundon $\mathbf{A}_\theta$ could give an upper bound on the ``distance'' between the kernels on the left and right of \ref{eq:mult_by_inverse} using an appropriate replacement for $\mathbf{A}_\theta^{-1}$.


\begin{lemma}\label{lem:eq_disints}
Given strictly positive probability measure $\gamma\in \Delta(\mathcal{D})$ and Markov kernels $\mathbf{X}:D\to \Delta(\mathcal{E})$ and $\mathbf{Y}:D\to \Delta(\mathcal{E})$ if
\begin{align}
\begin{tikzpicture}
 \path (0,0) node[dist] (G) {$\gamma$}
 ++(0.5,0) coordinate (copy0)
 ++(0.5,0) node[kernel] (X) {$\mathbf{X}$}
 ++(0.7,0) node (Xout) {$\RV{E}$}
 +(0,0.5) node (Gout) {$\RV{D}$};
 \draw (G) -- (X) --(Xout);
 \draw (copy0) to [bend left] (Gout);
\end{tikzpicture}= 
\begin{tikzpicture}
 \path (0,0) node[dist] (G) {$\gamma$}
 ++(0.5,0) coordinate (copy0)
 ++(0.5,0) node[kernel] (X) {$\mathbf{Y}$}
 ++(0.7,0) node (Xout) {$\RV{E}$}
 +(0,0.5) node (Gout) {$\RV{D}$};
 \draw (G) -- (X) --(Xout);
 \draw (copy0) to [bend left] (Gout);
\end{tikzpicture}\label{eq:disints}
\end{align}

Then $\mathbf{X}=\mathbf{Y}$.

\end{lemma}

\begin{proof}
We note that both $\RV{X}$ and $\RV{Y}$ are $\RV{D}\dashrightarrow \RV{E}$ disintegrations of \ref{eq:disints}, and so they must be $\gamma$-almost surely equal. Strict positivity means they must in fact be equal.
\end{proof}




\bibliographystyle{plainnat}
\bibliography{references}

\appendix
\newpage
\section*{Appendix:}

% %!TEX root = main.tex

\section{Appendix: CBN representation as a causal theory}\label{sec:cbn_as_ct}

\begin{definition}[Elementary Causal Bayesian Network]

Given $D$, $E$, $\Theta$, random variables $\{\RV{X}^i\}_{i\in [n]}$ on $E$, a distinguished variable $\RV{X}^0$ taking values in $D$ and a causal theory $T:\Theta\times D\to \Delta(\mathcal{E}\otimes\mathcal{E})$ with $H:= T(\mathrm{Id}_E\otimes *_E)$ and $C:= T(*_E\otimes \mathrm{Id}_E)$, an \emph{elementary Causal Bayesian Network} (eCBN) compatible with $T$ is a directed acyclic graph (DAG) $\mathcal{G}$ with nodes $\{\RV{X}^i\}_{i\in [n]}$ such that

\begin{enumerate}
    \item $H_\theta$ and $C_{\theta,d}$ are compatible with $\mathcal{G}$ (see \citet{pearl_causality:_2009})
    \item $C_{\theta,d} F_{\RV{X}^i}=\delta_{d}$
    \item For all $i\neq 0$, $C_{\theta|\PA{\mathcal{G}}{\RV{X}^i}} F_{\RV{X}^i}=H_{\theta|\PA{\mathcal{G}}{\RV{X}^i}}F_{\RV{X}^i} $, $H_\theta$-almost surely
\end{enumerate}
\end{definition}


\section{Appendix: No causes in no causes out}\label{sec:ncinco}

A key result in statistical learning theory is the requirement that, in order for a hypothesis class to be learnable, it must have finite VC-dimension. The concept of controlling the size of the hypothesis class plays a fundamental role across the field of machine learning, from formal proofs of learnability to techniques based less formally on the notion of the bias-variance tradeoff. CSDPs are closely related to statistical learning problems, and it is highly likely that results of this type can be developed for causal problems.

Apart from any inductive biases necessary for learnability, causal theories also require a \emph{decision bias} - a causal theory that does not distinguish decisions yields only trivial results. This is distinct from a restriction on the flexibility or capacity of a causal theory. Given a prior, the requirement is that, conditional on some set of observations, a causal theory yields different consequences for different decisions. 

Define the pairwise swap $U_{dd'}:D\to \Delta(\mathcal{D})$ to be the kernel that sends $d\mapsto \delta_{d'}$, $d'\mapsto \delta_d$ and all other $d''\to \delta_{d''}$.

\begin{theorem}[No causes in, no causes out (Bayes)]\label{th:ncinco}
If a causal theory $T:\Theta\times D\to \Delta(\mathcal{E}\otimes\mathcal{F})$ and a prior $\xi\in \Delta(\Theta)$ are such that for all pairwise swaps $U_{dd'}:D\to \Delta(\mathcal{D})$, $(\xi\otimes U_{dd'})T = (\xi\otimes I)T$ and $D$ is discrete then all decision strategies are Bayes.
\end{theorem}

\begin{proof}
Defining $F_{\_d_0}:d\mapsto \delta_{d_0}$ for all $d\in D$, we will show that for all $J$, $S_\xi(J)=S_\xi(JF_{\_d_0}):=S_0$.

By assumption, for all $d\in D$, utility functions $u$:
\begin{align}
	\int_\Theta H_\theta J(\{d\}) C_\theta u(d) d\xi &= \int_\Theta H_\theta J(\{d\}) U_{dd_0} C_\theta u(d) d\xi\\
													 &= \int_\Theta H_\theta J(\{d\}) F_{\_d_0} C_\theta u(d) d\xi\label{eq:agree_on_d}\\
\therefore \sum_{d\in D} \int_\Theta H_\theta J(\{d\}) F_{\_d_0} C_\theta(d;A) d\xi &= \sum_{d\in D} \int_\Theta H_\theta J(\{d\}) C_\theta u(d) d\xi\\
													 &= \int_\Theta \sum_{d\in D} H_\theta J(\{d\}) C_\theta u (d) d\xi\\
													 &= \int_\Theta H_\theta J C_\theta u d\xi\\
													 &= S_\xi(J)\\
													 &= S_\xi(JF_{\_d_0})
\end{align}
Where \ref{eq:agree_on_d} follows from the fact that evaluation at $d$ guarantees $U_{dd_0} C_\theta u(d) = F_{\_d_0} C_\theta u(d)$.
\end{proof}

\begin{corollary}
If a causal theory $T$ with a prior $\xi$ and discrete decision set $D$ yields a nontrivial ordering of decision strategies, then there exists $d,d'\in D$ such that $(\xi\otimes \delta_d) T\neq (\xi\otimes \delta_{d'}) T$.
\end{corollary}

Somewhat surprisingly, the minimax rule may yield preferences over decisions under such circumstances; in particular, a uniform strategy is always minimax, though other strategies may not be. This is because the consequences of a uniform strategy may be less extreme than the consequences of any other strategy.

\begin{theorem}[No causes in, uniform strategy out (minimax)]
If a causal theory $T:\Theta\times D\to \Delta(\mathcal{E}\otimes\mathcal{F})$ with finite $D$ is such that for all pairwise swaps $U_{dd'}:D\to \Delta(\mathcal{D})$, $\theta\in \Theta$ there is some $\theta'$ such that $T_{\theta,\cdot} = (I\otimes U)T_{\theta',\cdot}$ then the uniform decision strategy is minimax.
\end{theorem}

\begin{proof}
Note that for finite $D$, the invertible maps $D\to \Delta(\mathcal{D})$ are permutation maps which can be factorised as a sequence of pairwise swaps.

Call $J_U$ the stubborn uniform strategy $J_U:x\mapsto U(\mathcal{D})$ for all $x\in E$. Suppose there is some nonuniform $J$ such that $\max_\theta S(J,\theta) < \max_\theta S(J_U,\theta)$. Suppose $S(J_U,\theta)$ is maximised in some state $\theta^0$ where $S(J_d,\theta^0)=S(J_{d'},\theta^0)$ for all $d,d'\in D$. Then $S(J,\theta^0)=S(J_U,\theta^0)$, contraticting our assumption that $J$ achieved lower risk in the worst case. Suppose $S(J_U,\theta)$ is maximised in some state $\theta^1$ where there are some $d,d'\in D$ such that $S(J_d,\theta^1)>S(J_{d'},\theta^1)$. Then there are most $|D|/2$ decisions where $S(J_d,\theta^1)$ is greater than the median of $A=\{S(J_d,\theta^1)|d\in D\}$ and at least one such decision, and at least $|D|/2$ decisions such that $\mu_{\theta^1} J(d)$ is greater than or equal to the median of $B=\{\mu_{\theta^1} J(d)|d\in D\}$, with at least one strictly greater. Thus there is an invertible map $f:D\to D$ such that $f(A)\subset B$. But then there is some $\theta^2$ such that $S(J_d,\theta^1)=S(J_{f(d)},\theta^2)$ for all $d\in D$ and thus $S(J,\theta^2)> S(J_U,\theta^2) = S(J_U,\theta^1)$ contradicting our assumption that $J$ was better by the minimax rule than $J_U$.
\end{proof}

\begin{corollary}
If the risk of the uniform strategy is maximised in some state $\theta^*$ such that $S(J_d,\theta^*)>S(J_{d'},\theta^*)$ for some $d,d'$, then the uniform strategy is strictly better than any nonuniform strategy.
\end{corollary}

Thus for a causal theory to support nontrivial results, we require for Bayes rules a prior $xi$ such that $(\xi\otimes \delta_d)T$ depends on $d$, or for the minimax rule that the \emph{set} of distributions mapped by the theory $\mathscr{T}_d:=\{T_{\theta,d}|\theta\in\Theta\}$ depends on the decision $d$. We will say that such theories/priors exhibit a \emph{decision bias}. 

From one point of view, this result might be expected: if we believe
\begin{itemize}
\item Any possible consequence of $d_1$ might equally be a consequence of $d_2$ and vise versa
\item Any data we encounter is equally consistent with $d_1$ having some set of consequences or with $d_2$ having that same set of consequences
\end{itemize}
Then we ought to be indifferent between $d_1$ or $d_2$ whatever data we see.

No causes in, no causes out (NCINCO) implies that some common principles commonly applied to causal inference, in isolation, can only yield trivial theoreis. Without any notion of intervention, causal inference based solely on principles such as the invariance of conditionals \citet{arjovsky_invariant_2019,peters_causal_2016}, a preference for low complexity consequences \citet{lemeire_replacing_2013} or faithfulness \citet{spirtes_causation_1993} would yield triviality. As discussed in Section \ref{sec:counterfactuals}, we also require assumptions on the effects of decisions to to get a causal theory from a potential outcomes model.

\end{document}
