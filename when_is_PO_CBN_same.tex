\section{Equivalence of causal problems}

Under what conditions could we consider a consequence consistent theory $\mathscr{T}^{cc}$ associated with some distribution over potential outcomes to be ``equivalent'' to some causal theory $\mathscr{T}^{\mathcal{G}}$ associated with a CBN $\mathcal{G}$ or vise versa? 

The question of whether $\mathscr{T}^{\mathcal{G}}$ is consequence consistent with respect to some distribution over potential outcomes is easy to answer in the affirmative as consequence consistency is a trivial requirement if we choose potential outcomes $\RV{X}_y:=\RV{X}$ for all $y\in D$.

The question of whether a consequence consistent theory $\mathscr{T}^{cc}$ can in general be represented by a Causal Bayesian Network is then also straightforwardly answered in the negative, as conditions 2 and 3 of Definition \ref{def:CBN} are in general non-trivial (condition 1 is trivial given a fully connected DAG $\mathcal{G}$).

The trivial potential outcome $\RV{X}_y=\RV{X}$ clashes with the informal idea that a potential outcome represents the value $\RV{X}$ would have taken had action $y$ been taken - we might expect, for example, if $\delta_y\kappa F_{\RV{X}}\neq \mu F_{\RV{X}}$ then $\RV{X}$ would at least sometimes take a different value if the action $y$ is taken than if it is not. 

We might tentatively propose a more extensive set of assumptions to characterise a ``Potential Outcomes'' theory, which we will write $\mathscr{T}^{po}$.

\begin{definition}[Potential Outcomes Causal Theory]
A causal theory $\mathscr{T}^{po}$ is a ``Potential Outcomes'' theory with respect to random variable $\RV{X}:E\to X$ and potential outcome variable $\RV{X}_i:E\to X$, $i\in D$ if for every $(\mu,\kappa)\in \mathscr{T}$, $\kappa$ is consequence consistent (Eq. \ref{eq:oc_consist}) and 
\begin{align}
    \mu F_{\RV{X}_i} = \delta_i\kappa F_{\RV{X}_i}
\end{align}
\end{definition}

\todo[inline]{If we consider only joint distributions over potential outcomes, a PO causal theory associates a unique consequence with each distribution}

Note that the condition of consistency \citep{richardson2013single}, which is a very standard condition in the Potential Outcomes literature, is:
\begin{align}
    \mu_{|\{\RV{X}_i,\RV{Z}\}} \RV{F}_{\RV{X}} (w;A) = \delta_{\RV{X}_i(w)} (A) \qquad w\in \RV{Z}^{-1}(i)
\end{align}
Where the random variable $\RV{Z}$ is a variable that is informally understood to be ``intervenable'' in a similar manner to intervention in Causal Bayesian Networks. A Potential Outcomes Causal Theory invokes a very general notion of Potential Outcomes where such intervenable variables may not exist, and so consistency may not be a sensible notion.

We can specify causal theories with a CBN $\mathcal{G}$ that are not potential outcomes causal theories. Consider the graph $\RV{X}$ (with a single node and no edges). By condition 2 of Definition \ref{def:CBN}, the consequences in $\mathscr{T}^{\mathcal{G}}$ will all yield $\RV{X}$ distributed as a delta function for certain decisions. However, in general $\mathscr{T}^{\mathcal{G}}$ will contain distributions on the observation space $E$ for which no variable is distributed according to a delta function. $\mathscr{T}^{\mathcal{G}}$ therefore cannot be a Potential Outcomes Causal Theory. We will outline below how a Potential Outcomes theory is not, in general, a theory associated with any CBN $\mathcal{G}$.

Rather than demand that we can represent the same theory with a CBN and with PO, we might ask instead if a problem featuring a PO theory can in general be reduced to a problem featuring a CBN theory and vise versa. This is in keeping with our approach that a CSDP represents at a high level a two person game and the latter determines the decision-relevant aspects of the problem.


\begin{definition}[Potential Outcomes CSDP]
A CSDP $\langle (\mathscr{T},(E,\mathcal{E}),\RV{X}),(D,\mathcal{D}), (U,(F,\mathcal{F})\rangle$ is a \emph{Potential Outcomes CSDP} (POCSDP) if $E=F$, $D$ is denumerable and there exists a set of potential outcome variables $\RV{X}_i:E\to X$, $i\in D$ with respect to which $\mathscr{T}$ is a Potential Outcomes causal theory.
\end{definition}


\begin{definition}[CBN CSDP]
A CSDP $\langle (\mathscr{T},(E,\mathcal{E}),\RV{X}),(D,\mathcal{D}), (U,(F,\mathcal{F})\rangle$ is a \emph{Causal Bayesian Network CSDP} (CBNCSDP) with respect to some finite DAG $\mathcal{G}=(V,W)$ if $E=F$ and $\mathscr{T}$ is the theory induced by $\mathcal{G}$ \todo{(...and all the other stuff you need).}

Theorem \ref{th:red_to_PO} shows that, supposing $D$ is denumerable, every CSDP can be reduced to a PO CSDP. For denumerable $D$, then, it suffices to show that conditions 1-3 of Definition \ref{def:CBN} are nontrivial. Take some CSDP $\alpha=\langle (\mathscr{T},E,\RV{X}),D,(U,E) \rangle$ and suppose there is no $(\kappa,\mu)\in \mathscr{T}$, $y\in D$, $z\in X$ such that $\delta_y\kappa F_{\RV{X}} (A) = \delta_{z} (A)$. Then it is straightforward to see that $\alpha$ cannot satisfy condition 3 of Definition \ref{def:CBN}. Suppose that there is no $(\kappa,\mu)\in \mathscr{T}$, $y\in D$ such that $\delta_y \kappa = \mu$; it is then straightforward that conditions 1 and 2 of Definition \ref{def:CBN} cannot be simultaneously satisfied. \todo[inline]{In both cases it is straightforward to posit generalised utilities such that $\alpha$ cannot be reduced.} 

Lifting condition 2 from the definition of a CBN yields CBNs with \emph{generalized interventions}. \todo[inline]{I \emph{strongly suspect} this corresponds to the class of influence diagrams of \citep{dawid_beware_2010}}. Because conditions 1+2 are nontrivial, there exist POCSDPs that cannot be reduced to CSDPs based on CBNs with generalised interventions. Lifting conditions 2 and 3 yields a causal theory where we require only that the distributions given by every consequence $\kappa$ are compatible with some DAG $\mathcal{G}$, which we will call an \emph{independence-only CBN} \todo[inline]{I \emph{strongly suspect} this is closely related to the notion of Extended Conditional Independence of \citep{dawid_decision-theoretic_2012}}. Condition 1 of Definition \ref{def:CBN} can always be satisfied by choosing a graph $\mathcal{G}$ that is fully connected, so lifting conditions 2 and 3 is sufficient to ensure that every POCSDP can be reduced to a CSDP featuring an independence-only CBN, and in fact an independence-only CBN can represent every PO causal theory. \todo[inline]{The single world intervention graphs of \citet{richardson2013single} are DAGs that represent independences among distributions over potential outcome variables. They might be interpretable as POCSDPs.}

\todo[inline]{The generalised versions of CBNs yield theories that generally associate multiple consequences with each given distribution. However a generalized CBN still yields a unique causal theory}

% ,  and for each node $V^k\in V$, $k\in[N]$ there exist $M$ random variables $\RV{X}^k_j:E\to X^k$, $j\in [M]$ such that $\RV{X}=\splitter{0.13}(\otimes_{k\in [N], j\in [M]} \RV{X}^k_j)$, for each $(\kappa,\mu)\in\mathscr{T}$, $i,j\in [M]$ we have 
% \begin{enumerate}
%     \item $\mu F_{\splitter{0.13}(\otimes_{k\in[N]} \RV{X}^k_i)} = \mu F_{\splitter{0.13}(\otimes_{k\in[N]} \RV{X}^k_j)}$ and $\kappa F_{\splitter{0.13}(\otimes_{k\in[N]} \RV{X}^k_i)} = \kappa F_{\splitter{0.13}(\otimes_{k\in[N]} \RV{X}^k_j)}$
%     \item $\kappa F_{\splitter{0.13}(\otimes_{k\in[N]} \RV{X}^k_i)} = \mathrm{CBN}(G,\mu F_{\splitter{0.13}(\otimes_{k\in[N]} \RV{X}^k_i)})$ with respect to the random variables $\RV{D}^k:D\to \{*\}\cup X^k$ for $k\in [N]$.
% \end{enumerate}

% 1) is the requirement that $\RV{X}$ is a sequence of IID random variables, appropriately generalised to causal theories and 2) is the requirement that the theory $\mathscr{T}$ is the theory induced by the CBN $\mathcal{G}$.
\end{definition}

\begin{theorem}[Reduction to PO]\label{th:red_to_PO}
A CSDP $\alpha=\langle (\mathscr{T},E,\RV{X}),D,(U,E) \rangle$ where $D$ is denumerable can be reduced to a PO CSDP.
\end{theorem}

\begin{proof}
Suppose $D=[M]$ or $D=\mathbb{N}^+$. Take $E' = E\times E^D$ and for $i\in D\cup\{0\}$, $x:=(x_0,x_1,...)\in E'$ define the projection $\RV{P}_i(x_0,x_1,...) := x_i$ and the potential outcome variable $\RV{X}_i:=\RV{X}\circ \RV{P}_i$. 

Take a map $f$ from $\mathscr{T}$ to causal states on $E'$ such that, letting $(\kappa' F_{\RV{X}},\mu'):=f(\kappa,\mu)$, for all $y\in D$ and $A_0,A_1, ...\in\mathcal{E}$:
\begin{align}
    \mu^{po}(A_1\times...) &:= \prod_{y'\in D} \delta_{y'} \kappa (A_{y'})\\
    \kappa' (y; A_0\times A_1\times...) &:= \int_{A_1\times ...} \delta_{x_y} (A_0) \mu^{po}(dx) \\
    &= \prod_{y'\in D\setminus\{y\}} \delta_{y'} \kappa (A_{y'}) \int_{A_y} \delta_{x_y} (A_0) \delta_y \kappa(dx_y) \label{eq:PO_kappap_def}\\
    \mu' ( A_0\times A_1\times...) &:= \prod_{y'\in D\setminus\{y\}} \delta_{y'} \kappa (A_{y'}) \int_{A_y} \delta_{x_y} (A_0) \mu(dx_y) \label{eq:consist}
\end{align}
It can be verified that $\kappa'$ is a Markov kernel.

Note that by the definition of conditional probability, for $A,B\in \mathcal{X}$, $\int_{\RV{X}_y^{-1}(A)} (\delta_y \kappa')_{|\RV{X}_y} F_{\RV{X}_0} (x;B) \delta_y\kappa' (dx)=\delta_y \kappa' \splitter{0.13}(F_{\RV{X}_0}\otimes F_{\RV{X}_y}) (A, B)$. Thus by \ref{eq:PO_kappap_def}, $\delta_x (A)$ is a version of $(\delta_y \kappa')_{|\RV{X}_y} F_{\RV{X}_0} (x;A)$, so $\kappa'$ is consequence consistent.

Furthermore, $\mu' F_{\RV{X}_y}=\delta_y \kappa F_{\RV{X}}=\delta_y \kappa' F_{\RV{X}_y}$ for $y\geq 1$ . Therefore defining $\mathscr{T}'$ to be the image of $\mathscr{T}$ under $f$, we can see that $\mathscr{T}'$ is a PO causal theory with respect to ``observable'' $\RV{X}_0$ and ``potential outcomes'' $\RV{X}_y$, $y\in D$.

For $A\in \mathcal{E}$:
\begin{align}
    \kappa' F_{\RV{P}_0} (y;A) &= \int_E \delta_{z} (A) \delta_y \kappa (dz)\\
                               &= \int_{A} \kappa(y;dz)\\
                               &= \kappa (y;A)
\end{align}


For all $B\in\mathscr{X}$
\begin{align}
    \mu' F_{\RV{X}_0} (B) &= \int_E \delta_{z} (\RV{X}^{-1}(B)) \mu (dz)\\
                               &= \mu F_{\RV{X}} (B)
\end{align}

For all $J\in \mathscr{J}$ we have
\begin{align}
    U(\mu F_{\RV{X}} J\splitter{0.13}(I_{(D)}\otimes \kappa) &= U(\mu' F_{\RV{X}_0} J\splitter{0.13}(I_{(D)}\otimes \kappa' F_{\RV{P}_0}))\\
\end{align}

Therefore, given the PO CSDP $\beta=\langle (\mathscr{T}',E',\RV{X}_0),D,(U,E) \rangle$, for all $J\in \mathscr{J}$, $R^\alpha(J,\kappa,\mu)=R^\beta(J,f(\kappa,\mu))$. Thus $\beta$  is a reduction of $\alpha$ witnessed by $f$.
\end{proof}

\begin{corollary}
A CBN CSDP for which $D$ is a denumerable set can be reduced to a PO CSDP.
\end{corollary}

