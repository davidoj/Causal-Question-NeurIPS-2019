
%!TEX root = main.tex

\todo[inline]{I have too many adjectives here. I think I should either go with \emph{consequence maps} and \emph{stochastic consequence spaces} or \emph{subjunctive probabilities} and \emph{subjunctive probability spaces}. The latter is more precise, but also harder to parse before you read the definition.}

\section{Probability spaces and subjunctive probability spaces}

In a very broad sense, a decision maker - whether they be man or machine - is someone or something that takes some kind of provocation - data, problem specification, circumstances - and chooses a decision. Were we considering beings of a suitable level of divinity, we might doubt whether all the things they might choose could be contained in a set, but it is a basic property of mortal or mechanical decision makers that there is some set $D$ of things that they might ultimately choose. Conversely, there is a set $D$ such that every decision that a decision maker might choose is an element of $D$.

That is, given some set $A$ of provocations and $D$ of decisions, a decision maker implements a map $A\to D$. Another distinguishing feature of a decision maker concerns the internal details of how she implements this map. We take the view that, in order to be considered a decision maker, she must implement this map by comparing the available decisions in light of the given information $a\in A$. Abstractly, she posesses some higher order function $f:A\to E^D$ which, broadly speaking, takes some piece of information $a\in A$ and returns a function $f_a$ which determines the result of each $d\in D$ given the information $a$. A second higher order function $\mathrm{ch}:E^D\to D$ considers each decision $d\in D$ along with its result $f_a(d)\in E$ and chooses one according to some criterion. Our decision maker implements $\mathrm{ch}\circ f$.

Concretely, our decision maker might be responsible for producing a binary classifier that takes some piece of data $X$ and returns a class in $\{0,1\}$, leaving us with $D=\{0,1\}^X$. The given information is a set of $n$ labeled examples $A=X\times\{0,1\}^n$, and $f$ might be an \emph{empirical risk} functional that takes data $a\in A$ and a possible classifier $d\in D$ and returns the number of misclassified examples when $d$ is applied to $A$. In such a case, $\mathrm{ch}$ might be the functional given by $\argmin$ with some means of breaking ties.

Another type of decision maker is one that tries to affect the world in some way. Suppose we have some set $E$ that represents possible configurations of the world and a utility function $u:E\to \mathbb{R}$ that rates the desirability of any given state of the world. A decision maker that tries to affect the world is one whose choice function chooses the ``most preferred'' state of the world. This might correspond to a choice function given by utility maximisation $\mathrm{ch}:g\mapsto \argmax_{d\in D}u(g(d))$, for example. For example, consider a robot that only wants to switch a light on -- for this robot, we might take relevant states of the world to be $E=\{\mathrm{off},\mathrm{on}\}$, the utility to be $u:\mathrm{on}\mapsto 1$ and $\mathrm{off}\mapsto 0$, $D$ to be sequences of signals it might send to each of its actuators, the background information $A=\{\text{switch up}\mapsto\text{light on},\text{switch down}\mapsto \text{light on}\}$ stipulates the relationship between a switch and the light and the function $f:a\mapsto a\circ f_0$ takes the prior knowledge of whether a motor sequence $d\in D$ leaves the switch in the up or down position and then applies the specification $a$ to this knowledge. This robots choice function could be given by utility maximisation.

As a general principle, decision makers implement some function $h:A\to D$ which factorises as $\mathrm{ch}\circ f$ for some choice rule $\mathrm{ch}$ and some results model $f$. We take this to be a general model of something that chooses a decision by comparing available options. We are particularly interested in decision makers that try to affect the world, in particular those that use the principle of \emph{expected utility maximisation} as their choice rule. Expected utility maximisation is a very standard rule for choosing preferred states of the world subject to some uncertainty, and we will simply accept it at the outset to facilitate a thorough exploration of the results model $f$.

Expected utility maximisation (henceforth ``EU'') necessitates that the class of results models $f$ have a particular signature. The EU choice rule compares a set of probability distributions over states of the world $E$ and returns a decision associated with the preferred probability distribution. Thus, denoting by $\Delta(\mathcal{E})$ the set of probability distribution over $E$ (now equipped with $\sigma$-algebra $\mathcal{E}$), $f$ must have the signature $f:A\to \Delta(\mathcal{E})^D$. In particular $f$ returns stochastic functions of the type $D\to \Delta(\mathcal{E})$.

So far we have not made any structural assumptions about the set $D$. One such assumption that we do want to make is that stochastic decisions are possible. That is, if we can choose $d_1\in D$ and we can choose $d_2 \in D$, then it is also possible to choose $d_1$ with probability $\alpha$ and $d_2$ with probability $(1-\alpha)$ for $0\leq \alpha \leq 1$. This is easy enough to do - we simply have to assert that there is some $d_{(\alpha,1,2)}\in D$ such that $d_{(\alpha,1,2)}$ corresponds to the choice ``$d_1$ with probability $\alpha$ and $d_2$ with probability $(1-\alpha)$''. We would additionally like our results model $f$ to have the property that, for all $a\in A$, 
\begin{align}
	f_a(d_{(\alpha,1,2)}) = \alpha f_a(d_1) + (1-\alpha) f_a(d_2) \label{eq:convexity_or_something}
\end{align}

To strengthen this notion, suppose $(D,\mathcal{D})$ is measurable. Recall that for all $d\in D$, $f_a(d)$ is a probability measure on $E$. For $G\in \mathcal{E}$, write $f_a(d;G)$ for the probability measure $f_a(d)$ evaluated at $G$, and $f_a(\cdot;G)$ for the map $d\mapsto f_a(d;G)$. Then we will call a result map $f$ regular if for all $a\in A$, $G\in \mathcal{E}$ and all $\gamma\in \Delta(\mathcal{D})$ $f_a$, $f_a(\cdot;G)$ is measurable and there exists some $d_\gamma$ such that
\begin{align}
	f_a(d_\gamma;G) = \int_D f_a(d';G) d\gamma(d') \label{eq:strong_convexit_or_something}
\end{align}

We can define $f'_a:\Delta(\mathcal{D})\to \Delta(\mathcal{E})$ by $f^*_a(\gamma;G) = f_a(d_\gamma;G)$. Note that for every $d\in D$ we have $\delta_d\in \Delta(\mathcal{D})$ such that $f_a(d;G) = f^*_a(\delta_d;G)$. Thus we can model the problem as one of selection of a probability measure from $\Delta(\mathcal{D})$ instead of a decision from $d$, and probability measures are related to consequences by the map $f'$. The actual decision chosen can be found by the map $h:\gamma\mapsto d_\gamma$.

Note that by the assumption of measurability, we have that $f_a$ is a Markov kernel for all $a\in A$. If our basic decision set $D$ is a polish space (a separable, completely metrizable topological space) and it's Borel sets form the $\sigma$-algebra $\mathcal{D}$, then the set $\Delta(\mathcal{D})$ is convex closed and can therefore be described as the closure of the convex hull of its extreme points. Denote by $D'$ the extreme points of $\Delta(\mathcal{D})$, and suppose $D'$ is countable. Then $\Delta(\mathcal{D}')$ is equivalent to $\Delta(\mathcal{D})$. Under these assumptions, which we will hold to, we can describe the set of decisions as the set of probability distributions over an underlying discrete set $D'$. Furthermore, for an element $d\in D'$, we can define $f':A\times D\to \Delta(\mathcal{E})$ such that

\begin{align}
	f'_a(d;G) = f_a(d';G)
\end{align}

Then for all $\gamma\in \Delta(\mathcal{D}')$ we have

\begin{align}
	f_a(d_\gamma;G) = \int_D f'_a(d';G) d\gamma(d') \label{eq:basic_decisions}
\end{align}

\todo[inline]{I don't know that $D'$ has to be countable, the assumption that $D$ is standard Borel and hence $D'$ exists may be enough. I also don't really understand what the assumption of standard Borelness means, only that it can be satisfied by letting the set of decisions be the set of probability distributions over a countable set}


As we are chiefly interested in the set of elementary decisions $D'$, we will henceforth simply use $(D,\mathcal{D})$ for this set. Similarly, we will simply use $f$ to refer to the map given by $f'$ in Equation \ref{eq:basic_decisions}. Note that at this point $f_a$ is a Markov kernel $D\to \Delta(\mathcal{E})$, hence we can use product notation.

Assumption \ref{eq:strong_convexit_or_something} still permits some unexpected behaviour with respect to randomised decisions. Consider the following variations of a problem:

\begin{example}[Fighting children problem]
A decision maker is mediating a dispute over a toy between two children. If he chooses to give the toy to the first child, the second child will cry, and if he chooses to give the toy to the second child, the first child will cry (the underlying decision set $D'=\{\text{give to first child}, \text{give to second child}\}$. If he chooses a random procedure:

\paragraph{Version 1} These children strongly object to randomness, and both will cry no matter the outcome
\paragraph{Version 2} These are children with a strong insistence on a version of procedural fairness which no doubt make a great deal of sense to them. In particular, if a procedure is used which gives the first child a chance $p$ of receiving the toy, the first child will cry with probability $p$ irrespective of the outcome. The second child, meanwhile, will cry iff the first child does not
\paragraph{Version 3} Whether or not the children cry depends only on whether or not they receive the toy
\end{example}

In version 1, assumption \ref{eq:strong_convexit_or_something} is violated - the outcome when choosing a random procedure is not simply a mixture of the outcomes of each deterministic decision. In version 3, assumption \ref{eq:strong_convexit_or_something} holds, and the results depend only on the outcome of randomisation, as is desired. In version 2, assumption \ref{eq:strong_convexit_or_something} \emph{also} holds - one child cries iff the other does not, and each child cries with the same probability as the probability that they receive the toy. However, randomising and then giving the first child the toy is not the same as simply giving the first child the toy. To discount the possibility of this kind of behaviour, we need a stronger version of \ref{eq:strong_convexit_or_something}. 

First, we add the assumption that one consequence of choosing $d\in D$ is that $d$ was chosen. That is, we assume the sample space can be written as $E\times D$ and, defining the random variable $\RV{D}:E\times D\to D$, $\RV{D}:(e,d)\mapsto d$. Suppose

\begin{align}
	(f_{(a,d)})_{\RV{D}} = \delta_d \label{eq:assumption_decision_forsure}
\end{align}

We find in \cite{joyce_why_2000} the assumption of a ``supposition function'' $\mathrm{prob}(\cdot\|\cdot)$ that $\mathrm{prob}(C\|C)=1$ where $C$ is ``some distinguished condition''. Assumption \ref{eq:assumption_decision_forsure} expresses a similar idea - if we take a decision $d$, then that decision will have been taken for sure.

In addition, suppose for all $\gamma\in \Delta(\mathcal{D})$
\begin{align}
	(\gamma f_{a})_{|\RV{D}} = f_a \label{eq:consequence_is_conditional}
\end{align}

That is, for any stochastic decision $\gamma$, the consequence conditional on $\RV{D}$ is the consequence map $f_a$ itself. This assumption rules out both versions 1 and 2 of the fighting children problem above.


Note that this strengthening depends on the structural assumptions on $D$; in particular, it depends on the possibility of finding an ``elementary'' set of decisions $D'$. This is why it is introduced after assumption \ref{eq:strong_convexit_or_something}.





For our causal analysis we work with \emph{subjunctive probability spaces}, a generalisation of the more familiar probability spaces. The subjunctive mood is used to describe hypothetical or supposed states of the world\todo{Does this definition need to be here? I didn't know what this word meant before I looked it up}, and subjunctive probability spaces are models that ask for some hypothetical state and give us back a probability space. Subjunctive probability spaces are also different to \emph{conditional probability spaces} \citet{renyi_conditional_1956} as \emph{hypothesising} or \emph{supposing} (that is, those things described by the subjunctive mood) are different to \emph{conditioning}, which is more closely analogous to \emph{focussing your attention}.

We use subjunctive probability spaces because \emph{supposition} is a core part of decision problems, one we cannot get away from. In contrast, modelling supposition with conditional probability (which would be necessary if we were to insist on using probability spaces) adds additional structure to our models which isn't clearly warranted and is potentially confusing.

Any decision problem must involve the comparison of different decisions. This comparison takes the form
\begin{itemize}
	\item \emph{Suppose} I choose the first decision - then the consequence would be $X$
	\item \emph{Suppose} I choose the second decision - then the consequence would be $Y$
	\item Etc.
\end{itemize}

If we prefer $X$ to $Y$, then perhaps we should choose the first decision. We may be ambivalent as to what type of thing $X$ and $Y$ represent, how we ought to determine what values $X$ and $Y$ take or how we ought to determine what is preferable, but it is hard to do away with supposing that different decisions were taken and that these decisions come with their own consequences and still have what could reasonably be considered a decision problem. This process of supposition implicitly invokes a ``subjunctive function'' - we provide a hypothetical decision, and we are given a consequence of that hypothetical. Of particular interest are models that, for each hypothetical choice, return a probability distribution over space $\Omega$. Such models are called \emph{supposition functions} by \citet{joyce_why_2000}, but we will call them \emph{consequence maps} in this work. We consider this particular type of subjunctive function - from possible decisions to probability distributions - to be attractive because we consider probability to be a sound choice for modelling uncertainty and stochasticity.

Formally, a consequence map $\mathscr{C}$ is a stochastic function or \emph{Markov kernel} from $D\to \Delta(\Omega)$, where $\Delta(\Omega)$ represents the set of all probability distributions on $\Omega$. One might recall that, given two random variables $\RV{Y}:\Omega\to Y$ and $\RV{X}:\Omega \to X$ on some probability space $(\mathbb{P},\Omega,\mathcal{F})$, the probability of $\RV{Y}$ conditional on $\RV{X}$ is a Markov kernel $X\to \Delta(\mathcal{Y})$. It is possible, in general, to define a probability distribution on the expanded space $\Omega\times D$ such that $\mathscr{C}$ is a conditional probability. However, there are good reasons to keep the concepts of consequence maps and conditional probability separate. Firstly, there are technical issues such as the fact that it is not always possible to find a distribution on $\Omega\times D$ that yields $\mathscr{C}$ as a \emph{unique} conditional probability \citep{hajek_what_2003} (though this requires an uncountable set of possible decisions, which is not a problem we consider in this work). Secondly, it is not clear that that the way we ought to handle consequence maps is identical to the way we handle conditional probabilities. Consider an expanded set of options:

\begin{enumerate}
	\item Suppose I choose the first decision $d_1$ - then the consequence would be $P_1$
	\item Suppose I choose the second decision $d_2$ - then the consequence would be $P_2$
	\item Suppose I choose either the first or second decision $d_1$ or $d_2$ - then the consequence would be ???
	\item Suppose I choose $d_1$ with probability $0\leq q\leq 1$ and $d_2$ otherwise - then the consequence would be ???
\end{enumerate}

If we regard the consequence map as a conditional probability then it would be natural to consider the result of the third option to be a unique probability distribution obtained by conditioning on $\{d_1,d_2\}$, equal to  $\alpha P_1 + (1-\alpha)P_2$ for some fixed $0\leq \alpha\leq 1$. However, there is no obvious reason that these should be mixed in some fixed proportion $\alpha$ - it seems more appropriate to me to say that if $d_1$ or $d_2$ is chosen then the result will be $P_1$ or $P_2$.

On the other hand, the result of the fourth option seems like it should be given by $q P_1 + (1-q) P_2$, at least for most ordinary problems. If we were confronted with a mind reader who could tell the difference between us having chosen $d_1$ and us having randomised between $d_1$ and $d_2$ but come up with $d_1$ anyway then we might have reason to revise this assumption, but we will generally proceed under the assumption that such mind readers are absent. For any ambient probability measure, we can choose $q$ not to be equal to $\alpha$ (as defined in the previous paragraph), and so the result will not be equal to the ambient measure conditioned on $\{d_1,d_2\}$. 

In general, choosing the consequence map to be a conditional probability requires there to exist some joint distribution over the decision $\RV{D}$ and all other random variables in the problem. However, when we adopt hypotheses about what decisions we might make, we throw away key parts of this joint distribution (for example, we throw away the marginal distribution of $\RV{D}$) and replace it with whatever we want to suppose instead. Instead of adopting a full joint distribution and then throwing away some parts to get hold of the consequence map, as we would if we were working with a probability space, a subjunctive probability space only supplies those parts which we intend to keep - i.e. in only supplies the consequence map.
