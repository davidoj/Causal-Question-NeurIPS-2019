\section{Appendix: CBN is a causal theory}\label{app:cbn_ct}

\begin{theorem}
Given a graph $\mathcal{G}$, a measurable set $(E,\mathcal{E})$ and a decision set $D=\times_{i\in [N]} X^i\cup\{*\}$ with the $\sigma$-algebra $\mathcal{D}=\otimes_{i\in [N]} \sigma(\mathcal{X}^i\cup\{*\})$, let $P^x\in \Delta(\mathcal{E})$ be an interventional distribution compatible with $\mathcal{G}$ as defined in Definition \ref{def:CBN}. 

Then the map $\kappa_{\mathcal{G}}:D\to \Delta(\mathcal{X})$ given by $x\mapsto P^x$ is a Markov kernel.
\end{theorem}

\begin{proof}
For each pair $(\RV{X}^i,\PA{\mathcal{G}}{\RV{X}^i})$, there is a Markov kernel $\kappa_i:\{X^j|j\in \PA{\mathcal{G}}{i}\}\to \Delta(\mathcal{X}^i)$ that is a version of $P^*(\RV{X}^i|\PA{\mathcal{G}}{\RV{X}^i}$ (\cite{cinlar_probability_2011}). 

Let $\PA{\mathcal{G}}{X^i} = \{X^j|\RV{X}^j\in \PA{\mathcal{G}}{\RV{X}^i}\}$.

Consider $\kappa'_i:D\times \PA{\mathcal{G}}{X^i} \to \Delta(\mathcal{X}^i)$ given by \begin{align}
    \kappa'_i: (*,pa^i;A)&\mapsto \kappa_i(pa^i;A)&\label{eq:passive}\\
               (d,pa^i;A)&\mapsto \delta_{\RV{D}^i(d)}(A)&d\neq *\label{eq:npassive}
\end{align}

Clearly for every $(d,pa^i) \in \PA{\mathcal{G}}{X^i}\cup D$ the map $A\mapsto \kappa'_i(d,pa^i;A)$ is a probability distribution on $\mathcal{X}^i$. Fix $B\in\mathcal{X}_i$ and let $\kappa'^B_i=\kappa'_i(\cdot;B)$.

Then for any $A\in \mathcal{B}([0,1])$
\begin{align}
    [\kappa'^B_i]^{-1}(A) &= \{*\}\times[\kappa^B_i]^{-1}(A) &\text{if }0,1\not\in A\\
    &= \{*\}\times[\kappa^B_i]^{-1}(A)\cup \PA{\mathcal{G}}{X^i}\times A &\text{if }1\in A\land 0\not\in A\\
    &= \{*\}\times[\kappa^B_i]^{-1}(A)\cup \PA{\mathcal{G}}{X^i}\times A^C &\text{if }0\in A\land 1\not\in A\\
    &= \{*\}\times[\kappa^B_i]^{-1}(A)\cup \PA{\mathcal{G}}{X^i}\times X^i &\text{if }0\in A\land 1\in A
\end{align}
In every case, the result is clearly an element of $\otimes_{j\in \PA{\mathcal{G}}{X^i}} \mathcal{X}^j \otimes \sigma(\mathcal{X}^i\cup\{*\})$.

Without loss of generality, suppose the ordering $X^0,...,X^N$ is compatible with the graph $\mathcal{G}$. Then $\iota_{\mathcal{G}}:D\to \Delta(\mathcal{X})$ defined below is a Markov kernel.
\begin{align}
    \iota_{\mathcal{G}}:d\mapsto \int_{A^0} \kappa'_0(\RV{D}^0(d);dx^0) ... \int_{A^{N-1}} \kappa'_{N-1}(\RV{D}^{N-1}(d),x^{n-2};dx^{n-1}) \kappa'_N(\RV{D}^N(d),x^{n-1};A^N) \label{eq:bigproduct}
\end{align}

for $d\in D$, $(A^0,...,A^N)\in X^0\times...\times X^N$ and where $\RV{D}^i(d)$ projects the $i$-th element of $d$.

From Equation \ref{eq:bigproduct} and \ref{eq:passive} we can verify that, given some $i\in N$, if $\RV{D}^i(d)=\{*\}$ then $\iota_{\mathcal{G},\RV{X}^i|\PA{\mathcal{G}}{\RV{X}^i}}(d;\cdot) = \kappa_i$ and if $\RV{D}^i(d)\neq\{*\}$ then $\iota_{\mathcal{G},\RV{X}^i|\PA{\mathcal{G}}{\RV{X}^i}}(d;\cdot) = I_{\RV{D^i}(d)}$. These properties are equivalent to truncated factorisation, which is in turn equivalent to \ref{def:CBN}. Therefore $\iota_\mathcal{G}=\kappa_\mathcal{G}$.
\end{proof}

\section{Appendix: Counterfactuals and one-shot inference}

Recall that we have proposed making the connection between data, decisions and outcomes in two steps: firstly a causal theory relates data to possible consequences (the ``inference'' step), and secondly a consequence then relates decisions to outcomes (the ``control'' step). We could consider a generalised consequence $D\times E\to \mathscr{P}(\Delta(\mathcal{X}))$ that jointly performs the inference and control steps. We speculate that this generalisation provides an alternative connection between SCDPs and counterfactual reasoning; in particular, Nonparametric Structural Equation Models (NPSEMs) which are often considered appropriate tools for modelling counterfactual distributions (\cite{pearl_causality:_2009,richardson2013single}) can be seen as a special case of generalised consequences.

\begin{definition}[NPSEM]\label{def:NPSEM}
A non-parametric structural equation model (NPSEM) is a tuple $\langle \{\RV{X}^i, \RV{U}^i, f^i\}_{i\in[N]}, (D,\mathcal{D}), (E,\mathcal{E})\rangle$ where, for all $i\in N$, $\RV{X}^i:E\times D \to X^i$, $\RV{U}^i:E\to U^i$, $\mathscr{H}\subset\Delta(\mathcal{E}$ and $D=\times_{i\in[N]} X^i\cup\{*\}$ and $f^i:\times_{j<i} X^i\times U^i\to X^i$ are functions measurable with respect to the implied product sigma algebras. The $\RV{X}^i$ are given by

\begin{align}
    \RV{X}^i(e,d) = \begin{cases} f^i(\RV{X}^{<i}(e,d),\RV{U}^i(e)) &\RV{D}^i(d)=*\\ 
    \RV{D}^i(d)  &\RV{D}^i(d)\neq * \end{cases}
\end{align}

Where $\RV{X}^{<i}(e,d)=[\RV{X}^0(e,d),...\RV{X}^{i-1}(e,d)]$.
\end{definition}

Given an NPSEM $\mathscr{M}:\langle \{\RV{X}_i, \RV{U}_i, f_i\}_{i\in[N]}, (D,\mathcal{D}), (E,\mathcal{E})\rangle$ we can let $\RV{X}$ be the joint space of all the $\RV{X}_i$ and an NPSEM induces a measurable function $M:D\times E\to X$. In general there are many NPSEMs that induce the same measurable function in this manner. For example $f_0:e\mapsto e$ and $f_1:(x_0,e)\mapsto x_0$ induces the same function $E\to X_0\times X_1$ as $f_0:e\mapsto e$ and $f_1:(x_0,e)\mapsto e$. If consider all NPSEMs inducing the same function $M$ to be equivalent, then we can, abusing terminology somewhat, consider an NPSEM to be a measurable function $D\times E\to X$. A straightforward generalisation of this is a stochastic NPSEM $D\times E\to \Delta(\mathscr{X})$, and a set of stochastic NSPEMs induces a set valued stochastic map $D\times E\to \mathscr{P}(\Delta(\mathcal{X}))$.


