%!TEX root = main.tex

\section{Recoverability}

A natural assumption suggested by the notion of a CSDP is that of \emph{recoverability} - that a causal theory $\mathscr{T}:E\times D\rightarrowtriangle E$ permits some decision function that reproduces the distribution of the observed data. That is, we assume that for every $(\kappa_\theta,\mu_\theta):=\theta\in \mathscr{T}$ there exists $\gamma_{\theta}\in \Delta(\mathcal{D})$ such that
\begin{align}
	\gamma_{\theta}\kappa_\theta = \mu_\theta \label{eq:recoverability}
\end{align}

Suppose also that we have some $\kappa^*$ that, for all $\theta \in \mathscr{T}$, is a Bayesian inversion of $\gamma_\theta$ and $\kappa_\theta$; that is:

\begin{align}
\begin{tikzpicture}
	\path (0,0) node[dist] (gam) {$\gamma_\theta$}
	++(0.5,0) coordinate (copy0)
	+(0.5,0.5) node[kernel] (kap) {$\kappa_\theta$}
	+(0.5,-0.5) coordinate (ph)
	+(1,0.5) node (E) {$\RV{E}$}
	+(1,-0.5) node (D) {$\RV{D}$};
	\draw (gam) -- (copy0);
	\draw (copy0) to [bend left] (kap);
	\draw (copy0) to [bend right] (ph);
	\draw (kap) -- (E);
	\draw (ph) -- (D); 
\end{tikzpicture}= \begin{tikzpicture}
	\path (0,0) node[dist] (gam) {$\gamma_\theta$}
	++(0.6,0) node[kernel] (kap) {$\kappa_\theta$}
	++(0.5,0) coordinate (copy0)
	+(0.5,-0.5) node[kernel] (kapdag) {$\kappa^*$}
	+(0.5,0.5) coordinate (ph)
	+(1,0.5) node (E) {$\RV{E}$}
	+(1,-0.5) node (D) {$\RV{D}$};
	\draw (gam) -- (kap);
	\draw (kap) -- (copy0);
	\draw (copy0) to [bend right] (kapdag);
	\draw (copy0) to [bend left] (ph);
	\draw (kapdag) -- (D);
	\draw (ph) -- (E); 
\end{tikzpicture} \label{eq:kappa_BI}
\end{align}

A sufficient condition for the existence of such a $\kappa^*$ is the assumption that decisions correspond to \emph{variable setting} - that is, there is some variable $\RV{X}:E\to X$ such that for all $a\in D$, $\theta\in\mathscr{T}$ we have $\delta_a \kappa_\theta F_{\RV{X}} = \delta_a$ (such an assumption arises in graphical models as hard interventions, and in potential outcomes as ``potential-outcome identifiers''). Indeed $F_{\RV{X}}$ is in this case a candidate for $\kappa^*$. It is not necessary that $\kappa^*$ be deterministic, however - suppose every $\kappa$ ignores $D$. Then choose $\gamma_\theta=\gamma$ for arbitrary $\gamma\in \Delta(\mathcal{D})$ and it can be verified that $\kappa^*:b\mapsto \gamma$ satisfies \ref{eq:kappa_BI}. 

I believe a weaker sufficient condition for the existence of a universal $\kappa^*$ is that every $\kappa_\theta$ factorises as $\kappa_\theta = h \splitter{0.1}(\mathrm{Id}_F\otimes j_\theta)$ for some fixed $h:D\to \Delta(\mathcal{F})$, but I have not yet shown this.

We will proceed somewhat rashly: suppose that by defining $\gamma:\mathscr{T}\to \Delta(\mathcal{D})$, $\mu:\mathscr{T}\to \Delta(\mathcal{E})$ and $\kappa:\mathscr{T}\times D\to \Delta(\mathcal{E}$ by $\gamma:\theta\to \gamma_\theta$, $\mu:\theta\to \mu_\theta$ and $\kappa:(\theta,d)\to \kappa_\theta(d;\cdot)$ that all resulting objects are Markov kernels, and that $\mathscr{T}$ is a standard measurable space.

By previous assumptions, we have the following properties:
\begin{align}
\begin{tikzpicture}
	\path (0,0) coordinate (A)
	++ (0.5,0) node[kernel] (B) {$\mu$}
	++ (0.5,0) coordinate (C);
	\draw (A) -- (B);
	\draw (B) -- (C);
\end{tikzpicture}
&=
\begin{tikzpicture}
	\path (0,0) coordinate (A)
	++ (0.3,0) coordinate (copy0)
	++(0.5,0) node[kernel] (gam) {$\gamma$}
	+ (0,0.3) coordinate (B)
	++(0.5,0) node[kernel] (kap) {$\kappa$}
	++(0.5,0) coordinate (C);
	\draw (A) -- (copy0);
	\draw (copy0) to [bend left] (B);
	\draw (copy0) -- (gam);
	\draw (B) to [bend left] (kap.west);
	\draw (gam) -- (kap);
	\draw (kap) -- (C);
\end{tikzpicture}\label{eq:recoverabilityd}\\
\begin{tikzpicture}
	\path (0,0) coordinate (A)
	++ (0.3,0) coordinate (copy0)
	++ (0.5,0) node[kernel] (gam) {$\gamma$}
	++ (0.5,0) coordinate (copy1)
	+ (0.5,-0.5) node[kernel] (kap) {$\kappa$}
	+ (0.5,0.5) coordinate (B)
	+ (1,-0.5) node (E) {$\RV{E}$}
	+ (1,0.5) node (D) {$\RV{D}$};
	\draw (A) -- (copy0);
	\draw (copy0) -- (gam);
	\draw (copy0) to [bend right] (kap);
	\draw (gam) -- (copy1);
	\draw (copy1) to [bend left] (B);
	\draw (copy1) to [bend right] (kap);
	\draw (kap) to (E);
	\draw (B) to (D);
\end{tikzpicture}
&=
\begin{tikzpicture}
	\path (0,0) coordinate (A)
	++ (0.3,0) coordinate (copy0)
	++ (0.5,0) node[kernel] (gam) {$\gamma$}
	++ (0.5,0) node[kernel] (kap) {$\kappa$}
	++ (0.5,0) coordinate (copy1)
	+ (0.5,0.5) node[kernel] (kapdag) {$\kappa^*$}
	+ (0.5,-0.5) coordinate (B)
	+ (1,-0.5) node (E) {$\RV{E}$}
	+ (1,0.5) node (D) {$\RV{D}$};
	\draw (A) -- (copy0);
	\draw (copy0) -- (gam);
	\draw (copy0) to [bend right=40] (kap);
	\draw (gam) -- (kap);
	\draw (kap) -- (copy1);
	\draw (copy1) to [bend right] (B);
	\draw (copy1) to [bend left] (kapdag);
	\draw (kapdag) to (D);
	\draw (B) to (E);
\end{tikzpicture} \label{eq:general_kapdag}\\
&= 
\begin{tikzpicture}
	\path (0,0) coordinate (A)
	++ (0.3,0) coordinate (copy0)
	++ (0.5,0) node[kernel] (mu) {$\mu$}
	++ (0.5,0) coordinate (copy1)
	+ (0.5,0.5) node[kernel] (kapdag) {$\kappa^*$}
	+ (0.5,-0.5) coordinate (B)
	+ (1,-0.5) node (E) {$\RV{E}$}
	+ (1,0.5) node (D) {$\RV{D}$};
	\draw (A) -- (copy0);
	\draw (copy0) -- (mu);
	\draw (mu) -- (copy1);
	\draw (copy1) to [bend right] (B);
	\draw (copy1) to [bend left] (kapdag);
	\draw (kapdag) to (D);
	\draw (B) to (E);
\end{tikzpicture}\label{eq:gamkap_2_mu}
\end{align}
From \ref{eq:general_kapdag} we also have

\begin{align}
\begin{tikzpicture}
	\path (0,0) coordinate (A)
	++ (0.3,0) coordinate (copy0)
	++ (0.5,0) node[kernel] (gam) {$\gamma$}
	++ (0.5,0) coordinate (copy1)
	+ (0.5,-0.5) node[kernel] (kap) {$\kappa$}
	+ (0.5,0.5) coordinate (B)
	+ (1.1,-0.5) node (E) {}
	+ (1.1,0.5) node (D) {$\RV{D}$};
	\draw (A) -- (copy0);
	\draw (copy0) -- (gam);
	\draw (copy0) to [bend right] (kap);
	\draw (gam) -- (copy1);
	\draw (copy1) to [bend left] (B);
	\draw (copy1) to [bend right] (kap);
	\draw[-{Rays [n=8]}] (kap) to (E);
	\draw (B) to (D);
\end{tikzpicture}
&=
\begin{tikzpicture}
	\path (0,0) coordinate (A)
	++ (0.3,0) coordinate (copy0)
	++ (0.5,0) node[kernel] (gam) {$\gamma$}
	++ (0.5,0) node[kernel] (kap) {$\kappa$}
	++ (0.5,0) coordinate (copy1)
	+ (0.5,0.5) node[kernel] (kapdag) {$\kappa^*$}
	+ (0.5,-0.5) coordinate (B)
	+ (1.1,-0.5) node (E) {}
	+ (1.1,0.5) node (D) {$\RV{D}$};
	\draw (A) -- (copy0);
	\draw (copy0) -- (gam);
	\draw (copy0) to [bend right=40] (kap);
	\draw (gam) -- (kap);
	\draw (kap) -- (copy1);
	\draw (copy1) to [bend right] (B);
	\draw (copy1) to [bend left] (kapdag);
	\draw (kapdag) to (D);
	\draw[-{Rays [n=8]}] (B) to (E);
\end{tikzpicture} \\
\begin{tikzpicture}
	\path (0,0) coordinate (A)
	++ (0.3,0) coordinate (copy0)
	++ (0.5,0) node[kernel] (gam) {$\gamma$}
	+ (0.5,0) node (D) {$\RV{D}$};
	\draw (A) -- (copy0);
	\draw (copy0) -- (gam);
	\draw (gam) -- (D);
\end{tikzpicture}
&=
\begin{tikzpicture}
	\path (0,0) coordinate (A)
	++ (0.5,0) node[kernel] (mu) {$\mu$}
	+ (0.5,0) node[kernel] (kapdag) {$\kappa^*$}
	+ (1.1,0) node (D) {$\RV{D}$};
	\draw (A) -- (mu);
	\draw (mu) --  (kapdag);
	\draw (kapdag) to (D);
\end{tikzpicture} \label{eq:kappa_cancel}
\end{align}
Where \ref{eq:kappa_cancel} follows from \ref{eq:recoverability}.

Suppose we have that $\mu$ is deterministic, relieving us of having to deal with any issues regarding inferring the observed distribution from a finite sample (e.g. we are in the world of classical physics). $\mu$ therefore has a left inverse $^\dagger\mu$. 

A corollary of Lemma \ref{th:rightleft_inverse} is that left inverses have the following property:

\begin{align}
\begin{tikzpicture}
 \path (0,0) node (A) {}
 ++(0.5,0) node[kernel] (C) {$\;^\dagger A\;$}
 ++(0.5,0) coordinate (D)
 +(0.5,-0.5) node[kernel] (X) {$A$}
 +(1.2,0.5) node (Y) {}
 +(1.2,-0.5) node (Z) {};
 \draw (A)--(C);
 \draw (C) -- (D);
 \draw (D) to [bend right] (X);
 \draw (X) -- (Z);
 \draw (D) to [bend left] (Y);
\end{tikzpicture} = \begin{tikzpicture}
 \path (0,0) node (A) {}
 ++(0.5,0) coordinate (D)
 +(0.5,0.5) node[kernel] (X) {$\;^\dagger A\;$}
 +(1,-0.5) node (Y) {}
 +(1,0.5) node (Z) {};
 \draw (A)--(D);
 \draw (D) to [bend left] (X);
 \draw (X) -- (Z);
 \draw (D) to [bend right] (Y);
\end{tikzpicture}\label{eq:rightinverse_commute}
\end{align}

And \citet{fong_causal_2013} has shown that for deterministic $A$ we have

\begin{align}
\begin{tikzpicture}
 \path (0,0) node (A) {}
 ++(0.5,0) node[kernel] (C) {$A$}
 ++(0.5,0) coordinate (D)
 +(1.2,0.5) node (Y) {}
 +(1.2,-0.5) node (Z) {};
 \draw (A)--(C);
 \draw (C) -- (D);
 \draw (D) to [bend right] (Z);
  \draw (D) to [bend left] (Y);
\end{tikzpicture} = \begin{tikzpicture}
 \path (0,0) node (A) {}
 ++(0.5,0) coordinate (D)
 +(0.5,0.5) node[kernel] (A1) {$A$}
 +(0.5,-0.5) node[kernel] (A2) {$A$}
 +(1,-0.5) node (Y) {}
 +(1,0.5) node (Z) {};
 \draw (A)--(D);
 \draw (D) to [bend right] (A2);
 \draw (A2) -- (Y);
 \draw (D) to [bend left] (A1);
 \draw (A1) -- (Z);
\end{tikzpicture}\label{eq:det_commute}
\end{align}

We then have

\begin{align}
\begin{tikzpicture}
 \path (0,0) node (A) {}
 ++(0.3,0) coordinate (copy0)
 ++(0.5,0) node[kernel] (mu) {$\mu$}
 ++(1,0) node[kernel] (kapdag) {$\kappa^*$}
 ++(0.4,0) coordinate (copy1)
 ++(0.5,0) node[kernel] (kap) {$\kappa$}
 +(1,0) node (E) {}
 +(1,-0.5) node (D) {}
 +(1,0.5) node (T) {};
 \draw (A)--(mu) -- (kapdag) -- (kap) -- (E);
 \draw (copy0) to [bend left] (kap) (copy0) to [bend left] (T);
 \draw (copy1) to [bend right] (D);
\end{tikzpicture}
 & \overset{\ref{eq:general_kapdag}}{=}
\begin{tikzpicture}
 \path (0,0) node (A) {}
 ++(0.3,0) coordinate (copy0)
 ++(0.5,0) node[kernel] (mu) {$\mu$}
 ++(1,0) node[kernel] (dagmumu) {$^\dagger\mu\mu$}
 ++(0.4,0) coordinate (copy1)
 ++(0.5,-0.5) node[kernel] (kapsta) {$\kappa^*$}
 +(1,0.5) node (E) {}
 +(1,0) node (D) {}
 +(1,1) node (T) {};
 \draw (A)--(mu) -- (dagmumu) -- (copy1);
 \draw (copy0) to [bend left] (T);
 \draw (copy1) to [bend right] (kapsta) (copy1) to [bend left] (E);
 \draw (kapsta) -- (D);
\end{tikzpicture}\\
&=
\begin{tikzpicture}
 \path (0,0) node (A) {}
 ++(0.3,0) coordinate (copy0)
 ++(0.7,0) node[kernel] (mudagmu) {$\mu^\dagger\mu$}
 ++(0.7,0) coordinate (copy1)
 ++(0.5,0) node[kernel] (mu) {$\mu$}
 ++(0.5,0) node[kernel] (kapsta0) {$\kappa^*$}
 ++(0.5,0) node[kernel] (kap) {$\kappa$}
 ++(0.4,0) coordinate (copy2)
 ++(0.5,-0.5) node[kernel] (kapsta1) {$\kappa^*$}
 +(1,0.5) node (E) {}
 +(1,0) node (D) {}
 +(1,1	) node (T) {};
 \draw (A)--(mudagmu) -- (mu) -- (kapsta0) -- (kap) -- (copy2);
 \draw (copy0) to [bend left] (T);
 \draw (copy1) to [bend left = 40] (kap);
 \draw (copy2) to [bend right] (kapsta1) (copy2) to [bend left] (E);
 \draw (kapsta1) -- (D);
\end{tikzpicture}\\
&\overset{\ref{eq:general_kapdag}}{=} 
\begin{tikzpicture}
 \path (0,0) node (A) {}
 ++(0.3,0) coordinate (copy0)
 ++(0.7,0) node[kernel] (mudagmu) {$\mu^\dagger\mu$}
 ++(0.7,0) coordinate (copy1)
 ++(0.5,0) node[kernel] (mu) {$\mu$}
 ++(0.5,0) node[kernel] (kapsta0) {$\kappa^*$}
 ++(0.4,0) coordinate (copy2)
 ++(0.5,0) node[kernel] (kap) {$\kappa$}
 +(1,0) node (E) {}
 +(1,-0.5) node (D) {}
 +(1,0.5) node (T) {};
 \draw (A)--(mudagmu) -- (mu) -- (kapsta0) -- (kap) -- (copy2);
 \draw (copy0) to [bend left] (T);
 \draw (copy1) to [bend left =40] (kap);
 \draw (copy2) to [bend right] (D) (kap) -- (E);
\end{tikzpicture}\\
&\overset{\ref{eq:rightinverse_commute}}{=}
\begin{tikzpicture}
 \path (0,0) node (A) {}
 ++(0.3,0) coordinate (copy0)
 ++(0.7,0) node[kernel] (mudagmu) {$\mu$}
 ++(0.7,0) coordinate (copy1)
 +(0.5,0.5) node[kernel] (dagmu) {$^\dagger\mu$}
 ++(0.5,0) node[kernel] (kapsta0) {$\kappa^*$}
 ++(0.4,0) coordinate (copy2)
 ++(0.5,0) node[kernel] (kap) {$\kappa$}
 +(1,0) node (E) {}
 +(1,-0.5) node (D) {}
 +(1,1) node (T) {};
 \draw (A)--(mudagmu) -- (kapsta0) -- (kap) -- (copy2);
 \draw (copy0) to [bend left] (T);
 \draw (copy1) to [bend left] (dagmu);
 \draw (dagmu) to [bend left] (kap);
 \draw (copy2) to [bend right] (D) (kap) -- (E);
\end{tikzpicture}\\
&\overset{\ref{eq:det_commute}}{=}
\begin{tikzpicture}
 \path (0,0) node (A) {}
 ++(0.3,0) coordinate (copy0)
 +(1.4,0.5) node[kernel] (dagmu) {$\mu^\dagger\mu$}
 ++(1.4,0) node[kernel] (kapsta0) {$\mu \kappa^*$}
 ++(0.4,0) coordinate (copy2)
 ++(0.5,0) node[kernel] (kap) {$\kappa$}
 +(1,0) node (E) {}
 +(1,-0.5) node (D) {}
 +(1,1) node (T) {};
 \draw (A)-- (kapsta0) -- (kap) -- (copy2);
 \draw (copy0) to [bend left=40] (T);
 \draw (copy0) to [bend left] (dagmu);
 \draw (dagmu) to [bend left] (kap);
 \draw (copy2) to [bend right] (D) (kap) -- (E);
\end{tikzpicture}\label{eq:skipmu}
\end{align}

% From \ref{eq:gamkap_2_mu} and \ref{eq:kappa_cancel} we have

% \begin{align}
% \begin{tikzpicture}
% 	\path (0,0) coordinate (A)
% 	+  (0,0.15) coordinate (copy0)
% 	++ (0,0.5) node[kernel] (mu) {$\mu$}
% 	++ (0,0.5) node[kernel] (kapdag) {$\kappa^{\dagger}$}
% 	++ (0,0.5) coordinate (copy1)
% 	+  (0.6,0.5) node[kernel] (kap) {$\kappa$}
% 	+  (-0.5,0.5) coordinate (B) 
% 	+  (0.6	,1) node (E) {$\RV{E}$}
% 	+  (-0.5,1) node (D) {$\RV{D}$};
% 	\draw (A) -- (mu);
% 	\draw (mu) -- (kapdag);
% 	\draw (kapdag) -- (copy1);
% 	\draw (copy1) to [bend left] (B);
% 	\draw (copy1) to [bend right]  (kap);
% 	\draw (kap) to (E);
% 	\draw (B) to (D);
% 	\draw (copy0) to [bend right] (kap);
% \end{tikzpicture} &= 
% \begin{tikzpicture}
% 	\path (0,0) coordinate (A)
% 	++ (0,0.25) coordinate (copy0)
% 	++ (0,0.5) node[kernel] (gam) {$\gamma$}
% 	++ (0,0.5) coordinate (copy1)
% 	+ (0.5,0.5) node[kernel] (kap) {$\kappa$}
% 	+ (-0.5,0.5) coordinate (B)
% 	+ (0.5,1) node (E) {$\RV{E}$}
% 	+ (-0.5,1) node (D) {$\RV{D}$};
% 	\draw (A) -- (copy0);
% 	\draw (copy0) -- (gam);
% 	\draw (copy0) to [bend right] (kap);
% 	\draw (gam) -- (copy1);
% 	\draw (copy1) to [bend left] (B);
% 	\draw (copy1) to [bend right] (kap);
% 	\draw (kap) to (E);
% 	\draw (B) to (D);
% \end{tikzpicture}
% \end{align}

% Note that, given some prior $\xi\in \Delta(\mathscr{T})$, the ``Bayes result'' of invoking some decision rule $J:E\to \Delta(\mathcal{D})$ is given by

% \begin{align}
% \begin{tikzpicture}
%  \path (0,0) node[dist] (A) {$\xi$}
%  ++(0,0.3) coordinate (copy0)
%  ++(0,0.5) node[kernel] (mu) {$\mu$}
%  ++(0,0.5) node[kernel] (kapdag) {$J$}
%  ++(0,0.7) node[kernel] (kap) {$\kappa$}
%  ++(0,0.5) node (E) {$\RV{E}$};
%  \draw (A)--(mu) -- (kapdag) -- (kap) -- (E);
%  \draw (copy0) to [bend right = 45] (kap);
% \end{tikzpicture}
% \end{align}

A key question is does \ref{eq:skipmu} imply anything non-trivial regarding the following ``identifiability'' condition for arbitrary $J:E\to \Delta(\mathcal{D})$:

\begin{align}
\begin{tikzpicture}
 \path (0,0) node (A) {}
 ++(0,0.3) coordinate (copy0)
 ++(0,0.5) node[kernel] (mu) {$\mu$}
 ++(0,0.5) node[kernel] (kapdag) {$J$}
 ++(0,0.7) node[kernel] (kap) {$\kappa$}
 ++(0,0.5) node (E) {$\RV{E}$};
 \draw (A)--(mu) -- (kapdag) -- (kap) -- (E);
 \draw (copy0) to [bend right = 45] (kap);
\end{tikzpicture} =
\begin{tikzpicture}
\path (0,0) node (xi) {}
++ (0,0.5) node[kernel] (mu0) {$\mu$}
++ (0,0.5) coordinate (copy0)
+ (0.5,0.5) node[kernel] (dmu) {$^\dagger\mu$}
++(0,0.5) node[kernel] (kapd) {$J$}
++(0,0.5) node[kernel] (kap) {$\kappa$}
++(0,0.5) node (E) {$\RV{E}$};
\draw (xi) -- (mu0) -- (kapd) -- (kap) -- (E);
\draw (copy0) to [bend right] (dmu) (dmu) -- (kap);
\end{tikzpicture}\label{eq:identifiability}
\end{align}

I call this identifiability because the left hand side is the kernel $\mathscr{T}\to \Delta(\mathcal{E})$ that computes the ``result'' of a given state and decision function, while the right hand side implies it is possible to find a $J$ that minimises the expected utility $\kappa u$ independent of the causal state (this is because we see one of the input wires and control the other).

We may be able to get some insight into this by asking, given matrices $A,B,C,D$ of appropriate shapes, if $BA=CA$ when does $BDA=CDA$?

% Given these two assumptions, we have
% \begin{align}
% 	\begin{tikzpicture}[scale=0.8]
% 		\path (0,0) node[dist] (A) {$\mu$}
% 		++(0,1) coordinate (C)
% 		+(1,2) node (F) {$*$}
% 		++(-1,1) node[kernel] (E) {$\kappa^*$}
% 		++(0,1) coordinate (G)
% 		+(1,0.5) node[kernel] (D) {$\kappa$}
% 		+(0,1) coordinate (H)
% 		+(1,1) coordinate (I);
% 		\draw (A) -- (C);
% 		\draw (C) to [bend right] (F.center);
% 		\draw (C) to [bend left] (E);
% 		\draw (E) -- (G);
% 		\draw (G) -- (H);
% 		\draw (G) to [bend right] (D);
% 		\draw (D) to (I);
% 	\end{tikzpicture}&=
% 		\begin{tikzpicture}[scale=0.8]
% 		\path (0,0) node[dist] (A) {$\mu$}
% 		++(0,1) node[kernel] (E) {$\kappa^*$}
% 		++(0,0.5) coordinate (G)
% 		+(1,0.5) node[kernel] (D) {$\kappa$}
% 		+(-1,1) coordinate (H)
% 		+(1,1) coordinate (I);
% 		\draw (A) -- (E);
% 		\draw (E) -- (G);
% 		\draw (G) to [bend left] (H);
% 		\draw (G) to [bend right] (D);
% 		\draw (D) to (I);
% 	\end{tikzpicture}\\
% 	&=
% 		\begin{tikzpicture}[scale=0.8]
% 		\path (0,0) node[dist] (A) {$\mu$}
% 		++(0,1) coordinate (G)
% 		+(1,1) node[kernel] (B) {$\kappa^*$}
% 		+(-1,1) node[kernel] (C) {$\kappa^*$}
% 		+(1,2) node[kernel] (D) {$\kappa$}
% 		+(-1,3) coordinate (H)
% 		+(1,3) coordinate (I);
% 		\draw (A) -- (G);
% 		\draw (G) to [bend left] (C);
% 		\draw (G) to [bend right] (B);
% 		\draw (C) to (H);
% 		\draw (B) to (D);
% 		\draw (D) to (I);
% 	\end{tikzpicture}\\
% 	&=
% 	\begin{tikzpicture}[scale=0.8]
% 		\path (0,0) node[dist] (A) {$\mu$}
% 		++(0,1) coordinate (B) 
% 		+(1,2) coordinate (C)
% 		+(-1,1) node[kernel] (D) {$\kappa^*$}
% 		+(-1,2) coordinate (E);
% 		\draw (A) -- (B);
% 		\draw (B) to [bend right] (C);
% 		\draw (B) to [bend left] (D);
% 		\draw (D) -- (E);
% 	\end{tikzpicture}\\
% 	&:= \mu^{\dagger|}
% \end{align}

% Here we recall the definition of a conditional probability: a conditional probability \inkernel{\RV{Y}|\RV{X}}{}{} with respect to $\nu\in \Delta(\mathcal{E})$ and RVs $\RV{X}:E\to X$, $\RV{Y}:E\to Y$ is any Markov kernel $X\to \Delta(\mathcal{Y})$ such that
% \begin{align}
% 	\begin{tikzpicture}[scale=0.7]
% 	\path (0,0) node[dist,inner sep=2pt] (A) {$\nu$}
% 	+ (0.5,1) node (B) {$X$}
% 	+ (-0.5,1) node (C) {$Y$};
% 	\draw (B) -- (B|- A.north);
% 	\draw (C) -- (C|- A.north);
% 	\end{tikzpicture} = 
% 	\begin{tikzpicture}[scale=0.7]
% 	\path (0,0) node[dist,inner sep=2pt] (A) {$\nu$}
% 	+ (-0.5,1) node (C) {$*$}
% 	++ (0.5,1) coordinate (B)
% 	+ (-1,1) node[kernel] (D) {$\RV{Y}|\RV{X}$}
% 	+(0,2) node (E) {$X$}
% 	+(-1,2) node (F) {$Y$};
% 	\draw (B) -- (B|- A.north);
% 	\draw (C.center) -- (C|- A.north);
% 	\draw (B) -- (E);
% 	\draw (B) to [bend left] (D);
% 	\draw (D) -- (F);
% 	\end{tikzpicture}
% \end{align}

% Thus we can see that $\kappa$ is a conditional probability \inkernel{\RV{E}|\RV{D}}{}{} with respect to $\mu^{\dagger|}$ the RVs $\RV{D}:E\times D\to D$ and $\RV{E}:E\times D\to E$ given by the respective projections.

% This result is in line with two familiar cases from the causal literature: an \emph{ignorable} treatment in potential outcomes or a graphical model with no back-door paths between $\RV{D}$ and $\RV{E}$ both imply that the theory's respective version of ``causal effect'' can be determined directly from the conditional probability. This suggests that reproducibility is in fact quite a strong condition, as it is the exception rather than the norm where causal effects can be determined directly from conditional probabilities. There are two important caveats here, however: firstly we've assumed $\kappa$ has a known right-inverse $\kappa^*$, and secondly this result only allows $\kappa$ to be determined $\mu^{\dagger|}$-almost surely.

% \todo[inline]{A notion of ``approximate right-invertibility'' would be very helpful here. Something like $\kappa$ is approximately right invertible if there is some $\tilde{\kappa}^\dagger$ such that $\gamma \kappa\tilde{\kappa}^\dagger$ is always ``close'' to $\gamma$. Intuitively, a consequence that is \emph{not} approximately right invertible has a channel capacity of 0, and \textbf{Conjecture:} a set of kernels that do not collectively admit some approximate right-inverse do not admit preferences between decisions under the minimax rule}

% \section{More Complex Cases}

% Suppose we retain the right invertibility of $\kappa$ via some known $\kappa^*$, which is consistent with (for example) hard interventions and counterfactual variable forcing. A more common way that there fails to be a memoryless function $J_\theta$ such that $J_\theta\kappa=\mu$ arises from the fact that we may often regard the consequence as dynamic. Assign an index $\theta\in \Theta$ to each causal state, which we now denote $(\kappa_\theta,\mu_\theta)\in \mathscr{T}$ and let $\kappa^\Theta:\Theta\times D\to \Delta(\mathcal{E})$ be defined by $\kappa^\Theta:(\theta,y;A)\mapsto \kappa_\theta(y;A)$ which we will assume is also a Markov kernel (noting that this intoduces additional restrictions on $\mathscr{T}$ in the general case). We may assume, then, for some state-dependent generating distribution $\xi_\theta\in \Delta(\Theta)$ and some state dependent ``passive decision map'' $K_\theta:\Theta\to \Delta(\mathcal{D})$, we have $\mu_\theta = \xi_\theta\splitter{0.1}(I_{(\Theta)}\otimes K_\theta)\kappa^\Theta$. 

% \indist{a}{b}

% \inkernel{ab}{a}{b}