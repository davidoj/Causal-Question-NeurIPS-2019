\section{Causal Bayesian Networks}

A causal Bayesian network is an example of a causal theory. A CBN is a non-trivial setup, so we need to define a number of moving parts to make the connection. Given a measurable space $(E,\mathcal{E})$ and a decision space $(D,\mathcal{D})$, a CBN is a causal theory $\Delta(\mathcal{E})\to (D\to \Delta(\mathcal{E}))$.

Suppose we have a graph $\mathcal{G}=(V,E)$ where $V=\{V_i|i\in [N]\}$.

Take some sequence of variables $\RV{X}:(E,\mathcal{E})\to X$ where $\RV{X} = \otimes\underline[\RV{X}_j]$, $j\in \mathbb{N}$, and suppose that they are IID with respect to both $\mathscr{H}$ and $\mathscr{T}$ (that is, they are IID for all $\mu\in\mathscr{H}$ and all  $\tau(\mu)(d;\cdot)$ for $\tau \in \mathscr{T}$). Pick some $j\in \mathbb{N}$ and omit the subscript henceforth: $\RV{X}_j:=\RV{X}$. Suppose $\RV{X} = \otimes_{i\in[N]}\underline[\RV{X}^i]$. Take $\RV{X}^i:(E,\mathcal{E})\to \mathbb{R}$. The $\RV{X}^i$ represent the variables in a regular CBN.

For each $i\in [N]$ define a random variable $\RV{D}^i:D\to \mathbb{R}\cup\{*\}$. The variable $\RV{D}^i$ represents a do-intervention on $\RV{X}^i$.

A causal theory $\tau$ must satisfy three conditions in order to be a CBN with respect to $\mathcal{G}$.

\begin{itemize}
    \item For $x\in \mathbb{R}$, $\mu\in \mathscr{H}$, $\xi\in \Delta(\mathcal{D})$,  $P^{\xi \tau^\mu}(\RV{X}^i|\RV{D}^i=x)=\delta_x(\RV{X}^i)$ ($\RV{D}^i=x\in \mathbb{R}$ represents an active intervention on $\RV{X}^i$)
    \item $P^{\xi \tau^\mu}(\RV{X}^i|\RV{D}_i=*,\PA{\mathcal{G}}{\RV{X}^i}) = P^\mu(\RV{X}^i|\PA{\mathcal{G}}{\RV{X}^i})$ ($\RV{D}^i=*$ represents a passive intervention on $\RV{X}^i$)
    \item For all $\mu\in \mathscr{H}$, $d\in D$, $P^{\delta_d \tau^\mu}(\RV{X})$ is Markov with respect to $\mathcal{G}$
\end{itemize}



\subsection{Supergraph equivalence}

\section{Single World Intervetion Graphs}

\section{Potential Outcomes}