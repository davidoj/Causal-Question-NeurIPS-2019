%!TEX root = main.tex

\section{Causal Bayesian Networks}

Suppose we have a set of ``interventions'' $R$ which factorises as $R=\otimes_{i\in [n]} \{\#\}\cup X^i$ for some $n\in \mathbb{N}$, collection of sets $\{X^i\}_{i\in [n]}$ and distinguished element $*\not\in R^i$ for any $i$. Suppose we also have a measurable space $E$ and set of random variables $\{\RV{X}^i|i\in \mathbb{N}\}$ such that $\RV{X}^i:E\to X^i$. We denote an element $(x^0,\#,...,\#,x^n)\in R$, $x^0,x^n\neq \#$ by the notation $do(\RV{X}^0=x^0,\RV{X}^n=x^n)$ where occurrences of the distinguished element $*$ are ommitted. Denote by $\underline{\#}$ the element of $R$ consisting entirely of $\#$ (equivalently, $do()$).

For $n\in \mathbb{N}$, directed acyclic graph (DAG) of degree $n$ is a graph $\mathcal{G}=(V,A)$ where $V$ is a set of vertices such that $|V|=n$ and $A\subset V\times V$ is a set of directed edges (``arrows'') such that $A$ induces no cycles (for a definition of cycles see \citet{pearl_causality:_2009}). 

Strictly, we are considering labeled graphs $\mathcal{G}$ and sets $\{\RV{X}^i\}_{[n]}$ of random variables. That is, we have bijective functions $f:V\to [n]$ and $g:\{\RV{X}^i\}_{[n]}\to [n]$ and we adopt the convention that $f(i):=V^i$ and $g(i):=\RV{X}^i$. In addition, we will sometimes let a set $U\subset V$ or $a\subset[n]$ to denote a set of random variables rather than vertices or natural numbers; this is licenced by the bijections $f$ and $g$. 

We also suppose we have surjective $h:R\to \mathscr{P}([n])$ such that $h:(x^0,...,x^n)\mapsto \{i|x^i\neq *\}$. That is, $h$ picks out the indices that aren't suppressed in the $do(...)$ notation for elements of $V$. Define $\RV{X}^{i\prime}:R\to \{\#\}\cup X^i$ by the function returning the $i$-th element of $r$ for $r\in R$. Again, we suppose we have a bijection between primed random variables and natural numbers and can therefore pick out corresponding sets of primed RVs, unprimed RVs, nodes and natural numbers.

\begin{definition}[Causal Bayesian Network]\label{def:CBN}

Given $R$, $E$ and $P_*:R\to \Delta(\mathcal{E})$ and $\{\RV{X}^i\}_{i\in [n]}$, a Causal Bayesian Network (CBN) compatible with $P_*$ is a directed acyclic graph (DAG) $\mathcal{G}$ of degree $n$ such that for all $r\in R$

\begin{enumerate}
    \item $P_*(r)$ is compatible with $\mathcal{G}$ (see \citet{pearl_causality:_2009})
    \item For all $i\in h(r)$, $P_*(r)F_{\RV{X}^i}=\delta_{\RV{X}^{i\prime}(r)} F_{\RV{X}^i}$
    \item For all $i\not \in h(r)$, $P_*(r)_{|\PA{\mathcal{G}}{\RV{X}^i}} F_{\RV{X}^i}=P_*(\underline{\#})_{|\PA{\mathcal{G}}{\RV{X}^i}}F_{\RV{X}^i} $, $P_*(\underline{\#})$-almost surely
\end{enumerate}
\end{definition}

Condition 3 presents some difficulties in the presence of measure 0 sets, as when a conditional probability such as $P_*(\#)_{|\PA{\mathcal{G}}{\RV{V}^i}}$ (or, using more common notation, $P_{\#}(\cdot|\PA{\mathcal{G}}{\RV{V}^i})$) may be variously intended to mean a particular representative of the class of conditional probabilities, an arbitrary representative or the entire class; condition 3 will have different implications for these various interpretations. If we interpret it as the entire class, for example, then any $P_*$ such that there is some $\RV{X}^i$ with nonempty parents $P_*(\#) F_{\PA{\mathcal{G}}{\RV{X}^i}$ assigns measure 0 to any set has no compatible CBN. \todo{Surely Pearl has dealt with this? Somewhere? I have looked...}

At least in the case of discrete $E$ and $P_*(\#)$ positive definite, we have from this definition that for any $r\in V$ we have the \emph{truncated factorisation} property:
\begin{align}
S	P_* F_{\mathbf{X}}(r;A) = \prod_{i\in h(r)} \delta\sum_{(x^0,...,x^n)\in X} P_* F
\end{align}

So far, this is a standard definition of a CBN; the extra additions are making explicit some implicit elements of the definition found in \citet{pearl_causality:_2009}.

As a consequence of (existence of conditional probability), given $\mathcal{G}$ and $P_*(\#) there exists a unique set of interventional distributions $P_*(r)$, $r\in V$ rendering $\mathcal{G}$ compatible.