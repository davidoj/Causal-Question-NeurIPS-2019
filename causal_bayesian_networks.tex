\section{Causal Bayesian Networks}

A causal Bayesian network can be understood to be a causal theory. The CBN convention is to call the elements of the decision space $D$ ``interventions'' and denote then with $do()$ notation. Given a random variable $\RV{X}^i$ on the output space $(F,\mathcal{F})\to (X^i,\mathcal{X}^i)$, we identify the intervention $do(\RV{X}^i=x)$ with an element $x\in X^i$ and the absence of any do intervention with the special ``passive'' element $*$.

\begin{definition}[Causal Bayesian Network]\label{def:CBN}
The definition here follows \cite{pearl_causality:_2009}.

Consider a directed acyclic graph $\mathcal{G}$ with edges $\mathbf{V}=\{\RV{X}^i|i\in [N]\}$, a measurable space $(F,\mathcal{F})$ and a set of random variables $\RV{X}^i:F\to X^i$ and let $X=\cup_{i\in[N]} X^i$. For each $x\in X\cup\{*\}$ suppose we have an \emph{interventional distribution} $P^{x}_\RV{X}$, and let the set of all such distributions be denoted $\mathbf{P}^{X\cup\{*\}}$. Let $P^*_{\RV{X}}$ be the passive distribution given by the intervention $x = (*,...,*)$.

Given any $x\in X\cup\{*\}$ let $S\subset[N]$ be the set of all indices $i$ such that $x^i\neq *$. The graph $\mathcal{G}$ is a causal Bayesian network compatible with $\mathbf{P}^{X\cup\{*\}}$ iff for all $x\in X$ and $S\subset [N]$:
\begin{enumerate}
    \item $P^{x}_{\RV{X}}$ is compatible with $\mathcal{G}$ for all $x\in X\cup\{*\}$
    \item $P^x_{\RV{X}}(\RV{X}^S)=\delta_{x^S}(\RV{X}^S)$
    \item For $i\in S^C$, $P^x_{\RV{X}}(\RV{X}^i|\PA{\mathcal{G}}{\RV{X}^i})=P^*_\RV{X}(\RV{X}^i|\PA{\mathcal{G}}{\RV{X}^i})$, $P^x$-almost surely
\end{enumerate}
\end{definition}

Taking $(F,\mathcal{F})=(X^{\mathbb{N}},\mathcal{X}^{\mathbb{N}})$ and considering only distributions where the sequence $\RV{X}_0,\RV{X}_1,..$ is IID, the above three conditions are sufficient that, given some graph $\mathcal{G}$ and $P^*_{\RV{X}}\in \Delta(\mathcal{X})$, one obtains a unique set of interventional distributions $\mathbf{P}^{X\cup\{*\}}$ (this follows from the truncated factorisation property given by \cite{pearl_causality:_2009}). 

Taking the map $\kappa: x\mapsto P^x$ as a consequence, a graph $\mathcal{G}$ therefore defines the causal theory $P^*\mapsto \kappa$. Theorem \ref{th:cbn_MK} establishes that the object $\kappa$ is a Markov kernel if $\mathcal{G}$ has a finite number of nodes, so it is strictly correct to consider this map a causal theory.

\begin{theorem}\label{th:cbn_MK}
Given a graph $\mathcal{G}$ and a set $\mathscr{H}\subset\Delta{\mathcal{X}}$ of probability distributions compatible with $\mathcal{G}$, the map $\tau_{\mathcal{G}}:\mathscr{H}\to (D\to \mathscr{H})$ given by $\mu\mapsto (x\mapsto P^x)$ for $\mu\in \Delta(X)$ is a causal theory.
\end{theorem}

The proof is given in Appendix \ref{app:cbn_ct}.

\subsection{Supergraph equivalence}

A CBN defines a causal theory, so naturally a set of CBNs defines a causal prospect. 

It is well known that in Definition \ref{def:CBN}, condition 1 is vacuous if $\mathcal{G}$ is a fully connected graph. For a fully connected graph, therefore, conditions 2 and 3 are sufficient to define the causal theory associated with $\mathcal{G}$. 



Theorem \ref{th:sup_equiv} establishes that in fact, a set of fully connected graphs and hence conditions 2 and 3 alone are enough to specify any causal prospect (that is, any set of causal theories) that can be specified by a set of causal Bayesian networks.

This has consequences for the practice of learning causal graphs. In particular, any algorithm that learns to delete edges from a set of graphs 

\begin{theorem}[Supergraph equivalence]\label{th:sup_equiv}

\end{theorem}

\section{Single World Intervetion Graphs}

\section{Potential Outcomes}

The definition of 