%!TEX root = main.tex

\section{No causes in no causes out}

A key result in statistical learning theory is the requirement that, in order for a hypothesis class to be learnable, it must have finite VC-dimension. The concept of controlling the size of the hypothesis class plays a fundamental role across the field of machine learning, from formal proofs of learnability to techniques based less formally on the notion of the bias-variance tradeoff. CSDPs are closely related to statistical learning problems, and it is highly likely that results of this type can be developed for causal problems.

For a hypothesis class of functions to be learnable, we can make assumptions that do not require the class of functions to collectively have a tendency to treat certain points differently to other points. For example, a class of functions limited by VC-dimension is closed under the operation of precomposition with any invertible function on the class' shared domain - the union of images of any set of points over the entire class depends only on the size of the set of points and not on which particular points are represented. However, if a causal theory is closed under the analogous operation, or if we assume a prior that is invariant under this operation, we must prefer the uniform decision strategy or find no preference over strategies whatsoever.

\begin{theorem}[No causes in, no causes out (Bayes)]\label{th:ncinco}
If a causal theory $T:\Theta\times D\to \Delta(\mathcal{E}\otimes\mathcal{F})$ and a prior $\xi\in \Delta(\Theta)$ are such that for all invertible $U:D\to \Delta(\mathcal{D})$, $(\xi\otimes U)T = (\xi\otimes I)T$ then all decision strategies are Bayes.
\end{theorem}

\begin{proof}
Consider the stubborn strategy $J_d:x\mapsto \delta_d$ for all $x\in E$. Define $U_{dd'}:D\to \Delta(\mathcal{D})$ to be the swap map that sends $d\to \delta_{d'}$, $d'\to \delta_{d}$ and leaves other decisions the same. Clearly, $U_{dd'} U_{dd'} = \mathrm{Id}_D$. For all $d,d'\in D$
\begin{align}
	\int_\Theta \delta_d C_\theta u d\xi &= \int_\Theta \delta_d U_{dd'} C_\theta u d\xi\\
										&= \int_\Theta \delta_{d'} C_\theta u d\xi\\
										 &:= S_0
\end{align}
Thus for any $J\in \Delta(\mathcal{D})^E$,
\begin{align}
	\int_\Theta \mu_\theta J C_\theta u d\xi &= \int_\theta \int_D \delta_d C_\theta u d\gamma d\mu_\theta J\\
											 &= \int_D \int_\theta \delta_d C_\theta u d\gamma d\mu_\theta J\\
											 &= S_0
\end{align}
\end{proof}

Somewhat surprisingly, the minimax rule may yield preferences over decisions under such circumstances; in particular, a uniform strategy is always minimax, though other strategies may not be.

\begin{theorem}[No causes in, uniform strategy out (minimax)]
If a causal theory $T:\Theta\times D\to \Delta(\mathcal{E}\otimes\mathcal{F})$ with finite $D$ is such that for every invertible $U:D\to \Delta(\mathcal{D})$, $\theta\in \Theta$ there is some $\theta'$ such that $T_{\theta,\cdot} = (I\otimes U)T_{\theta',\cdot}$ then the uniform decision strategy is minimax.
\end{theorem}

\begin{proof}
Call $J_U$ the stubborn uniform strategy $J_U:x\mapsto U(\mathcal{D})$ for all $x\in E$. Suppose there is some nonuniform $J$ such that $\max_\theta S(J,\theta) < \max_\theta S(J_U,\theta)$. Suppose $S(J_U,\theta)$ is maximised in some state $\theta^0$ where $S(J_d,\theta^0)=S(J_{d'},\theta^0)$ for all $d,d'\in D$. Then $S(J,\theta^0)=S(J_U,\theta^0)$, contraticting our assumption that $J$ achieved lower risk in the worst case. Suppose $S(J_U,\theta)$ is maximised in some state $\theta^1$ where there are some $d,d'\in D$ such that $S(J_d,\theta^1)>S(J_{d'},\theta^1)$. Then there are most $|D|/2$ decisions where $S(J_d,\theta^1)$ is greater than the median of $A=\{S(J_d,\theta^1)|d\in D\}$ and at least one such decision, and at least $|D|/2$ decisions such that $\mu_{\theta^1} J(d)$ is greater than or equal to the median of $B=\{\mu_{\theta^1} J(d)|d\in D\}$, with at least one strictly greater. Thus there is an invertible map $f:D\to D$ such that $f(A)\subset B$. But then there is some $\theta^2$ such that $S(J_d,\theta^1)=S(J_{f(d)},\theta^2)$ for all $d\in D$ and thus $S(J,\theta^2)> S(J_U,\theta^2) = S(J_U,\theta^1)$ contradicting our assumption that $J$ was better by the minimax rule than $J_U$.
\end{proof}

\begin{corollary}
If the risk of the uniform strategy is maximised in some state $\theta^*$ such that $S(J_d,\theta^*)>S(J_{d'},\theta^*)$ for some $d,d'$, then the uniform strategy is strictly better than any nonuniform strategy.
\end{corollary}

While there are numerous differences between the Potential Outcomes and Causal Bayesian Network approach to causality, it is interesting to reflect on their different approaches to handling Theorem \ref{th:ncinco}. The CBN approach fixes a number of conditional probabilities among variables unless they are directly intervened on and combines this with a standard ``hard-intervention'' operation. Potential outcomes, in the form we discuss here, represents by potential outcomes a fixed set of properties of outcomes (``the science'', as Rubin calls it) which are then partially revealed by an assignment function which may respond in problem specific ways to decisions. As we have discussed regarding ETT, the potential outcomes approach is capable of representing decisions that are known to have certain effects but we may be uncertain as to how exactly they achieve these effects, though this is not always enough to satisfactorily represent the problem of interest (see the discussion of ITT). The CBN approach features a hard transition between fixed flexible conditional probabilities - either an intervention is not on a node, in which case its probability conditional on its parents is fixed, or it is on that node in which case the conditional probability is vastly different. This doesn't appear to be ideal for representing uncertainty over how a decision might correspond to an intervention. In fact \emph{any} CBN with hard interventions can in fact represent any causal theory if we permit decisions to correspond to unknown, state-dependent mixtures of interventions (this reflects the fact that every joint probability distribution can be achieved with the right mixture of hard interventions on every node). The fact that we lose dependence on the graph suggests that a na\"ive approach to uncertainty over the decision bias may be too general. There are many versions of CBNs with generalised interventions that may address this issue.

It is interesting to consider whether there might be principles of causal inference that eschew the two part approach of fixing some underlying notion of ``the science'' and separately adding in some kind of decision bias. One could imagine, for example, a causal theory that posits that consequences minimise some combination of causally appropriate dissimilarity measures from the observational distribution and from a decision-dependent target distribution without any clear commitments to invariant principles of science. It's not obvious how we should construct such measures without appealing to some notion of the underlying science, but we regard it as an interesting question nonetheless.