\section{Markov Kernels}\label{app:markov_kernels}

This is an expanded version of Section \ref{sec:dfin} that explains some notation more thoroughly.

A measurable space $(E,\mathcal{E})$ is a set $E$ and a $\sigma$-algebra $\mathcal{E}\subset\mathcal{P}(\mathcal{E})$ containing the measurable sets. A probability measure $\mu\in \Delta(\mathcal{E})$ is a nonnegative map $\mathcal{E}\to[0,1]$ such that $\mu(\emptyset)=0$, $\mu(E)=1$ and for countable $\{E_i\}\in \mathcal{E}$, $\mu(\cup_i E_i) = \sum_i \mu(E_i)$.

We assume all measurable spaces discussed are standard. That is, they are isomorphic to either a subset of $\mathbb{N}$ with the discrete $\sigma$-algebra, or $\mathbb{R}$ with the Borel $\sigma$-algebra.

Given two measureable sets $(E,\mathcal{E})$ and $(F,\mathcal{F})$, a \emph{Markov kernel} $K$ is a map $E\times \mathcal{F} \to [0,1]$ where
\begin{enumerate}
    \item The map $x\mapsto K(x;B)$ is $\mathcal{E}$-measurable for every $B\in\mathcal{F}$
    \item The map $B\mapsto K(x;B)$ is a probability measure on $(F,\mathcal{F})$ for every $x\in E$ 
\end{enumerate}

Abusing notation somewhat, we will give Markov kernels the alternate type signature $K:E\to \Delta(\mathcal{F})$, noting that due to part 1 not every map with this type is a Markov kernel. We will sometimes write the set of Markov kernels of type $E\to \Delta(\mathcal{F})$ as $\Delta(\mathcal{F})^D$, noting again that given part 1, the set of Markov kernels of this type may be smaller than $\Delta(\mathcal{F})^D$.

If we have two random variables $\RV{X}:\_\to X$ and $\RV{Y}:\_\to Y$, the conditional probability $P(\RV{Y}|\RV{X})$ is a Markov kernel $X\to \Delta(\mathcal{Y})$. Formally, given $\mu\in \Delta(\mathcal{E})$ and a sub-$\sigma$-algebra $\mathcal{E}'\subset\mathcal{E}$, there is a Markov kernel $\mu_{|\mathcal{E}'}:E\to\Delta(\mathcal{E})$ such that for $A\in\mathcal{E}$ and $B\in \mathcal{E}'$, $\int_B \mu_{|\RV{E}'}(y;A) d\mu(y) = \mu(A\cap B)$. $\mu_{|\mathcal{E}'}$ is a \emph{conditional probability distribution} with respect to $\mathcal{E}'$. This result may not hold if $(E,\mathcal{E})$ is not a standard measureable space \citep{cinlar_probability_2011}.

Given a set of random variables $\mathbf{X}=\{\RV{X}^i\}_{i\in [N]}$ with domain $(E,\mathcal{E})$, $\mu_{|\mathbf{X}}:E\to \Delta(\mathcal{E})$ is a conditional probability distribution with respect to the $\sigma$-algebra generated by $\mathbf{X}$: $\sigma(\cup_{i\in[N]}\sigma(\mathcal{X}^i))$.  We will use this subscript notation rather than the more common bar notation (e.g. $\mu(\cdot|\mathbf{X})$) to express conditional probability from here onwards.

Two Markov kernels $K:E\to \Delta(\mathcal{F})$ and $K':E\to \Delta(\mathcal{F})$ are $\mu$-almost surely equivalent given $\mu\in \Delta(\mathcal{E})$ if
\begin{align}
    \int_A K(x;B) d\mu = \int_A K'(x;B) d\mu\qquad\forall A\in \mathcal{E}, B\in\mathcal{F}
\end{align}

\subsection{Operations with Markov kernels}

For the following, assume $K$ is a Markov kernel from $E\to \Delta(\mathcal{F})$, $K'$ a kernel $E\to \Delta(\mathcal{H})$, L is a Markov kernel $F\to \Delta(\mathcal{G})$, $\mu$ is a probability measure on $(E,\mathcal{E})$, $\nu$ is a probability measure on $(F,\mathcal{F})$ and $f$ is a nonnegative measurable function $F\to \mathbb{F}$.

The notation here borrows heavily from \cite{cinlar_probability_2011} and \cite{fong_causal_2013}.

\subsubsection{Kernel products}

The kernel-kernel product $KL$ is a Markov kernel $E\to \Delta(\mathcal{G})$ such that $KL(x;B):= \int_F K(x;dy) L(y;B),\qquad x\in E, B\in \mathcal{G}$.

The measure-kernel product of $\mu$ and $K$, $\mu K$ is a probability measure on $(F,\mathcal{F})$ such that $\mu K(B)=\int_E \mu(dx) K(x;B),\qquad B\in\mathcal{F}$. 

The kernel-function product $Kf$ is a nonnegative measurable function $E\to \mathbb{R}$ such that $Kf(x) := \int_F K(x;dy)f(y), \qquad x\in E$.

Kernel products are in general associative: $(KL)M=K(LM)$.

\subsubsection{Special kernels}

$I_{(E)}$ is a kernel $E\to \Delta(\mathcal{E})$ defined by $x\mapsto \delta_x$. It has the properties $\mu I_{(E)}=\mu$, $KI_{(F)} = K$, $I_{(E)} K = K$, $I_{(F)} f=f$.

$\splitter{0.15}_E$ is a kernel $E\to \Delta(\mathcal{E}\otimes\mathcal{E})$ defined by $x\mapsto \delta_{(x,x)}$. We will subsequently leave the space implicit. The symbol $\splitter{0.15}$ is pronounced ``splitter''.

Given $M:H\to \Delta(\mathcal{I})$, $K\otimes M$ is a Markov kernel $E\times H\to \Delta(\mathcal{F}\otimes\mathcal{I})$ where
\begin{align}
    K\otimes M(x,y;A\times B) := K(x;A) M(y;B)
\end{align}

Given $N:I\to \Delta(\mathcal{J})$, it can be verified that $(K\otimes M)(L\otimes N)=KL\otimes MN$.

$\splitter{0.15}(K\otimes K')$ is a Markov kernel $E\to \Delta(\mathcal{F}\otimes\mathcal{H})$ and

\begin{align}
    \splitter{0.15}(K\otimes K')(x;A\times B) &= \int_E K(x';A)K'(x'';B) \delta_{(x,x)} (dx'\times dx'')\\ 
                                              &= K(x;A)K'(x;B) \label{eq:outer_product}
\end{align}

We can overload notation to use $\splitter{0.15}(K\otimes K'\otimes K'')$ for the nested construction $\splitter{0.15}(K\otimes \splitter{0.15}(K'\otimes K''))$. 

Let $(*,\{\emptyset,*\})$ be an indiscrete measurable set. $\stopper{0.15}_E$ is a kernel $E\to \Delta(\{\emptyset,*\})$ defined by $x\mapsto \mathds{1}_*$. We have $\splitter{0.15}(I\otimes \stopper{0.15}) = I$. The symbol $\stopper{0.15}$ is pronounced ``stopper''.

Given some measurable function $g:E\to F$, the kernel $F_g:E\to \Delta(\mathcal{F})$ is defined by $x\mapsto \delta_{g(x)}$. It is easy to check that $F_g F_g = F_g$. For $\mu\in \Delta(\mathcal{E})$, the product $\mu F_g$ is the push forward measure $g_*\mu$.

\begin{align}
    \mu F_g (A) &= \int_E \delta_{g(x)}(A) d\mu\\
                &= \mu(g^{-1}(A))\\
                &= g_*\mu(A)
\end{align}

Given two random variables $\RV{X}:(E,\mathcal{E})\to (X,\mathcal{X})$ and $\RV{Y}:(E,\mathcal{E})\to (Y,\mathcal{Y})$, the product $\mu\splitter{0.15}(F_{\RV{X}}\otimes F_{\RV{Y}})$ is the joint distribution of $\RV{X}$ and $\RV{Y}$.

\begin{align}
    \mu \splitter{0.15}(F_{\RV{X}}\otimes F_{\RV{Y}}) (A, B) &= \int_E \delta_{\RV{X}(x)}(A) \delta_{\RV{Y}(x)}(B) d\mu \\
                        &= \mu(\RV{X}^{-1}(A)\cap \RV{Y}^{-1}(B))
\end{align}
