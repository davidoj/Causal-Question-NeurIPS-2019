%!TEX root = main.tex

\section{Invariance and Capital C Causality}

CSDT features \emph{consequences} - that is, probabilistic relations between decisions and results - but it does not feature \emph{causal effects}, which seem to be probabilistic relations between random variables on the observation space $E$ that are not necessarily disintegrations of a joint distribution. Here is a provisional account of how causal effects might arise in CSDT.

Suppose $\mathbf{C}:D\times \Theta\to\Delta(\mathcal{E})$ is the consequence of interest, and furthermore that given some $\theta\in\Theta$, the observations are distributed according to
\begin{align}
	\begin{tikzpicture}
		\path (0,0) coordinate (S)
		++ (0.5,0) coordinate (copy0)
		++ (0.5,0) node[kernel] (C) {$\mathbf{C}_\theta$}
		+  (0,1) node[kernel] (A) {$\mathbf{A}_\theta$}
		++ (0.5,0) coordinate (copy1)
		++ (0.5,0) node[kernel] (B) {$\mathbf{B}$}
		+  (0,0.5) node[kernel] (Y) {$\mathbf{Y}$}
		++ (0.5,0) coordinate (Bout)
		+  (0,1) coordinate (Aout)
		+  (0,0.5) coordinate (Yout);
		\draw (S) -- (C) -- (B) -- (Bout);
		\draw (copy0) to [bend left] (A) (A) -- (Aout);
		\draw (copy1) to [bend left] (Y) (Y) -- (Yout);
	\end{tikzpicture}
\end{align}

For all $\theta$, $\mathbf{A}_\theta$ is invertible. This is a strong assumption that will be relaxed \todo[inline]{at a later date.}

