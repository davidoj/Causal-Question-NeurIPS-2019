
\section{Appendix: Counterfactuals}\label{app:counfac}

A causal theory for Potential Outcomes is associated with a much larger hypothesis class than any causal theory that works only with distributions over observable variables. Theorems \ref{th:CSDP_ob_red} and \ref{th:CSDP_u_red} show that given any SCDP based on Potential Outcomes, provided that the potential outcome variables are unobserved and the utility does not depend on them, a reduced SCDP can be constructed by marginalising over potential outcomes. Potential outcomes are not universally excluded by this; there are some examples of problems where one does care about the values of potential outcome variables. The \emph{effect of treatment on the treated} (ETT) that depends on counterfactual quantities and has some relevance to decision preferences \cite{rubin_estimating_1974}, though it is controversial whether this dependence is necessary \cite{geneletti2007defining}. More straightforwardly, the legal standard of ``no harm but for the defendant's negligence'' does seem to invoke fundamentally counterfactual considerations \cite{pearl_causality:_2009}.

\begin{example}[Performance bias]\label{ex:nonst_distn}
Suppose we have a CSDP $\langle (\mathscr{T}, E), D, \RV{X}, (U,E) \rangle$ where the observed data $\RV{X}$ is from a randomised controlled trial (RCT), $\RV{Y}_0:E\to Y$ and $\RV{Y}_1:E\to Y$ are random variables representing a particular outcome of interest under no treatment and treatment respectively and $\RV{Y}:E\to Y$ represents the ``realised'' outcome of interest and for $\xi\in \Delta(\mathcal{E})$, $U(\xi) = \mathbb{E}_{\xi}[\RV{Y}]$. 

Under usual assumptions about RCTs, if we suppose the observed data are distributed according to $\mu \in \Delta(\mathcal{E})$ it is possible (given infinite data $\RV{X}$) to determine $\mathbb{E}_\mu[\RV{Y}_0]$ and $\mathbb{E}_\mu[\RV{Y}_1]$ \citep{rubin_causal_2005}.

Consequence consistency is assumed, but performance bias is suspected, which can lead to $\delta_i \kappa \RV{Y}_i$ differing from $\mathbb{E}_\mu[\RV{Y}_i]$ \citep{mansournia_biases_2017}.
\begin{enumerate}
    \item Assume performance bias is absent, so the theory must satisfy $\delta_i \kappa \RV{Y}_i = \mathbb{E}_\mu[\RV{Y}_i]$
    \item Assume performance bias has a uniform additive effect: the theory satisfies $\delta_i \kappa \RV{Y}_i = \mathbb{E}_\mu[\RV{Y}_i]+k$. In this case the average treatment effect can still be estimated from the data: $\delta_1 \kappa \RV{Y}_1 - \delta_0\kappa\RV{Y}_0 = \mathbb{E}[\RV{Y}_1]-\mathbb{E}[\RV{Y}_0]$ which may be sufficient to find a decision function minimising the risk
    \item Avoid assumptions about the effect of performance bias; the theory satisfies no particular relationship between $\mathbb{E}_\mu[\RV{Y}_i]$ and $\delta_i \kappa \RV{Y}_i$ and we may therefore expect preferred decision function to ignore the data
\end{enumerate}
The question of specifying this relationship arises naturally when we consider connecting Potential Outcomes to CSDPs. Nonetheless, the possibility of deviations from option 1 above are often treated as ``external to the causal problem''. For example,  \cite{mansournia_biases_2017} states:
\begin{quote}
    In this case, it might be more appropriate to say that the intention-to-treat effect from the trial is not generalizable or transportable to other settings rather than saying that it is “biased”
\end{quote}
\end{example}
